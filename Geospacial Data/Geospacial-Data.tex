\documentclass[]{article}
\usepackage{lmodern}
\usepackage{amssymb,amsmath}
\usepackage{ifxetex,ifluatex}
\usepackage{fixltx2e} % provides \textsubscript
\ifnum 0\ifxetex 1\fi\ifluatex 1\fi=0 % if pdftex
  \usepackage[T1]{fontenc}
  \usepackage[utf8]{inputenc}
\else % if luatex or xelatex
  \ifxetex
    \usepackage{mathspec}
  \else
    \usepackage{fontspec}
  \fi
  \defaultfontfeatures{Ligatures=TeX,Scale=MatchLowercase}
\fi
% use upquote if available, for straight quotes in verbatim environments
\IfFileExists{upquote.sty}{\usepackage{upquote}}{}
% use microtype if available
\IfFileExists{microtype.sty}{%
\usepackage{microtype}
\UseMicrotypeSet[protrusion]{basicmath} % disable protrusion for tt fonts
}{}
\usepackage[margin=1in]{geometry}
\usepackage{hyperref}
\hypersetup{unicode=true,
            pdftitle={Working With Geospacial Data},
            pdfauthor={Ken Harmon},
            pdfborder={0 0 0},
            breaklinks=true}
\urlstyle{same}  % don't use monospace font for urls
\usepackage{color}
\usepackage{fancyvrb}
\newcommand{\VerbBar}{|}
\newcommand{\VERB}{\Verb[commandchars=\\\{\}]}
\DefineVerbatimEnvironment{Highlighting}{Verbatim}{commandchars=\\\{\}}
% Add ',fontsize=\small' for more characters per line
\usepackage{framed}
\definecolor{shadecolor}{RGB}{248,248,248}
\newenvironment{Shaded}{\begin{snugshade}}{\end{snugshade}}
\newcommand{\AlertTok}[1]{\textcolor[rgb]{0.94,0.16,0.16}{#1}}
\newcommand{\AnnotationTok}[1]{\textcolor[rgb]{0.56,0.35,0.01}{\textbf{\textit{#1}}}}
\newcommand{\AttributeTok}[1]{\textcolor[rgb]{0.77,0.63,0.00}{#1}}
\newcommand{\BaseNTok}[1]{\textcolor[rgb]{0.00,0.00,0.81}{#1}}
\newcommand{\BuiltInTok}[1]{#1}
\newcommand{\CharTok}[1]{\textcolor[rgb]{0.31,0.60,0.02}{#1}}
\newcommand{\CommentTok}[1]{\textcolor[rgb]{0.56,0.35,0.01}{\textit{#1}}}
\newcommand{\CommentVarTok}[1]{\textcolor[rgb]{0.56,0.35,0.01}{\textbf{\textit{#1}}}}
\newcommand{\ConstantTok}[1]{\textcolor[rgb]{0.00,0.00,0.00}{#1}}
\newcommand{\ControlFlowTok}[1]{\textcolor[rgb]{0.13,0.29,0.53}{\textbf{#1}}}
\newcommand{\DataTypeTok}[1]{\textcolor[rgb]{0.13,0.29,0.53}{#1}}
\newcommand{\DecValTok}[1]{\textcolor[rgb]{0.00,0.00,0.81}{#1}}
\newcommand{\DocumentationTok}[1]{\textcolor[rgb]{0.56,0.35,0.01}{\textbf{\textit{#1}}}}
\newcommand{\ErrorTok}[1]{\textcolor[rgb]{0.64,0.00,0.00}{\textbf{#1}}}
\newcommand{\ExtensionTok}[1]{#1}
\newcommand{\FloatTok}[1]{\textcolor[rgb]{0.00,0.00,0.81}{#1}}
\newcommand{\FunctionTok}[1]{\textcolor[rgb]{0.00,0.00,0.00}{#1}}
\newcommand{\ImportTok}[1]{#1}
\newcommand{\InformationTok}[1]{\textcolor[rgb]{0.56,0.35,0.01}{\textbf{\textit{#1}}}}
\newcommand{\KeywordTok}[1]{\textcolor[rgb]{0.13,0.29,0.53}{\textbf{#1}}}
\newcommand{\NormalTok}[1]{#1}
\newcommand{\OperatorTok}[1]{\textcolor[rgb]{0.81,0.36,0.00}{\textbf{#1}}}
\newcommand{\OtherTok}[1]{\textcolor[rgb]{0.56,0.35,0.01}{#1}}
\newcommand{\PreprocessorTok}[1]{\textcolor[rgb]{0.56,0.35,0.01}{\textit{#1}}}
\newcommand{\RegionMarkerTok}[1]{#1}
\newcommand{\SpecialCharTok}[1]{\textcolor[rgb]{0.00,0.00,0.00}{#1}}
\newcommand{\SpecialStringTok}[1]{\textcolor[rgb]{0.31,0.60,0.02}{#1}}
\newcommand{\StringTok}[1]{\textcolor[rgb]{0.31,0.60,0.02}{#1}}
\newcommand{\VariableTok}[1]{\textcolor[rgb]{0.00,0.00,0.00}{#1}}
\newcommand{\VerbatimStringTok}[1]{\textcolor[rgb]{0.31,0.60,0.02}{#1}}
\newcommand{\WarningTok}[1]{\textcolor[rgb]{0.56,0.35,0.01}{\textbf{\textit{#1}}}}
\usepackage{graphicx,grffile}
\makeatletter
\def\maxwidth{\ifdim\Gin@nat@width>\linewidth\linewidth\else\Gin@nat@width\fi}
\def\maxheight{\ifdim\Gin@nat@height>\textheight\textheight\else\Gin@nat@height\fi}
\makeatother
% Scale images if necessary, so that they will not overflow the page
% margins by default, and it is still possible to overwrite the defaults
% using explicit options in \includegraphics[width, height, ...]{}
\setkeys{Gin}{width=\maxwidth,height=\maxheight,keepaspectratio}
\IfFileExists{parskip.sty}{%
\usepackage{parskip}
}{% else
\setlength{\parindent}{0pt}
\setlength{\parskip}{6pt plus 2pt minus 1pt}
}
\setlength{\emergencystretch}{3em}  % prevent overfull lines
\providecommand{\tightlist}{%
  \setlength{\itemsep}{0pt}\setlength{\parskip}{0pt}}
\setcounter{secnumdepth}{0}
% Redefines (sub)paragraphs to behave more like sections
\ifx\paragraph\undefined\else
\let\oldparagraph\paragraph
\renewcommand{\paragraph}[1]{\oldparagraph{#1}\mbox{}}
\fi
\ifx\subparagraph\undefined\else
\let\oldsubparagraph\subparagraph
\renewcommand{\subparagraph}[1]{\oldsubparagraph{#1}\mbox{}}
\fi

%%% Use protect on footnotes to avoid problems with footnotes in titles
\let\rmarkdownfootnote\footnote%
\def\footnote{\protect\rmarkdownfootnote}

%%% Change title format to be more compact
\usepackage{titling}

% Create subtitle command for use in maketitle
\providecommand{\subtitle}[1]{
  \posttitle{
    \begin{center}\large#1\end{center}
    }
}

\setlength{\droptitle}{-2em}

  \title{Working With Geospacial Data}
    \pretitle{\vspace{\droptitle}\centering\huge}
  \posttitle{\par}
    \author{Ken Harmon}
    \preauthor{\centering\large\emph}
  \postauthor{\par}
      \predate{\centering\large\emph}
  \postdate{\par}
    \date{2019 October 23}


\begin{document}
\maketitle

\hypertarget{section}{%
\section{}\label{section}}

\url{https://campus.datacamp.com/courses/working-with-geospatial-data-in-r}
\url{https://github.com/cwickham/geospatial}

\hypertarget{google-mapping-api}{%
\subsection{Google Mapping API}\label{google-mapping-api}}

\url{https://cloud.google.com/maps-platform/}

\hypertarget{basic-mapping}{%
\subsection{Basic Mapping}\label{basic-mapping}}

Grabbing a background map There are two steps to adding a map to a
ggplot2 plot with ggmap:

Download a map using get\_map() Display the map using ggmap() As an
example, let's grab a map for New York City:

library(ggmap)

nyc \textless- c(lon = -74.0059, lat = 40.7128) nyc\_map \textless-
get\_map(location = nyc, zoom = 10) get\_map() has a number of arguments
that control what kind of map to get, but for now you'll mostly stick
with the defaults. The most important argument is the first, location,
where you can provide a longitude and latitude pair of coordinates where
you want the map centered. (We found these for NYC from a quick google
search of ``coordinates nyc''.) The next argument, zoom, takes an
integer between 3 and 21 and controls how far the mapped is zoomed in.
In this exercise, you'll set a third argument, scale, equal to 1. This
controls the resolution of the downloaded maps and you'll set it lower
(the default is 2) to reduce how long it takes for the downloads.

Displaying the map is then as simple as calling ggmap() with your
downloaded map as the only argument: ggmap(nyc\_map)

Your turn! We are going to be looking at house sales in Corvallis, but
you probably have no idea where that is! Let's find out.

Instructions 100 XP We've created for you a pair of coordinates called
corvallis. Get a map centered on Corvallis at the following zoom levels
and use ggmap() to plot each. Don't forget to set scale = 1 to reduce
download times.

zoom = 5 (Corvallis is in the State of Oregon on the West Coast of the
USA.) zoom = 13 (The Willamette River runs through town, and Corvallis
is the home of Oregon State University.)

\begin{Shaded}
\begin{Highlighting}[]
\NormalTok{corvallis <-}\StringTok{ }\KeywordTok{c}\NormalTok{(}\DataTypeTok{lon =} \FloatTok{-123.2620}\NormalTok{, }\DataTypeTok{lat =} \FloatTok{44.5646}\NormalTok{)}

\CommentTok{# Get map at zoom level 5: map_5}
\NormalTok{map_}\DecValTok{5}\NormalTok{ <-}\StringTok{ }\KeywordTok{get_map}\NormalTok{(corvallis, }\DataTypeTok{zoom =} \DecValTok{5}\NormalTok{, }\DataTypeTok{scale =} \DecValTok{1}\NormalTok{)}

\CommentTok{# Plot map at zoom level 5}
\KeywordTok{ggmap}\NormalTok{(map_}\DecValTok{5}\NormalTok{)}
\end{Highlighting}
\end{Shaded}

\includegraphics{Geospacial-Data_files/figure-latex/gbm-1.pdf}

\begin{Shaded}
\begin{Highlighting}[]
\CommentTok{# Get map at zoom level 13: corvallis_map}
\NormalTok{corvallis_map <-}\StringTok{ }\KeywordTok{get_map}\NormalTok{(corvallis, }\DataTypeTok{zoom =} \DecValTok{13}\NormalTok{, }\DataTypeTok{scale =} \DecValTok{1}\NormalTok{)}

\CommentTok{# Plot map at zoom level 13}
\KeywordTok{ggmap}\NormalTok{(corvallis_map)}
\end{Highlighting}
\end{Shaded}

\includegraphics{Geospacial-Data_files/figure-latex/gbm-2.pdf}

Putting it all together You now have a nice map of Corvallis, but how do
you put the locations of the house sales on top?

Similar to ggplot(), you can add layers of data to a ggmap() call
(e.g.~+ geom\_point()). It's important to note, however, that ggmap()
sets the map as the default dataset and also sets the default aesthetic
mappings.

This means that if you want to add a layer from something other than the
map (e.g.~sales), you need to explicitly specify both the mapping and
data arguments to the geom.

What does this look like? You've seen how you might make a basic plot of
the sales:

ggplot(sales, aes(lon, lat)) + geom\_point() An equivalent way to
specify the same plot is:

ggplot() + geom\_point(aes(lon, lat), data = sales) Here, we've
specified the data and mapping in the call to geom\_point() rather than
ggplot(). The benefit of specifying the plot this way is you can swap
out ggplot() for a call to ggmap() and get a map in the background of
the plot.

Instructions 100 XP The ggmap package has been loaded for you and
corvallis\_map from the previous exercise is available in your
workspace.

First, take a look at the head() of the sales data. Can you see the
columns specifying the location of the house? Swap out the call to
ggplot() with a call to ggmap() with corvallis\_map.

\begin{Shaded}
\begin{Highlighting}[]
\NormalTok{sales <-}\StringTok{ }\KeywordTok{read.csv}\NormalTok{(}\StringTok{"sales.csv"}\NormalTok{) }\OperatorTok\StringTok{ }\KeywordTok{as_tibble}\NormalTok{()}

\CommentTok{# Look at head() of sales}
\KeywordTok{head}\NormalTok{(sales)}
\end{Highlighting}
\end{Shaded}

\begin{verbatim}
## # A tibble: 6 x 20
##     lon   lat  price finished_square~ year_built date  address city  state
##   <dbl> <dbl>  <dbl>            <int>      <int> <fct> <fct>   <fct> <fct>
## 1 -123.  44.6 267500             1520       1967 2015~ 1112 N~ CORV~ OR   
## 2 -123.  44.6 255000             1665       1990 2015~ 1221 N~ CORV~ OR   
## 3 -123.  44.6 295000             1440       1948 2015~ 440 NW~ CORV~ OR   
## 4 -123.  44.6   5000              784       1978 2015~ 2655 N~ CORV~ OR   
## 5 -123.  44.5  13950             1344       1979 2015~ 300 SE~ CORV~ OR   
## 6 -123.  44.6 233000             1567       2002 2015~ 3006 N~ CORV~ OR   
## # ... with 11 more variables: zip <fct>, acres <dbl>, num_dwellings <int>,
## #   class <fct>, condition <fct>, total_squarefeet <int>, bedrooms <int>,
## #   full_baths <int>, half_baths <int>, month <int>, address_city <fct>
\end{verbatim}

\begin{Shaded}
\begin{Highlighting}[]
\CommentTok{# Swap out call to ggplot() with call to ggmap()}
\KeywordTok{ggmap}\NormalTok{(corvallis_map) }\OperatorTok{+}
\StringTok{  }\KeywordTok{geom_point}\NormalTok{(}\KeywordTok{aes}\NormalTok{(lon, lat), }\DataTypeTok{data =}\NormalTok{ sales)}
\end{Highlighting}
\end{Shaded}

\includegraphics{Geospacial-Data_files/figure-latex/piat-1.pdf}

Insight through aesthetics Adding a map to your plot of sales explains
some of the structure in the data: there are no house sales East of the
Willamette River or on the Oregon State University campus. This
structure is really just a consequence of where houses are in Corvallis;
you can't have a house sale where there are no houses!

The value of displaying data spatially really comes when you add other
variables to the display through the properties of your geometric
objects, like color or size. You already know how to do this with
ggplot2 plots: add additional mappings to the aesthetics of the geom.

Let's see what else you can learn about these houses in Corvallis.

NOTE: Many exercises in this course will require you to create more than
one plot. You can toggle between plots with the arrows at the bottom of
the `Plots' window and zoom in on a plot by clicking the arrows on the
tab at the top of the `Plots' window.

Instructions 3/3 34 XP Map the color of the points to year\_built. How
has Corvallis developed as a town?

Map the size of the points to bedrooms. Are there areas of houses with
fewer or more bedrooms?

Map the color of the points to price per squarefoot (i.e.~price /
finished\_squarefeet). Are there areas with better ``value'' than
others? What makes this plot unsuccessful?

\begin{Shaded}
\begin{Highlighting}[]
\CommentTok{# Map color to year_built}
\KeywordTok{ggmap}\NormalTok{(corvallis_map) }\OperatorTok{+}
\StringTok{    }\KeywordTok{geom_point}\NormalTok{(}\KeywordTok{aes}\NormalTok{(lon, lat, }\DataTypeTok{color =}\NormalTok{ year_built), }\DataTypeTok{data =}\NormalTok{ sales)}
\end{Highlighting}
\end{Shaded}

\includegraphics{Geospacial-Data_files/figure-latex/itta-1.pdf}

\begin{Shaded}
\begin{Highlighting}[]
\CommentTok{# Map size to bedrooms}
\KeywordTok{ggmap}\NormalTok{(corvallis_map) }\OperatorTok{+}
\StringTok{    }\KeywordTok{geom_point}\NormalTok{(}\KeywordTok{aes}\NormalTok{(lon, lat, }\DataTypeTok{size =}\NormalTok{ bedrooms), }\DataTypeTok{data =}\NormalTok{ sales)}
\end{Highlighting}
\end{Shaded}

\includegraphics{Geospacial-Data_files/figure-latex/itta-2.pdf}

\begin{Shaded}
\begin{Highlighting}[]
\CommentTok{# Map color to price / finished_squarefeet}
\KeywordTok{ggmap}\NormalTok{(corvallis_map) }\OperatorTok{+}
\StringTok{    }\KeywordTok{geom_point}\NormalTok{(}\KeywordTok{aes}\NormalTok{(lon, lat, }\DataTypeTok{color =}\NormalTok{ price }\OperatorTok{/}\StringTok{ }\NormalTok{finished_squarefeet), }\DataTypeTok{data =}\NormalTok{ sales)}
\end{Highlighting}
\end{Shaded}

\includegraphics{Geospacial-Data_files/figure-latex/itta-3.pdf}

Different maps The default Google map downloaded by get\_map() is useful
when you need major roads, basic terrain, and places of interest, but
visually it can be a little busy. You want your map to add to your data,
not distract from it, so it can be useful to have other ``quieter''
options.

Sometimes you aren't really interested in the roads and places, but more
what's on the ground (e.g.~grass, trees, desert, or snow), in which case
switching to a satellite view might be more useful. You can get Google
satellite images by changing the maptype argument to ``satellite''.

You can grab Stamen Maps by using source = ``stamen'' in get\_map(),
along with specifying a maptype argument. You can see all possible
values for the maptype argument by looking at ?get\_map, but they
correspond closely to the ``flavors'' described on the Stamen Maps site.
I like the ``toner'' variations, as they are greyscale and a bit simpler
than the Google map.

Let's try some other maps for your plot of house sales.

Instructions 2/2 50 XP Edit your original call to get\_map() to get a
``satellite'' image from Google by adding a maptype argument. Display a
plot of house sales coloured by year\_built using the satellite map.
Edit your original call to get\_map() to get a toner map from Stamen by
adding a source argument and a maptype argument. Display a plot of house
sales coloured by year\_built using the toner map.

\begin{Shaded}
\begin{Highlighting}[]
\NormalTok{corvallis <-}\StringTok{ }\KeywordTok{c}\NormalTok{(}\DataTypeTok{lon =} \FloatTok{-123.2620}\NormalTok{, }\DataTypeTok{lat =} \FloatTok{44.5646}\NormalTok{)}

\CommentTok{# Add a maptype argument to get a satellite map}
\NormalTok{corvallis_map_sat <-}\StringTok{ }\KeywordTok{get_map}\NormalTok{(corvallis, }\DataTypeTok{zoom =} \DecValTok{13}\NormalTok{, }\DataTypeTok{maptype =} \StringTok{"satellite"}\NormalTok{)}
 
 
\CommentTok{# Edit to display satellite map}
\KeywordTok{ggmap}\NormalTok{(corvallis_map_sat) }\OperatorTok{+}
\StringTok{  }\KeywordTok{geom_point}\NormalTok{(}\KeywordTok{aes}\NormalTok{(lon, lat, }\DataTypeTok{color =}\NormalTok{ year_built), }\DataTypeTok{data =}\NormalTok{ sales)}
\end{Highlighting}
\end{Shaded}

\includegraphics{Geospacial-Data_files/figure-latex/dm-1.pdf}

\begin{Shaded}
\begin{Highlighting}[]
\NormalTok{corvallis <-}\StringTok{ }\KeywordTok{c}\NormalTok{(}\DataTypeTok{lon =} \FloatTok{-123.2620}\NormalTok{, }\DataTypeTok{lat =} \FloatTok{44.5646}\NormalTok{)}
 
\CommentTok{# Add source and maptype to get toner map from Stamen Maps}
\NormalTok{corvallis_map_bw <-}\StringTok{ }\KeywordTok{get_map}\NormalTok{(corvallis, }\DataTypeTok{zoom =} \DecValTok{13}\NormalTok{, }\DataTypeTok{source =} \StringTok{"stamen"}\NormalTok{, }\DataTypeTok{maptype =} \StringTok{"toner"}\NormalTok{)}

\CommentTok{# Edit to display toner map}
\KeywordTok{ggmap}\NormalTok{(corvallis_map_bw) }\OperatorTok{+}
\StringTok{  }\KeywordTok{geom_point}\NormalTok{(}\KeywordTok{aes}\NormalTok{(lon, lat, }\DataTypeTok{color =}\NormalTok{ year_built), }\DataTypeTok{data =}\NormalTok{ sales)}
\end{Highlighting}
\end{Shaded}

\includegraphics{Geospacial-Data_files/figure-latex/dm-2.pdf}

Leveraging ggplot2's strengths You've seen you can add layers to a
ggmap() plot by adding geom\_***() layers and specifying the data and
mapping explicitly, but this approach has two big downsides: further
layers also need to specify the data and mappings, and facetting won't
work at all.

Luckily ggmap() provides a way around these downsides: the base\_layer
argument. You can pass base\_layer a normal ggplot() call that specifies
the default data and mappings for all layers.

For example, the initial plot:

ggmap(corvallis\_map) + geom\_point(data = sales, aes(lon, lat)) could
have instead been:

ggmap(corvallis\_map, base\_layer = ggplot(sales, aes(lon, lat))) +
geom\_point() By moving aes(x, y) and data from the initial
geom\_point() function to the ggplot() call within the ggmap() call, you
can add facets, or extra layers, the usual ggplot2 way.

Let's try it out.

Instructions 1/2 50 XP 1 2 Rewrite the first plot to use the base\_layer
argument of ggmap().

Add a base\_layer argument to the ggmap() call. This should call
ggplot(). Move the data and x and y mappings out of geom\_point(). Leave
the color argument inside the aes() function within your geom\_point()
call.

Rewrite the plot to use the base\_layer argument of ggmap(). Set the
color argument inside the aes() function to class. Add a facet\_wrap()
to facet by class. This function takes a formula.

\begin{Shaded}
\begin{Highlighting}[]
\CommentTok{# Use base_layer argument to ggmap() to specify data and x, y mappings}

  \KeywordTok{ggmap}\NormalTok{(corvallis_map_bw, }
    \DataTypeTok{base_layer =} \KeywordTok{ggplot}\NormalTok{(sales, }\KeywordTok{aes}\NormalTok{(lon, lat))) }\OperatorTok{+}
\StringTok{  }\KeywordTok{geom_point}\NormalTok{(}\KeywordTok{aes}\NormalTok{(}\DataTypeTok{color =}\NormalTok{ year_built))}
\end{Highlighting}
\end{Shaded}

\includegraphics{Geospacial-Data_files/figure-latex/lgs-1.pdf}

\begin{Shaded}
\begin{Highlighting}[]
\CommentTok{# Use base_layer argument to ggmap() and add facet_wrap()}
  \KeywordTok{ggmap}\NormalTok{(corvallis_map_bw, }
    \DataTypeTok{base_layer =} \KeywordTok{ggplot}\NormalTok{(sales, }\KeywordTok{aes}\NormalTok{(lon, lat))) }\OperatorTok{+}
\StringTok{  }\KeywordTok{geom_point}\NormalTok{(}\KeywordTok{aes}\NormalTok{(}\DataTypeTok{color =}\NormalTok{ class)) }\OperatorTok{+}
\StringTok{  }\KeywordTok{facet_wrap}\NormalTok{(}\KeywordTok{vars}\NormalTok{(class))}
\end{Highlighting}
\end{Shaded}

\includegraphics{Geospacial-Data_files/figure-latex/lgs-2.pdf}

A quick alternative ggmap also provides a quick alternative to ggmap().
Like qplot() in ggplot2, qmplot() is less flexible than a full
specification, but often involves significantly less typing. qmplot()
replaces both steps -- downloading the map and displaying the map -- and
its syntax is a blend between qplot(), get\_map(), and ggmap().

Let's take a look at the qmplot() version of the faceted plot from the
previous exercise:

qmplot(lon, lat, data = sales, geom = ``point'', color = class) +
facet\_wrap(\textasciitilde{} class) Notice we didn't specify a map,
since qmplot() will grab one on its own. Otherwise the qmplot() call
looks a lot like the corresponding qplot() call: use points to display
the sales data, mapping lon to the x-axis, lat to the y-axis, and class
to color. qmplot() also sets the default dataset and mapping (without
the need for base\_layer) so you can add facets without any extra work.

Instructions 100 XP Using the example as a guide, use qmplot() to create
a plot of the house sales where color is mapped to bedrooms, faceted by
month.

\begin{Shaded}
\begin{Highlighting}[]
\CommentTok{# Plot house sales using qmplot()}
\KeywordTok{qmplot}\NormalTok{(lon, lat, }\DataTypeTok{data =}\NormalTok{ sales, }
       \DataTypeTok{geom =} \StringTok{"point"}\NormalTok{, }\DataTypeTok{color =}\NormalTok{ bedrooms) }\OperatorTok{+}
\StringTok{  }\KeywordTok{facet_wrap}\NormalTok{(}\OperatorTok{~}\StringTok{ }\NormalTok{month)}
\end{Highlighting}
\end{Shaded}

\includegraphics{Geospacial-Data_files/figure-latex/aqa-1.pdf}

Drawing polygons A choropleth map describes a map where polygons are
colored according to some variable. In the ward\_sales data frame, you
have information on the house sales summarised to the ward level. Your
goal is to create a map where each ward is colored by one of your
summaries: the number of sales or the average sales price.

In the data frame, each row describes one point on the boundary of a
ward. The lon and lat variables describe its location and ward describes
which ward it belongs to, but what are group and order?

Remember the two tricky things about polygons? An area may be described
by more than one polygon and order matters. group is an identifier for a
single polygon, but a ward may be composed of more than one polygon, so
you would see more than one value of group for such a ward. order
describes the order in which the points should be drawn to create the
correct shapes.

In ggplot2, polygons are drawn with geom\_polygon(). Each row of your
data is one point on the boundary and points are joined up in the order
in which they appear in the data frame. You specify which variables
describe position using the x and y aesthetics and which points belong
to a single polygon using the group aesthetic.

This is a little tricky, so before you make your desired plot, let's
explore this a little more.

Instructions 4/4 25 XP The ward\_sales data frame is loaded in your
workspace. You may want to take a look with head(ward\_sales).

Add a geom\_point() layer with the color aesthetic mapped to ward. How
many wards are in Corvallis? Add a geom\_point() layer with the color
aesthetic mapped to group. Can you see some wards that are described by
more than one polygon? Add a geom\_path() layer with the group aesthetic
mapped to group. See how points in the same group are joined. Finally,
add a geom\_polygon() layer with the fill aesthetic mapped to ward and
the group aesthetic mapped to group.

\begin{Shaded}
\begin{Highlighting}[]
\NormalTok{ward_sales <-}\StringTok{ }\KeywordTok{read.csv}\NormalTok{(}\StringTok{"ward_sales.csv"}\NormalTok{)}

\CommentTok{# Add a point layer with color mapped to ward}
\KeywordTok{ggplot}\NormalTok{(ward_sales, }\KeywordTok{aes}\NormalTok{(lon, lat)) }\OperatorTok{+}
\KeywordTok{geom_point}\NormalTok{(}\KeywordTok{aes}\NormalTok{(}\DataTypeTok{color =}\NormalTok{ ward))}
\end{Highlighting}
\end{Shaded}

\includegraphics{Geospacial-Data_files/figure-latex/dp-1.pdf}

\begin{Shaded}
\begin{Highlighting}[]
\CommentTok{# Add a point layer with color mapped to group}

\KeywordTok{ggplot}\NormalTok{(ward_sales, }\KeywordTok{aes}\NormalTok{(lon, lat)) }\OperatorTok{+}
\KeywordTok{geom_point}\NormalTok{(}\KeywordTok{aes}\NormalTok{(}\DataTypeTok{color =}\NormalTok{ group))}
\end{Highlighting}
\end{Shaded}

\includegraphics{Geospacial-Data_files/figure-latex/dp-2.pdf}

\begin{Shaded}
\begin{Highlighting}[]
\CommentTok{# Add a path layer with group mapped to group}
\KeywordTok{ggplot}\NormalTok{(ward_sales, }\KeywordTok{aes}\NormalTok{(lon, lat)) }\OperatorTok{+}
\KeywordTok{geom_path}\NormalTok{(}\KeywordTok{aes}\NormalTok{(}\DataTypeTok{group =}\NormalTok{ group))}
\end{Highlighting}
\end{Shaded}

\includegraphics{Geospacial-Data_files/figure-latex/dp-3.pdf}

\begin{Shaded}
\begin{Highlighting}[]
\CommentTok{# Add a polygon layer with fill mapped to ward, and group to group}
\KeywordTok{ggplot}\NormalTok{(ward_sales, }\KeywordTok{aes}\NormalTok{(lon, lat)) }\OperatorTok{+}
\KeywordTok{geom_polygon}\NormalTok{(}\KeywordTok{aes}\NormalTok{(}\DataTypeTok{fill =} \KeywordTok{as.factor}\NormalTok{(ward), }\DataTypeTok{group =}\NormalTok{ group))}
\end{Highlighting}
\end{Shaded}

\includegraphics{Geospacial-Data_files/figure-latex/dp-4.pdf}

Choropleth map Now that you understand drawing polygons, let's get your
polygons on a map. Remember, you replace your ggplot() call with a
ggmap() call and the original ggplot() call moves to the base\_layer()
argument, then you add your polygon layer as usual:

ggmap(corvallis\_map\_bw, base\_layer = ggplot(ward\_sales, aes(lon,
lat))) + geom\_polygon(aes(group = group, fill = ward)) Try it out in
the console now!

Uh oh, things don't look right. Wards 1, 3 and 8 look jaggardy and
wrong. What's happened? Part of the ward boundaries are beyond the map
boundary. Due to the default settings in ggmap(), any data off the map
is dropped before plotting, so some polygon boundaries are dropped and
when the remaining points are joined up you get the wrong shapes.

Don't worry, there is a solution: ggmap() provides some arguments to
control this behaviour. Arguments extent = ``normal'' along with
maprange = FALSE force the plot to use the data range rather than the
map range to define the plotting boundaries.

Instructions 3/3 30 XP Update the ggmap() call to fix the polygon
cropping. Set extent to ``normal'' and maprange to FALSE. Update the
plot, swapping the polygon fill color from ward to num\_sales. Update
the plot again, mapping fill to avg\_price. Also, set alpha to 0.8 in
your call to geom\_polygon() to allow the map to show through.

\begin{Shaded}
\begin{Highlighting}[]
\CommentTok{# Fix the polygon cropping}
\KeywordTok{ggmap}\NormalTok{(corvallis_map_bw, }
      \DataTypeTok{base_layer =} \KeywordTok{ggplot}\NormalTok{(ward_sales, }\KeywordTok{aes}\NormalTok{(lon, lat)), }\DataTypeTok{extent =} \StringTok{"normal"}\NormalTok{, }\DataTypeTok{maprange =} \OtherTok{FALSE}\NormalTok{) }\OperatorTok{+}
\StringTok{  }\KeywordTok{geom_polygon}\NormalTok{(}\KeywordTok{aes}\NormalTok{(}\DataTypeTok{group =}\NormalTok{ group, }\DataTypeTok{fill =}\NormalTok{ ward))}
\end{Highlighting}
\end{Shaded}

\includegraphics{Geospacial-Data_files/figure-latex/cm-1.pdf}

\begin{Shaded}
\begin{Highlighting}[]
\CommentTok{# Repeat, but map fill to num_sales}
\KeywordTok{ggmap}\NormalTok{(corvallis_map_bw, }
      \DataTypeTok{base_layer =} \KeywordTok{ggplot}\NormalTok{(ward_sales, }\KeywordTok{aes}\NormalTok{(lon, lat)),}
      \DataTypeTok{extent =} \StringTok{"normal"}\NormalTok{, }\DataTypeTok{maprange =} \OtherTok{FALSE}\NormalTok{) }\OperatorTok{+}
\StringTok{  }\KeywordTok{geom_polygon}\NormalTok{(}\KeywordTok{aes}\NormalTok{(}\DataTypeTok{group =}\NormalTok{ group, }\DataTypeTok{fill =}\NormalTok{ num_sales))}
\end{Highlighting}
\end{Shaded}

\includegraphics{Geospacial-Data_files/figure-latex/cm-2.pdf}

\begin{Shaded}
\begin{Highlighting}[]
\CommentTok{# Repeat again, but map fill to avg_price}
\KeywordTok{ggmap}\NormalTok{(corvallis_map_bw, }
      \DataTypeTok{base_layer =} \KeywordTok{ggplot}\NormalTok{(ward_sales, }\KeywordTok{aes}\NormalTok{(lon, lat)),}
      \DataTypeTok{extent =} \StringTok{"normal"}\NormalTok{, }\DataTypeTok{maprange =} \OtherTok{FALSE}\NormalTok{) }\OperatorTok{+}
\StringTok{  }\KeywordTok{geom_polygon}\NormalTok{(}\KeywordTok{aes}\NormalTok{(}\DataTypeTok{group =}\NormalTok{ group, }\DataTypeTok{fill =}\NormalTok{ avg_price), }\DataTypeTok{alpha =} \FloatTok{.8}\NormalTok{)}
\end{Highlighting}
\end{Shaded}

\includegraphics{Geospacial-Data_files/figure-latex/cm-3.pdf}

Raster data as a heatmap The predicted house prices in preds are called
raster data: you have a variable measured (or in this case predicted) at
every location in a regular grid.

Looking at head(preds) in the console, you can see the lat values
stepping up in intervals of about 0.002, as lon is constant. After 40
rows, lon increases by about 0.003, as lat runs through the same values.
For each lat/lon location, you also have a predicted\_price. You'll see
later in Chapter 3, that a more useful way to think about (and store)
this kind of data is in a matrix.

When data forms a regular grid, one approach to displaying it is as a
heatmap. geom\_tile() in ggplot2 draws a rectangle that is centered on
each location that fills the space between it and the next location, in
effect tiling the whole space. By mapping a variable to the fill
aesthetic, you end up with a heatmap.

Instructions 3/3 0 XP Create a simple dot plot of the locations in preds
by adding a geom\_point() layer to the first ggplot() call. Verify that
the locations form a regular grid. To the second ggplot(), swap
geom\_point() for geom\_tile(), where predicted\_price is mapped to
fill. Remember that fill is an argument to aes(), which is the first and
only argument in your call to geom\_tile(). Create a ggmap() using the
corvallis\_map\_bw map. Add a geom\_tile() layer with lon, lat, and
predicted\_price aesthetics from the second plot. Use preds as the
layer's data. Set the layer's alpha transparency to 0.8.

\begin{Shaded}
\begin{Highlighting}[]
\NormalTok{preds <-}\StringTok{ }\KeywordTok{read.csv}\NormalTok{(}\StringTok{"preds.csv"}\NormalTok{)}

\CommentTok{# Add a geom_point() layer}
\KeywordTok{ggplot}\NormalTok{(preds, }\KeywordTok{aes}\NormalTok{(lon, lat)) }\OperatorTok{+}\StringTok{ }\KeywordTok{geom_point}\NormalTok{()}
\end{Highlighting}
\end{Shaded}

\includegraphics{Geospacial-Data_files/figure-latex/rdhm-1.pdf}

\begin{Shaded}
\begin{Highlighting}[]
\CommentTok{# Add a tile layer with fill mapped to predicted_price}
\KeywordTok{ggplot}\NormalTok{(preds, }\KeywordTok{aes}\NormalTok{(lon, lat)) }\OperatorTok{+}\StringTok{ }\KeywordTok{geom_tile}\NormalTok{(}\KeywordTok{aes}\NormalTok{(}\DataTypeTok{fill=}\NormalTok{predicted_price))}
\end{Highlighting}
\end{Shaded}

\includegraphics{Geospacial-Data_files/figure-latex/rdhm-2.pdf}

\begin{Shaded}
\begin{Highlighting}[]
\CommentTok{# Use ggmap() instead of ggplot()}
\KeywordTok{ggmap}\NormalTok{(corvallis_map_bw) }\OperatorTok{+}
\StringTok{  }\KeywordTok{geom_tile}\NormalTok{(}\KeywordTok{aes}\NormalTok{(lon, lat, }\DataTypeTok{fill =}\NormalTok{ predicted_price), }
            \DataTypeTok{data =}\NormalTok{ preds, }\DataTypeTok{alpha =} \FloatTok{0.8}\NormalTok{)}
\end{Highlighting}
\end{Shaded}

\includegraphics{Geospacial-Data_files/figure-latex/rdhm-3.pdf}

\hypertarget{points-and-polygons}{%
\subsection{Points and Polygons}\label{points-and-polygons}}

Let's take a look at a spatial object We've loaded a particular sp
object into your workspace: countries\_sp. There are special print(),
summary() and plot() methods for these objects. What's a method? It's a
special version of a function that gets used based on the type of object
you pass to it. It's common when a package creates new types of objects
for it to contain methods for simple exploration and display.

In practice, this means you can call plot(countries\_sp) and if there is
a method for the class of countries\_sp, it gets called. The print()
method is the one called when you just type an object's name in the
console.

Can you figure out what kind of object this countries\_sp is? Can you
see what coordinate system this spatial data uses? What does the data in
the object describe?

Instructions 100 XP Print countries\_sp. Why isn't this very useful?.
Call summary() on countries\_sp. Call plot() on countries\_sp.

\begin{Shaded}
\begin{Highlighting}[]
\KeywordTok{load}\NormalTok{(}\DataTypeTok{file =} \StringTok{"countries_sp.rda"}\NormalTok{)}

\CommentTok{# Print countries_sp}
\KeywordTok{print}\NormalTok{(countries_sp)}
\end{Highlighting}
\end{Shaded}

\begin{verbatim}
## class       : SpatialPolygons 
## features    : 177 
## extent      : -180, 180, -89.9999, 83.64513  (xmin, xmax, ymin, ymax)
## crs         : +proj=longlat +datum=WGS84 +no_defs +ellps=WGS84 +towgs84=0,0,0
\end{verbatim}

\begin{Shaded}
\begin{Highlighting}[]
\CommentTok{# Call summary() on countries_sp}
\KeywordTok{summary}\NormalTok{(countries_sp)}
\end{Highlighting}
\end{Shaded}

\begin{verbatim}
## Object of class SpatialPolygons
## Coordinates:
##         min       max
## x -180.0000 180.00000
## y  -89.9999  83.64513
## Is projected: FALSE 
## proj4string :
## [+proj=longlat +datum=WGS84 +no_defs +ellps=WGS84 +towgs84=0,0,0]
\end{verbatim}

\begin{Shaded}
\begin{Highlighting}[]
\CommentTok{# Call plot() on countries_sp}
\KeywordTok{plot}\NormalTok{(countries_sp)}
\end{Highlighting}
\end{Shaded}

\includegraphics{Geospacial-Data_files/figure-latex/splook-1.pdf}

What's inside a spatial object? What did you learn about the methods in
the previous exercise? print() gives a printed form of the object, but
it is often too long and not very helpful. summary() provides a much
more concise description of the object, including its class (in this
case SpatialPolygons), the extent of the spatial data, and the
coordinate reference system information (you'll learn more about this in
Chapter 4). plot() displays the contents, in this case drawing a map of
the world.

But, how is that information stored in the SpatialPolygons object? In
this exercise you'll explore the structure of this object. You already
know about using str() to look at R objects, but what you might not know
is that it takes an optional argument max.level that restricts how far
down the hierarchy of the object str() prints. This can be useful to
limit how much information you have to handle.

Let's see if you can get a handle on how this object is structured.

Instructions 100 XP Call str() on countries\_sp. This won't be very
helpful, except to convince you this is a complicated stucture! Call
str() on countries\_sp, setting max.level to 2. What is at the highest
level of this object? Can you see where things might be stored?

\begin{Shaded}
\begin{Highlighting}[]
\CommentTok{# Call str() on countries_sp}
\KeywordTok{str}\NormalTok{(countries_sp)}
\end{Highlighting}
\end{Shaded}

\begin{verbatim}
## Formal class 'SpatialPolygons' [package "sp"] with 4 slots
##   ..@ polygons   :List of 177
##   .. ..$ :Formal class 'Polygons' [package "sp"] with 5 slots
##   .. .. .. ..@ Polygons :List of 1
##   .. .. .. .. ..$ :Formal class 'Polygon' [package "sp"] with 5 slots
##   .. .. .. .. .. .. ..@ labpt  : num [1:2] 66.1 33.9
##   .. .. .. .. .. .. ..@ area   : num 63.6
##   .. .. .. .. .. .. ..@ hole   : logi FALSE
##   .. .. .. .. .. .. ..@ ringDir: int 1
##   .. .. .. .. .. .. ..@ coords : num [1:69, 1:2] 61.2 62.2 63 63.2 64 ...
##   .. .. .. .. .. .. .. ..- attr(*, "dimnames")=List of 2
##   .. .. .. .. .. .. .. .. ..$ : NULL
##   .. .. .. .. .. .. .. .. ..$ : chr [1:2] "x" "y"
##   .. .. .. ..@ plotOrder: int 1
##   .. .. .. ..@ labpt    : num [1:2] 66.1 33.9
##   .. .. .. ..@ ID       : chr "0"
##   .. .. .. ..@ area     : num 63.6
##   .. ..$ :Formal class 'Polygons' [package "sp"] with 5 slots
##   .. .. .. ..@ Polygons :List of 2
##   .. .. .. .. ..$ :Formal class 'Polygon' [package "sp"] with 5 slots
##   .. .. .. .. .. .. ..@ labpt  : num [1:2] 17.5 -12.3
##   .. .. .. .. .. .. ..@ area   : num 103
##   .. .. .. .. .. .. ..@ hole   : logi FALSE
##   .. .. .. .. .. .. ..@ ringDir: int 1
##   .. .. .. .. .. .. ..@ coords : num [1:66, 1:2] 16.3 16.6 16.9 17.1 17.5 ...
##   .. .. .. .. .. .. .. ..- attr(*, "dimnames")=List of 2
##   .. .. .. .. .. .. .. .. ..$ : NULL
##   .. .. .. .. .. .. .. .. ..$ : chr [1:2] "x" "y"
##   .. .. .. .. ..$ :Formal class 'Polygon' [package "sp"] with 5 slots
##   .. .. .. .. .. .. ..@ labpt  : num [1:2] 12.38 -5.05
##   .. .. .. .. .. .. ..@ area   : num 0.653
##   .. .. .. .. .. .. ..@ hole   : logi FALSE
##   .. .. .. .. .. .. ..@ ringDir: int 1
##   .. .. .. .. .. .. ..@ coords : num [1:9, 1:2] 12.4 12.2 11.9 12.3 12.6 ...
##   .. .. .. .. .. .. .. ..- attr(*, "dimnames")=List of 2
##   .. .. .. .. .. .. .. .. ..$ : NULL
##   .. .. .. .. .. .. .. .. ..$ : chr [1:2] "x" "y"
##   .. .. .. ..@ plotOrder: int [1:2] 1 2
##   .. .. .. ..@ labpt    : num [1:2] 17.5 -12.3
##   .. .. .. ..@ ID       : chr "1"
##   .. .. .. ..@ area     : num 104
##   .. ..$ :Formal class 'Polygons' [package "sp"] with 5 slots
##   .. .. .. ..@ Polygons :List of 1
##   .. .. .. .. ..$ :Formal class 'Polygon' [package "sp"] with 5 slots
##   .. .. .. .. .. .. ..@ labpt  : num [1:2] 20 41.1
##   .. .. .. .. .. .. ..@ area   : num 3.19
##   .. .. .. .. .. .. ..@ hole   : logi FALSE
##   .. .. .. .. .. .. ..@ ringDir: int 1
##   .. .. .. .. .. .. ..@ coords : num [1:22, 1:2] 20.6 20.5 20.6 21 21 ...
##   .. .. .. .. .. .. .. ..- attr(*, "dimnames")=List of 2
##   .. .. .. .. .. .. .. .. ..$ : NULL
##   .. .. .. .. .. .. .. .. ..$ : chr [1:2] "x" "y"
##   .. .. .. ..@ plotOrder: int 1
##   .. .. .. ..@ labpt    : num [1:2] 20 41.1
##   .. .. .. ..@ ID       : chr "2"
##   .. .. .. ..@ area     : num 3.19
##   .. ..$ :Formal class 'Polygons' [package "sp"] with 5 slots
##   .. .. .. ..@ Polygons :List of 1
##   .. .. .. .. ..$ :Formal class 'Polygon' [package "sp"] with 5 slots
##   .. .. .. .. .. .. ..@ labpt  : num [1:2] 54.2 23.9
##   .. .. .. .. .. .. ..@ area   : num 7.1
##   .. .. .. .. .. .. ..@ hole   : logi FALSE
##   .. .. .. .. .. .. ..@ ringDir: int 1
##   .. .. .. .. .. .. ..@ coords : num [1:22, 1:2] 51.6 51.8 51.8 52.6 53.4 ...
##   .. .. .. .. .. .. .. ..- attr(*, "dimnames")=List of 2
##   .. .. .. .. .. .. .. .. ..$ : NULL
##   .. .. .. .. .. .. .. .. ..$ : chr [1:2] "x" "y"
##   .. .. .. ..@ plotOrder: int 1
##   .. .. .. ..@ labpt    : num [1:2] 54.2 23.9
##   .. .. .. ..@ ID       : chr "3"
##   .. .. .. ..@ area     : num 7.1
##   .. ..$ :Formal class 'Polygons' [package "sp"] with 5 slots
##   .. .. .. ..@ Polygons :List of 2
##   .. .. .. .. ..$ :Formal class 'Polygon' [package "sp"] with 5 slots
##   .. .. .. .. .. .. ..@ labpt  : num [1:2] -67.3 -54.4
##   .. .. .. .. .. .. ..@ area   : num 3.3
##   .. .. .. .. .. .. ..@ hole   : logi FALSE
##   .. .. .. .. .. .. ..@ ringDir: int 1
##   .. .. .. .. .. .. ..@ coords : num [1:11, 1:2] -65.5 -66.5 -67 -67.6 -68.6 ...
##   .. .. .. .. .. .. .. ..- attr(*, "dimnames")=List of 2
##   .. .. .. .. .. .. .. .. ..$ : NULL
##   .. .. .. .. .. .. .. .. ..$ : chr [1:2] "x" "y"
##   .. .. .. .. ..$ :Formal class 'Polygon' [package "sp"] with 5 slots
##   .. .. .. .. .. .. ..@ labpt  : num [1:2] -65.1 -35.2
##   .. .. .. .. .. .. ..@ area   : num 276
##   .. .. .. .. .. .. ..@ hole   : logi FALSE
##   .. .. .. .. .. .. ..@ ringDir: int 1
##   .. .. .. .. .. .. ..@ coords : num [1:110, 1:2] -65 -64.4 -64 -62.8 -62.7 ...
##   .. .. .. .. .. .. .. ..- attr(*, "dimnames")=List of 2
##   .. .. .. .. .. .. .. .. ..$ : NULL
##   .. .. .. .. .. .. .. .. ..$ : chr [1:2] "x" "y"
##   .. .. .. ..@ plotOrder: int [1:2] 2 1
##   .. .. .. ..@ labpt    : num [1:2] -65.1 -35.2
##   .. .. .. ..@ ID       : chr "4"
##   .. .. .. ..@ area     : num 279
##   .. ..$ :Formal class 'Polygons' [package "sp"] with 5 slots
##   .. .. .. ..@ Polygons :List of 1
##   .. .. .. .. ..$ :Formal class 'Polygon' [package "sp"] with 5 slots
##   .. .. .. .. .. .. ..@ labpt  : num [1:2] 45 40.2
##   .. .. .. .. .. .. ..@ area   : num 3.03
##   .. .. .. .. .. .. ..@ hole   : logi FALSE
##   .. .. .. .. .. .. ..@ ringDir: int 1
##   .. .. .. .. .. .. ..@ coords : num [1:20, 1:2] 43.6 45 45.2 45.6 45.4 ...
##   .. .. .. .. .. .. .. ..- attr(*, "dimnames")=List of 2
##   .. .. .. .. .. .. .. .. ..$ : NULL
##   .. .. .. .. .. .. .. .. ..$ : chr [1:2] "x" "y"
##   .. .. .. ..@ plotOrder: int 1
##   .. .. .. ..@ labpt    : num [1:2] 45 40.2
##   .. .. .. ..@ ID       : chr "5"
##   .. .. .. ..@ area     : num 3.03
##   .. ..$ :Formal class 'Polygons' [package "sp"] with 5 slots
##   .. .. .. ..@ Polygons :List of 8
##   .. .. .. .. ..$ :Formal class 'Polygon' [package "sp"] with 5 slots
##   .. .. .. .. .. .. ..@ labpt  : num [1:2] -62.3 -80.5
##   .. .. .. .. .. .. ..@ area   : num 4.2
##   .. .. .. .. .. .. ..@ hole   : logi FALSE
##   .. .. .. .. .. .. ..@ ringDir: int 1
##   .. .. .. .. .. .. ..@ coords : num [1:13, 1:2] -59.6 -59.9 -60.2 -62.3 -64.5 ...
##   .. .. .. .. .. .. .. ..- attr(*, "dimnames")=List of 2
##   .. .. .. .. .. .. .. .. ..$ : NULL
##   .. .. .. .. .. .. .. .. ..$ : chr [1:2] "x" "y"
##   .. .. .. .. ..$ :Formal class 'Polygon' [package "sp"] with 5 slots
##   .. .. .. .. .. .. ..@ labpt  : num [1:2] -161.4 -78.9
##   .. .. .. .. .. .. ..@ area   : num 3.72
##   .. .. .. .. .. .. ..@ hole   : logi FALSE
##   .. .. .. .. .. .. ..@ ringDir: int 1
##   .. .. .. .. .. .. ..@ coords : num [1:12, 1:2] -159 -161 -162 -163 -163 ...
##   .. .. .. .. .. .. .. ..- attr(*, "dimnames")=List of 2
##   .. .. .. .. .. .. .. .. ..$ : NULL
##   .. .. .. .. .. .. .. .. ..$ : chr [1:2] "x" "y"
##   .. .. .. .. ..$ :Formal class 'Polygon' [package "sp"] with 5 slots
##   .. .. .. .. .. .. ..@ labpt  : num [1:2] -48 -79.6
##   .. .. .. .. .. .. ..@ area   : num 20.4
##   .. .. .. .. .. .. ..@ hole   : logi FALSE
##   .. .. .. .. .. .. ..@ ringDir: int 1
##   .. .. .. .. .. .. ..@ coords : num [1:22, 1:2] -45.2 -43.9 -43.5 -43.4 -43.3 ...
##   .. .. .. .. .. .. .. ..- attr(*, "dimnames")=List of 2
##   .. .. .. .. .. .. .. .. ..$ : NULL
##   .. .. .. .. .. .. .. .. ..$ : chr [1:2] "x" "y"
##   .. .. .. .. ..$ :Formal class 'Polygon' [package "sp"] with 5 slots
##   .. .. .. .. .. .. ..@ labpt  : num [1:2] -120.9 -73.7
##   .. .. .. .. .. .. ..@ area   : num 1.54
##   .. .. .. .. .. .. ..@ hole   : logi FALSE
##   .. .. .. .. .. .. ..@ ringDir: int 1
##   .. .. .. .. .. .. ..@ coords : num [1:10, 1:2] -121 -120 -119 -119 -120 ...
##   .. .. .. .. .. .. .. ..- attr(*, "dimnames")=List of 2
##   .. .. .. .. .. .. .. .. ..$ : NULL
##   .. .. .. .. .. .. .. .. ..$ : chr [1:2] "x" "y"
##   .. .. .. .. ..$ :Formal class 'Polygon' [package "sp"] with 5 slots
##   .. .. .. .. .. .. ..@ labpt  : num [1:2] -125.9 -73.6
##   .. .. .. .. .. .. ..@ area   : num 0.683
##   .. .. .. .. .. .. ..@ hole   : logi FALSE
##   .. .. .. .. .. .. ..@ ringDir: int 1
##   .. .. .. .. .. .. ..@ coords : num [1:8, 1:2] -126 -124 -125 -126 -127 ...
##   .. .. .. .. .. .. .. ..- attr(*, "dimnames")=List of 2
##   .. .. .. .. .. .. .. .. ..$ : NULL
##   .. .. .. .. .. .. .. .. ..$ : chr [1:2] "x" "y"
##   .. .. .. .. ..$ :Formal class 'Polygon' [package "sp"] with 5 slots
##   .. .. .. .. .. .. ..@ labpt  : num [1:2] -99.5 -72.2
##   .. .. .. .. .. .. ..@ area   : num 3
##   .. .. .. .. .. .. ..@ hole   : logi FALSE
##   .. .. .. .. .. .. ..@ ringDir: int 1
##   .. .. .. .. .. .. ..@ coords : num [1:14, 1:2] -99 -97.9 -96.8 -96.2 -97 ...
##   .. .. .. .. .. .. .. ..- attr(*, "dimnames")=List of 2
##   .. .. .. .. .. .. .. .. ..$ : NULL
##   .. .. .. .. .. .. .. .. ..$ : chr [1:2] "x" "y"
##   .. .. .. .. ..$ :Formal class 'Polygon' [package "sp"] with 5 slots
##   .. .. .. .. .. .. ..@ labpt  : num [1:2] -71 -71.1
##   .. .. .. .. .. .. ..@ area   : num 12.7
##   .. .. .. .. .. .. ..@ hole   : logi FALSE
##   .. .. .. .. .. .. ..@ ringDir: int 1
##   .. .. .. .. .. .. ..@ coords : num [1:26, 1:2] -68.5 -68.3 -68.5 -68.8 -70 ...
##   .. .. .. .. .. .. .. ..- attr(*, "dimnames")=List of 2
##   .. .. .. .. .. .. .. .. ..$ : NULL
##   .. .. .. .. .. .. .. .. ..$ : chr [1:2] "x" "y"
##   .. .. .. .. ..$ :Formal class 'Polygon' [package "sp"] with 5 slots
##   .. .. .. .. .. .. ..@ labpt  : num [1:2] 21.3 -80.5
##   .. .. .. .. .. .. ..@ area   : num 5983
##   .. .. .. .. .. .. ..@ hole   : logi FALSE
##   .. .. .. .. .. .. ..@ ringDir: int 1
##   .. .. .. .. .. .. ..@ coords : num [1:556, 1:2] -58.6 -59 -59.8 -60.6 -61.3 ...
##   .. .. .. .. .. .. .. ..- attr(*, "dimnames")=List of 2
##   .. .. .. .. .. .. .. .. ..$ : NULL
##   .. .. .. .. .. .. .. .. ..$ : chr [1:2] "x" "y"
##   .. .. .. ..@ plotOrder: int [1:8] 8 3 7 1 2 6 4 5
##   .. .. .. ..@ labpt    : num [1:2] 21.3 -80.5
##   .. .. .. ..@ ID       : chr "6"
##   .. .. .. ..@ area     : num 6029
##   .. ..$ :Formal class 'Polygons' [package "sp"] with 5 slots
##   .. .. .. ..@ Polygons :List of 1
##   .. .. .. .. ..$ :Formal class 'Polygon' [package "sp"] with 5 slots
##   .. .. .. .. .. .. ..@ labpt  : num [1:2] 69.5 -49.3
##   .. .. .. .. .. .. ..@ area   : num 1.43
##   .. .. .. .. .. .. ..@ hole   : logi FALSE
##   .. .. .. .. .. .. ..@ ringDir: int 1
##   .. .. .. .. .. .. ..@ coords : num [1:9, 1:2] 68.9 69.6 70.5 70.6 70.3 ...
##   .. .. .. .. .. .. .. ..- attr(*, "dimnames")=List of 2
##   .. .. .. .. .. .. .. .. ..$ : NULL
##   .. .. .. .. .. .. .. .. ..$ : chr [1:2] "x" "y"
##   .. .. .. ..@ plotOrder: int 1
##   .. .. .. ..@ labpt    : num [1:2] 69.5 -49.3
##   .. .. .. ..@ ID       : chr "7"
##   .. .. .. ..@ area     : num 1.43
##   .. ..$ :Formal class 'Polygons' [package "sp"] with 5 slots
##   .. .. .. ..@ Polygons :List of 2
##   .. .. .. .. ..$ :Formal class 'Polygon' [package "sp"] with 5 slots
##   .. .. .. .. .. .. ..@ labpt  : num [1:2] 147 -42
##   .. .. .. .. .. .. ..@ area   : num 7.18
##   .. .. .. .. .. .. ..@ hole   : logi FALSE
##   .. .. .. .. .. .. ..@ ringDir: int 1
##   .. .. .. .. .. .. ..@ coords : num [1:17, 1:2] 145 146 147 148 148 ...
##   .. .. .. .. .. .. .. ..- attr(*, "dimnames")=List of 2
##   .. .. .. .. .. .. .. .. ..$ : NULL
##   .. .. .. .. .. .. .. .. ..$ : chr [1:2] "x" "y"
##   .. .. .. .. ..$ :Formal class 'Polygon' [package "sp"] with 5 slots
##   .. .. .. .. .. .. ..@ labpt  : num [1:2] 134.4 -25.6
##   .. .. .. .. .. .. ..@ area   : num 688
##   .. .. .. .. .. .. ..@ hole   : logi FALSE
##   .. .. .. .. .. .. ..@ ringDir: int 1
##   .. .. .. .. .. .. ..@ coords : num [1:224, 1:2] 144 144 145 145 145 ...
##   .. .. .. .. .. .. .. ..- attr(*, "dimnames")=List of 2
##   .. .. .. .. .. .. .. .. ..$ : NULL
##   .. .. .. .. .. .. .. .. ..$ : chr [1:2] "x" "y"
##   .. .. .. ..@ plotOrder: int [1:2] 2 1
##   .. .. .. ..@ labpt    : num [1:2] 134.4 -25.6
##   .. .. .. ..@ ID       : chr "8"
##   .. .. .. ..@ area     : num 696
##   .. ..$ :Formal class 'Polygons' [package "sp"] with 5 slots
##   .. .. .. ..@ Polygons :List of 1
##   .. .. .. .. ..$ :Formal class 'Polygon' [package "sp"] with 5 slots
##   .. .. .. .. .. .. ..@ labpt  : num [1:2] 14.1 47.6
##   .. .. .. .. .. .. ..@ area   : num 10.2
##   .. .. .. .. .. .. ..@ hole   : logi FALSE
##   .. .. .. .. .. .. ..@ ringDir: int 1
##   .. .. .. .. .. .. ..@ coords : num [1:37, 1:2] 17 16.9 16.3 16.5 16.2 ...
##   .. .. .. .. .. .. .. ..- attr(*, "dimnames")=List of 2
##   .. .. .. .. .. .. .. .. ..$ : NULL
##   .. .. .. .. .. .. .. .. ..$ : chr [1:2] "x" "y"
##   .. .. .. ..@ plotOrder: int 1
##   .. .. .. ..@ labpt    : num [1:2] 14.1 47.6
##   .. .. .. ..@ ID       : chr "9"
##   .. .. .. ..@ area     : num 10.2
##   .. ..$ :Formal class 'Polygons' [package "sp"] with 5 slots
##   .. .. .. ..@ Polygons :List of 2
##   .. .. .. .. ..$ :Formal class 'Polygon' [package "sp"] with 5 slots
##   .. .. .. .. .. .. ..@ labpt  : num [1:2] 45.4 39.2
##   .. .. .. .. .. .. ..@ area   : num 0.546
##   .. .. .. .. .. .. ..@ hole   : logi FALSE
##   .. .. .. .. .. .. ..@ ringDir: int 1
##   .. .. .. .. .. .. ..@ coords : num [1:9, 1:2] 45 45.3 45.7 45.7 46.1 ...
##   .. .. .. .. .. .. .. ..- attr(*, "dimnames")=List of 2
##   .. .. .. .. .. .. .. .. ..$ : NULL
##   .. .. .. .. .. .. .. .. ..$ : chr [1:2] "x" "y"
##   .. .. .. .. ..$ :Formal class 'Polygon' [package "sp"] with 5 slots
##   .. .. .. .. .. .. ..@ labpt  : num [1:2] 47.7 40.3
##   .. .. .. .. .. .. ..@ area   : num 9.1
##   .. .. .. .. .. .. ..@ hole   : logi FALSE
##   .. .. .. .. .. .. ..@ ringDir: int 1
##   .. .. .. .. .. .. ..@ coords : num [1:35, 1:2] 47.4 47.8 48 48.6 49.1 ...
##   .. .. .. .. .. .. .. ..- attr(*, "dimnames")=List of 2
##   .. .. .. .. .. .. .. .. ..$ : NULL
##   .. .. .. .. .. .. .. .. ..$ : chr [1:2] "x" "y"
##   .. .. .. ..@ plotOrder: int [1:2] 2 1
##   .. .. .. ..@ labpt    : num [1:2] 47.7 40.3
##   .. .. .. ..@ ID       : chr "10"
##   .. .. .. ..@ area     : num 9.64
##   .. ..$ :Formal class 'Polygons' [package "sp"] with 5 slots
##   .. .. .. ..@ Polygons :List of 1
##   .. .. .. .. ..$ :Formal class 'Polygon' [package "sp"] with 5 slots
##   .. .. .. .. .. .. ..@ labpt  : num [1:2] 29.91 -3.38
##   .. .. .. .. .. .. ..@ area   : num 2.14
##   .. .. .. .. .. .. ..@ hole   : logi FALSE
##   .. .. .. .. .. .. ..@ ringDir: int 1
##   .. .. .. .. .. .. ..@ coords : num [1:13, 1:2] 29.3 29.3 29 29.6 29.9 ...
##   .. .. .. .. .. .. .. ..- attr(*, "dimnames")=List of 2
##   .. .. .. .. .. .. .. .. ..$ : NULL
##   .. .. .. .. .. .. .. .. ..$ : chr [1:2] "x" "y"
##   .. .. .. ..@ plotOrder: int 1
##   .. .. .. ..@ labpt    : num [1:2] 29.91 -3.38
##   .. .. .. ..@ ID       : chr "11"
##   .. .. .. ..@ area     : num 2.14
##   .. ..$ :Formal class 'Polygons' [package "sp"] with 5 slots
##   .. .. .. ..@ Polygons :List of 1
##   .. .. .. .. ..$ :Formal class 'Polygon' [package "sp"] with 5 slots
##   .. .. .. .. .. .. ..@ labpt  : num [1:2] 4.58 50.65
##   .. .. .. .. .. .. ..@ area   : num 3.83
##   .. .. .. .. .. .. ..@ hole   : logi FALSE
##   .. .. .. .. .. .. ..@ ringDir: int 1
##   .. .. .. .. .. .. ..@ coords : num [1:15, 1:2] 3.31 4.05 4.97 5.61 6.16 ...
##   .. .. .. .. .. .. .. ..- attr(*, "dimnames")=List of 2
##   .. .. .. .. .. .. .. .. ..$ : NULL
##   .. .. .. .. .. .. .. .. ..$ : chr [1:2] "x" "y"
##   .. .. .. ..@ plotOrder: int 1
##   .. .. .. ..@ labpt    : num [1:2] 4.58 50.65
##   .. .. .. ..@ ID       : chr "12"
##   .. .. .. ..@ area     : num 3.83
##   .. ..$ :Formal class 'Polygons' [package "sp"] with 5 slots
##   .. .. .. ..@ Polygons :List of 1
##   .. .. .. .. ..$ :Formal class 'Polygon' [package "sp"] with 5 slots
##   .. .. .. .. .. .. ..@ labpt  : num [1:2] 2.34 9.65
##   .. .. .. .. .. .. ..@ area   : num 9.64
##   .. .. .. .. .. .. ..@ hole   : logi FALSE
##   .. .. .. .. .. .. ..@ ringDir: int 1
##   .. .. .. .. .. .. ..@ coords : num [1:25, 1:2] 2.69 1.87 1.62 1.66 1.46 ...
##   .. .. .. .. .. .. .. ..- attr(*, "dimnames")=List of 2
##   .. .. .. .. .. .. .. .. ..$ : NULL
##   .. .. .. .. .. .. .. .. ..$ : chr [1:2] "x" "y"
##   .. .. .. ..@ plotOrder: int 1
##   .. .. .. ..@ labpt    : num [1:2] 2.34 9.65
##   .. .. .. ..@ ID       : chr "13"
##   .. .. .. ..@ area     : num 9.64
##   .. ..$ :Formal class 'Polygons' [package "sp"] with 5 slots
##   .. .. .. ..@ Polygons :List of 1
##   .. .. .. .. ..$ :Formal class 'Polygon' [package "sp"] with 5 slots
##   .. .. .. .. .. .. ..@ labpt  : num [1:2] -1.78 12.31
##   .. .. .. .. .. .. ..@ area   : num 22.6
##   .. .. .. .. .. .. ..@ hole   : logi FALSE
##   .. .. .. .. .. .. ..@ ringDir: int 1
##   .. .. .. .. .. .. ..@ coords : num [1:39, 1:2] -2.83 -3.51 -3.98 -4.33 -4.78 ...
##   .. .. .. .. .. .. .. ..- attr(*, "dimnames")=List of 2
##   .. .. .. .. .. .. .. .. ..$ : NULL
##   .. .. .. .. .. .. .. .. ..$ : chr [1:2] "x" "y"
##   .. .. .. ..@ plotOrder: int 1
##   .. .. .. ..@ labpt    : num [1:2] -1.78 12.31
##   .. .. .. ..@ ID       : chr "14"
##   .. .. .. ..@ area     : num 22.6
##   .. ..$ :Formal class 'Polygons' [package "sp"] with 5 slots
##   .. .. .. ..@ Polygons :List of 1
##   .. .. .. .. ..$ :Formal class 'Polygon' [package "sp"] with 5 slots
##   .. .. .. .. .. .. ..@ labpt  : num [1:2] 90.3 23.8
##   .. .. .. .. .. .. ..@ area   : num 11.9
##   .. .. .. .. .. .. ..@ hole   : logi FALSE
##   .. .. .. .. .. .. ..@ ringDir: int 1
##   .. .. .. .. .. .. ..@ coords : num [1:36, 1:2] 92.7 92.7 92.3 92.4 92.1 ...
##   .. .. .. .. .. .. .. ..- attr(*, "dimnames")=List of 2
##   .. .. .. .. .. .. .. .. ..$ : NULL
##   .. .. .. .. .. .. .. .. ..$ : chr [1:2] "x" "y"
##   .. .. .. ..@ plotOrder: int 1
##   .. .. .. ..@ labpt    : num [1:2] 90.3 23.8
##   .. .. .. ..@ ID       : chr "15"
##   .. .. .. ..@ area     : num 11.9
##   .. ..$ :Formal class 'Polygons' [package "sp"] with 5 slots
##   .. .. .. ..@ Polygons :List of 1
##   .. .. .. .. ..$ :Formal class 'Polygon' [package "sp"] with 5 slots
##   .. .. .. .. .. .. ..@ labpt  : num [1:2] 25.2 42.8
##   .. .. .. .. .. .. ..@ area   : num 12.1
##   .. .. .. .. .. .. ..@ hole   : logi FALSE
##   .. .. .. .. .. .. ..@ ringDir: int 1
##   .. .. .. .. .. .. ..@ coords : num [1:28, 1:2] 22.7 22.9 23.3 24.1 25.6 ...
##   .. .. .. .. .. .. .. ..- attr(*, "dimnames")=List of 2
##   .. .. .. .. .. .. .. .. ..$ : NULL
##   .. .. .. .. .. .. .. .. ..$ : chr [1:2] "x" "y"
##   .. .. .. ..@ plotOrder: int 1
##   .. .. .. ..@ labpt    : num [1:2] 25.2 42.8
##   .. .. .. ..@ ID       : chr "16"
##   .. .. .. ..@ area     : num 12.1
##   .. ..$ :Formal class 'Polygons' [package "sp"] with 5 slots
##   .. .. .. ..@ Polygons :List of 3
##   .. .. .. .. ..$ :Formal class 'Polygon' [package "sp"] with 5 slots
##   .. .. .. .. .. .. ..@ labpt  : num [1:2] -77.9 24.5
##   .. .. .. .. .. .. ..@ area   : num 0.721
##   .. .. .. .. .. .. ..@ hole   : logi FALSE
##   .. .. .. .. .. .. ..@ ringDir: int 1
##   .. .. .. .. .. .. ..@ coords : num [1:8, 1:2] -77.5 -77.8 -78 -78.4 -78.2 ...
##   .. .. .. .. .. .. .. ..- attr(*, "dimnames")=List of 2
##   .. .. .. .. .. .. .. .. ..$ : NULL
##   .. .. .. .. .. .. .. .. ..$ : chr [1:2] "x" "y"
##   .. .. .. .. ..$ :Formal class 'Polygon' [package "sp"] with 5 slots
##   .. .. .. .. .. .. ..@ labpt  : num [1:2] -78.4 26.7
##   .. .. .. .. .. .. ..@ area   : num 0.388
##   .. .. .. .. .. .. ..@ hole   : logi FALSE
##   .. .. .. .. .. .. ..@ ringDir: int 1
##   .. .. .. .. .. .. ..@ coords : num [1:6, 1:2] -77.8 -78.9 -79 -78.5 -77.8 ...
##   .. .. .. .. .. .. .. ..- attr(*, "dimnames")=List of 2
##   .. .. .. .. .. .. .. .. ..$ : NULL
##   .. .. .. .. .. .. .. .. ..$ : chr [1:2] "x" "y"
##   .. .. .. .. ..$ :Formal class 'Polygon' [package "sp"] with 5 slots
##   .. .. .. .. .. .. ..@ labpt  : num [1:2] -77.3 26.5
##   .. .. .. .. .. .. ..@ area   : num 0.29
##   .. .. .. .. .. .. ..@ hole   : logi FALSE
##   .. .. .. .. .. .. ..@ ringDir: int 1
##   .. .. .. .. .. .. ..@ coords : num [1:7, 1:2] -77 -77.2 -77.4 -77.3 -77.8 ...
##   .. .. .. .. .. .. .. ..- attr(*, "dimnames")=List of 2
##   .. .. .. .. .. .. .. .. ..$ : NULL
##   .. .. .. .. .. .. .. .. ..$ : chr [1:2] "x" "y"
##   .. .. .. ..@ plotOrder: int [1:3] 1 2 3
##   .. .. .. ..@ labpt    : num [1:2] -77.9 24.5
##   .. .. .. ..@ ID       : chr "17"
##   .. .. .. ..@ area     : num 1.4
##   .. ..$ :Formal class 'Polygons' [package "sp"] with 5 slots
##   .. .. .. ..@ Polygons :List of 1
##   .. .. .. .. ..$ :Formal class 'Polygon' [package "sp"] with 5 slots
##   .. .. .. .. .. .. ..@ labpt  : num [1:2] 17.8 44.2
##   .. .. .. .. .. .. ..@ area   : num 5.7
##   .. .. .. .. .. .. ..@ hole   : logi FALSE
##   .. .. .. .. .. .. ..@ ringDir: int 1
##   .. .. .. .. .. .. ..@ coords : num [1:22, 1:2] 19 19.4 19.1 19.6 19.5 ...
##   .. .. .. .. .. .. .. ..- attr(*, "dimnames")=List of 2
##   .. .. .. .. .. .. .. .. ..$ : NULL
##   .. .. .. .. .. .. .. .. ..$ : chr [1:2] "x" "y"
##   .. .. .. ..@ plotOrder: int 1
##   .. .. .. ..@ labpt    : num [1:2] 17.8 44.2
##   .. .. .. ..@ ID       : chr "18"
##   .. .. .. ..@ area     : num 5.7
##   .. ..$ :Formal class 'Polygons' [package "sp"] with 5 slots
##   .. .. .. ..@ Polygons :List of 1
##   .. .. .. .. ..$ :Formal class 'Polygon' [package "sp"] with 5 slots
##   .. .. .. .. .. .. ..@ labpt  : num [1:2] 28 53.5
##   .. .. .. .. .. .. ..@ area   : num 28.3
##   .. .. .. .. .. .. ..@ hole   : logi FALSE
##   .. .. .. .. .. .. ..@ ringDir: int 1
##   .. .. .. .. .. .. ..@ coords : num [1:44, 1:2] 23.5 24.5 25.5 25.8 26.6 ...
##   .. .. .. .. .. .. .. ..- attr(*, "dimnames")=List of 2
##   .. .. .. .. .. .. .. .. ..$ : NULL
##   .. .. .. .. .. .. .. .. ..$ : chr [1:2] "x" "y"
##   .. .. .. ..@ plotOrder: int 1
##   .. .. .. ..@ labpt    : num [1:2] 28 53.5
##   .. .. .. ..@ ID       : chr "19"
##   .. .. .. ..@ area     : num 28.3
##   .. ..$ :Formal class 'Polygons' [package "sp"] with 5 slots
##   .. .. .. ..@ Polygons :List of 1
##   .. .. .. .. ..$ :Formal class 'Polygon' [package "sp"] with 5 slots
##   .. .. .. .. .. .. ..@ labpt  : num [1:2] -88.7 17.2
##   .. .. .. .. .. .. ..@ area   : num 1.87
##   .. .. .. .. .. .. ..@ hole   : logi FALSE
##   .. .. .. .. .. .. ..@ ringDir: int 1
##   .. .. .. .. .. .. ..@ coords : num [1:20, 1:2] -89.1 -89.2 -89 -88.8 -88.5 ...
##   .. .. .. .. .. .. .. ..- attr(*, "dimnames")=List of 2
##   .. .. .. .. .. .. .. .. ..$ : NULL
##   .. .. .. .. .. .. .. .. ..$ : chr [1:2] "x" "y"
##   .. .. .. ..@ plotOrder: int 1
##   .. .. .. ..@ labpt    : num [1:2] -88.7 17.2
##   .. .. .. ..@ ID       : chr "20"
##   .. .. .. ..@ area     : num 1.87
##   .. ..$ :Formal class 'Polygons' [package "sp"] with 5 slots
##   .. .. .. ..@ Polygons :List of 1
##   .. .. .. .. ..$ :Formal class 'Polygon' [package "sp"] with 5 slots
##   .. .. .. .. .. .. ..@ labpt  : num [1:2] -64.6 -16.7
##   .. .. .. .. .. .. ..@ area   : num 92.1
##   .. .. .. .. .. .. ..@ hole   : logi FALSE
##   .. .. .. .. .. .. ..@ ringDir: int 1
##   .. .. .. .. .. .. ..@ coords : num [1:60, 1:2] -62.8 -64 -64.4 -65 -66.3 ...
##   .. .. .. .. .. .. .. ..- attr(*, "dimnames")=List of 2
##   .. .. .. .. .. .. .. .. ..$ : NULL
##   .. .. .. .. .. .. .. .. ..$ : chr [1:2] "x" "y"
##   .. .. .. ..@ plotOrder: int 1
##   .. .. .. ..@ labpt    : num [1:2] -64.6 -16.7
##   .. .. .. ..@ ID       : chr "21"
##   .. .. .. ..@ area     : num 92.1
##   .. ..$ :Formal class 'Polygons' [package "sp"] with 5 slots
##   .. .. .. ..@ Polygons :List of 1
##   .. .. .. .. ..$ :Formal class 'Polygon' [package "sp"] with 5 slots
##   .. .. .. .. .. .. ..@ labpt  : num [1:2] -53.1 -10.8
##   .. .. .. .. .. .. ..@ area   : num 710
##   .. .. .. .. .. .. ..@ hole   : logi FALSE
##   .. .. .. .. .. .. ..@ ringDir: int 1
##   .. .. .. .. .. .. ..@ coords : num [1:203, 1:2] -57.6 -56.3 -55.2 -54.5 -53.6 ...
##   .. .. .. .. .. .. .. ..- attr(*, "dimnames")=List of 2
##   .. .. .. .. .. .. .. .. ..$ : NULL
##   .. .. .. .. .. .. .. .. ..$ : chr [1:2] "x" "y"
##   .. .. .. ..@ plotOrder: int 1
##   .. .. .. ..@ labpt    : num [1:2] -53.1 -10.8
##   .. .. .. ..@ ID       : chr "22"
##   .. .. .. ..@ area     : num 710
##   .. ..$ :Formal class 'Polygons' [package "sp"] with 5 slots
##   .. .. .. ..@ Polygons :List of 1
##   .. .. .. .. ..$ :Formal class 'Polygon' [package "sp"] with 5 slots
##   .. .. .. .. .. .. ..@ labpt  : num [1:2] 114.92 4.69
##   .. .. .. .. .. .. ..@ area   : num 0.872
##   .. .. .. .. .. .. ..@ hole   : logi FALSE
##   .. .. .. .. .. .. ..@ ringDir: int 1
##   .. .. .. .. .. .. ..@ coords : num [1:8, 1:2] 114 115 115 115 115 ...
##   .. .. .. .. .. .. .. ..- attr(*, "dimnames")=List of 2
##   .. .. .. .. .. .. .. .. ..$ : NULL
##   .. .. .. .. .. .. .. .. ..$ : chr [1:2] "x" "y"
##   .. .. .. ..@ plotOrder: int 1
##   .. .. .. ..@ labpt    : num [1:2] 114.92 4.69
##   .. .. .. ..@ ID       : chr "23"
##   .. .. .. ..@ area     : num 0.872
##   .. ..$ :Formal class 'Polygons' [package "sp"] with 5 slots
##   .. .. .. ..@ Polygons :List of 1
##   .. .. .. .. ..$ :Formal class 'Polygon' [package "sp"] with 5 slots
##   .. .. .. .. .. .. ..@ labpt  : num [1:2] 90.5 27.4
##   .. .. .. .. .. .. ..@ area   : num 3.59
##   .. .. .. .. .. .. ..@ hole   : logi FALSE
##   .. .. .. .. .. .. ..@ ringDir: int 1
##   .. .. .. .. .. .. ..@ coords : num [1:13, 1:2] 91.7 92.1 92 91.2 90.4 ...
##   .. .. .. .. .. .. .. ..- attr(*, "dimnames")=List of 2
##   .. .. .. .. .. .. .. .. ..$ : NULL
##   .. .. .. .. .. .. .. .. ..$ : chr [1:2] "x" "y"
##   .. .. .. ..@ plotOrder: int 1
##   .. .. .. ..@ labpt    : num [1:2] 90.5 27.4
##   .. .. .. ..@ ID       : chr "24"
##   .. .. .. ..@ area     : num 3.59
##   .. ..$ :Formal class 'Polygons' [package "sp"] with 5 slots
##   .. .. .. ..@ Polygons :List of 1
##   .. .. .. .. ..$ :Formal class 'Polygon' [package "sp"] with 5 slots
##   .. .. .. .. .. .. ..@ labpt  : num [1:2] 23.8 -22.1
##   .. .. .. .. .. .. ..@ area   : num 51.8
##   .. .. .. .. .. .. ..@ hole   : logi FALSE
##   .. .. .. .. .. .. ..@ ringDir: int 1
##   .. .. .. .. .. .. ..@ coords : num [1:40, 1:2] 25.6 25.9 26.2 27.3 27.7 ...
##   .. .. .. .. .. .. .. ..- attr(*, "dimnames")=List of 2
##   .. .. .. .. .. .. .. .. ..$ : NULL
##   .. .. .. .. .. .. .. .. ..$ : chr [1:2] "x" "y"
##   .. .. .. ..@ plotOrder: int 1
##   .. .. .. ..@ labpt    : num [1:2] 23.8 -22.1
##   .. .. .. ..@ ID       : chr "25"
##   .. .. .. ..@ area     : num 51.8
##   .. ..$ :Formal class 'Polygons' [package "sp"] with 5 slots
##   .. .. .. ..@ Polygons :List of 1
##   .. .. .. .. ..$ :Formal class 'Polygon' [package "sp"] with 5 slots
##   .. .. .. .. .. .. ..@ labpt  : num [1:2] 20.37 6.54
##   .. .. .. .. .. .. ..@ area   : num 50.9
##   .. .. .. .. .. .. ..@ hole   : logi FALSE
##   .. .. .. .. .. .. ..@ ringDir: int 1
##   .. .. .. .. .. .. ..@ coords : num [1:62, 1:2] 15.3 16.1 16.3 16.5 16.7 ...
##   .. .. .. .. .. .. .. ..- attr(*, "dimnames")=List of 2
##   .. .. .. .. .. .. .. .. ..$ : NULL
##   .. .. .. .. .. .. .. .. ..$ : chr [1:2] "x" "y"
##   .. .. .. ..@ plotOrder: int 1
##   .. .. .. ..@ labpt    : num [1:2] 20.37 6.54
##   .. .. .. ..@ ID       : chr "26"
##   .. .. .. ..@ area     : num 50.9
##   .. ..$ :Formal class 'Polygons' [package "sp"] with 5 slots
##   .. .. .. ..@ Polygons :List of 30
##   .. .. .. .. ..$ :Formal class 'Polygon' [package "sp"] with 5 slots
##   .. .. .. .. .. .. ..@ labpt  : num [1:2] -63.3 46.4
##   .. .. .. .. .. .. ..@ area   : num 0.873
##   .. .. .. .. .. .. ..@ hole   : logi FALSE
##   .. .. .. .. .. .. ..@ ringDir: int 1
##   .. .. .. .. .. .. ..@ coords : num [1:9, 1:2] -63.7 -62.9 -62 -62.5 -62.9 ...
##   .. .. .. .. .. .. .. ..- attr(*, "dimnames")=List of 2
##   .. .. .. .. .. .. .. .. ..$ : NULL
##   .. .. .. .. .. .. .. .. ..$ : chr [1:2] "x" "y"
##   .. .. .. .. ..$ :Formal class 'Polygon' [package "sp"] with 5 slots
##   .. .. .. .. .. .. ..@ labpt  : num [1:2] -63.1 49.5
##   .. .. .. .. .. .. ..@ area   : num 0.97
##   .. .. .. .. .. .. ..@ hole   : logi FALSE
##   .. .. .. .. .. .. ..@ ringDir: int 1
##   .. .. .. .. .. .. ..@ coords : num [1:8, 1:2] -61.8 -62.3 -63.6 -64.5 -64.2 ...
##   .. .. .. .. .. .. .. ..- attr(*, "dimnames")=List of 2
##   .. .. .. .. .. .. .. .. ..$ : NULL
##   .. .. .. .. .. .. .. .. ..$ : chr [1:2] "x" "y"
##   .. .. .. .. ..$ :Formal class 'Polygon' [package "sp"] with 5 slots
##   .. .. .. .. .. .. ..@ labpt  : num [1:2] -126 49.6
##   .. .. .. .. .. .. ..@ area   : num 4.06
##   .. .. .. .. .. .. ..@ hole   : logi FALSE
##   .. .. .. .. .. .. ..@ ringDir: int 1
##   .. .. .. .. .. .. ..@ coords : num [1:16, 1:2] -124 -124 -126 -126 -127 ...
##   .. .. .. .. .. .. .. ..- attr(*, "dimnames")=List of 2
##   .. .. .. .. .. .. .. .. ..$ : NULL
##   .. .. .. .. .. .. .. .. ..$ : chr [1:2] "x" "y"
##   .. .. .. .. ..$ :Formal class 'Polygon' [package "sp"] with 5 slots
##   .. .. .. .. .. .. ..@ labpt  : num [1:2] -56 48.7
##   .. .. .. .. .. .. ..@ area   : num 15.2
##   .. .. .. .. .. .. ..@ hole   : logi FALSE
##   .. .. .. .. .. .. ..@ ringDir: int 1
##   .. .. .. .. .. .. ..@ coords : num [1:33, 1:2] -56.1 -56.8 -56.1 -55.5 -55.8 ...
##   .. .. .. .. .. .. .. ..- attr(*, "dimnames")=List of 2
##   .. .. .. .. .. .. .. .. ..$ : NULL
##   .. .. .. .. .. .. .. .. ..$ : chr [1:2] "x" "y"
##   .. .. .. .. ..$ :Formal class 'Polygon' [package "sp"] with 5 slots
##   .. .. .. .. .. .. ..@ labpt  : num [1:2] -132.3 53.4
##   .. .. .. .. .. .. ..@ area   : num 1.57
##   .. .. .. .. .. .. ..@ hole   : logi FALSE
##   .. .. .. .. .. .. ..@ ringDir: int 1
##   .. .. .. .. .. .. ..@ coords : num [1:11, 1:2] -133 -133 -132 -132 -131 ...
##   .. .. .. .. .. .. .. ..- attr(*, "dimnames")=List of 2
##   .. .. .. .. .. .. .. .. ..$ : NULL
##   .. .. .. .. .. .. .. .. ..$ : chr [1:2] "x" "y"
##   .. .. .. .. ..$ :Formal class 'Polygon' [package "sp"] with 5 slots
##   .. .. .. .. .. .. ..@ labpt  : num [1:2] -79.8 62
##   .. .. .. .. .. .. ..@ area   : num 0.532
##   .. .. .. .. .. .. ..@ hole   : logi FALSE
##   .. .. .. .. .. .. ..@ ringDir: int 1
##   .. .. .. .. .. .. ..@ coords : num [1:8, 1:2] -79.3 -79.7 -80.1 -80.4 -80.3 ...
##   .. .. .. .. .. .. .. ..- attr(*, "dimnames")=List of 2
##   .. .. .. .. .. .. .. .. ..$ : NULL
##   .. .. .. .. .. .. .. .. ..$ : chr [1:2] "x" "y"
##   .. .. .. .. ..$ :Formal class 'Polygon' [package "sp"] with 5 slots
##   .. .. .. .. .. .. ..@ labpt  : num [1:2] -83 62.6
##   .. .. .. .. .. .. ..@ area   : num 1.04
##   .. .. .. .. .. .. ..@ hole   : logi FALSE
##   .. .. .. .. .. .. ..@ ringDir: int 1
##   .. .. .. .. .. .. ..@ coords : num [1:7, 1:2] -81.9 -83.1 -83.8 -84 -83.3 ...
##   .. .. .. .. .. .. .. ..- attr(*, "dimnames")=List of 2
##   .. .. .. .. .. .. .. .. ..$ : NULL
##   .. .. .. .. .. .. .. .. ..$ : chr [1:2] "x" "y"
##   .. .. .. .. ..$ :Formal class 'Polygon' [package "sp"] with 5 slots
##   .. .. .. .. .. .. ..@ labpt  : num [1:2] -84.1 64.3
##   .. .. .. .. .. .. ..@ area   : num 7.97
##   .. .. .. .. .. .. ..@ hole   : logi FALSE
##   .. .. .. .. .. .. ..@ ringDir: int 1
##   .. .. .. .. .. .. ..@ coords : num [1:20, 1:2] -85.2 -85 -84.5 -83.9 -82.8 ...
##   .. .. .. .. .. .. .. ..- attr(*, "dimnames")=List of 2
##   .. .. .. .. .. .. .. .. ..$ : NULL
##   .. .. .. .. .. .. .. .. ..$ : chr [1:2] "x" "y"
##   .. .. .. .. ..$ :Formal class 'Polygon' [package "sp"] with 5 slots
##   .. .. .. .. .. .. ..@ labpt  : num [1:2] -76.2 67.7
##   .. .. .. .. .. .. ..@ area   : num 1.92
##   .. .. .. .. .. .. ..@ hole   : logi FALSE
##   .. .. .. .. .. .. ..@ ringDir: int 1
##   .. .. .. .. .. .. ..@ coords : num [1:9, 1:2] -75.9 -77 -77.2 -76.8 -75.9 ...
##   .. .. .. .. .. .. .. ..- attr(*, "dimnames")=List of 2
##   .. .. .. .. .. .. .. .. ..$ : NULL
##   .. .. .. .. .. .. .. .. ..$ : chr [1:2] "x" "y"
##   .. .. .. .. ..$ :Formal class 'Polygon' [package "sp"] with 5 slots
##   .. .. .. .. .. .. ..@ labpt  : num [1:2] -97.6 69.4
##   .. .. .. .. .. .. ..@ area   : num 2.94
##   .. .. .. .. .. .. ..@ hole   : logi FALSE
##   .. .. .. .. .. .. ..@ ringDir: int 1
##   .. .. .. .. .. .. ..@ coords : num [1:11, 1:2] -95.6 -96.3 -97.6 -98.4 -99.8 ...
##   .. .. .. .. .. .. .. ..- attr(*, "dimnames")=List of 2
##   .. .. .. .. .. .. .. .. ..$ : NULL
##   .. .. .. .. .. .. .. .. ..$ : chr [1:2] "x" "y"
##   .. .. .. .. ..$ :Formal class 'Polygon' [package "sp"] with 5 slots
##   .. .. .. .. .. .. ..@ labpt  : num [1:2] -101.6 57.7
##   .. .. .. .. .. .. ..@ area   : num 1281
##   .. .. .. .. .. .. ..@ hole   : logi FALSE
##   .. .. .. .. .. .. ..@ ringDir: int 1
##   .. .. .. .. .. .. ..@ coords : num [1:272, 1:2] -90.5 -90.6 -89.2 -88 -88.3 ...
##   .. .. .. .. .. .. .. ..- attr(*, "dimnames")=List of 2
##   .. .. .. .. .. .. .. .. ..$ : NULL
##   .. .. .. .. .. .. .. .. ..$ : chr [1:2] "x" "y"
##   .. .. .. .. ..$ :Formal class 'Polygon' [package "sp"] with 5 slots
##   .. .. .. .. .. .. ..@ labpt  : num [1:2] -110.5 70.8
##   .. .. .. .. .. .. ..@ area   : num 54.4
##   .. .. .. .. .. .. ..@ hole   : logi FALSE
##   .. .. .. .. .. .. ..@ ringDir: int 1
##   .. .. .. .. .. .. ..@ coords : num [1:45, 1:2] -114 -115 -112 -111 -110 ...
##   .. .. .. .. .. .. .. ..- attr(*, "dimnames")=List of 2
##   .. .. .. .. .. .. .. .. ..$ : NULL
##   .. .. .. .. .. .. .. .. ..$ : chr [1:2] "x" "y"
##   .. .. .. .. ..$ :Formal class 'Polygon' [package "sp"] with 5 slots
##   .. .. .. .. .. .. ..@ labpt  : num [1:2] -105.6 73.3
##   .. .. .. .. .. .. ..@ area   : num 1.16
##   .. .. .. .. .. .. ..@ hole   : logi FALSE
##   .. .. .. .. .. .. ..@ ringDir: int 1
##   .. .. .. .. .. .. ..@ coords : num [1:6, 1:2] -104 -105 -107 -107 -105 ...
##   .. .. .. .. .. .. .. ..- attr(*, "dimnames")=List of 2
##   .. .. .. .. .. .. .. .. ..$ : NULL
##   .. .. .. .. .. .. .. .. ..$ : chr [1:2] "x" "y"
##   .. .. .. .. ..$ :Formal class 'Polygon' [package "sp"] with 5 slots
##   .. .. .. .. .. .. ..@ labpt  : num [1:2] -78.8 73.2
##   .. .. .. .. .. .. ..@ area   : num 3.15
##   .. .. .. .. .. .. ..@ hole   : logi FALSE
##   .. .. .. .. .. .. ..@ ringDir: int 1
##   .. .. .. .. .. .. ..@ coords : num [1:11, 1:2] -76.3 -76.3 -77.3 -78.4 -79.5 ...
##   .. .. .. .. .. .. .. ..- attr(*, "dimnames")=List of 2
##   .. .. .. .. .. .. .. .. ..$ : NULL
##   .. .. .. .. .. .. .. .. ..$ : chr [1:2] "x" "y"
##   .. .. .. .. ..$ :Formal class 'Polygon' [package "sp"] with 5 slots
##   .. .. .. .. .. .. ..@ labpt  : num [1:2] -74.9 68.6
##   .. .. .. .. .. .. ..@ area   : num 118
##   .. .. .. .. .. .. ..@ hole   : logi FALSE
##   .. .. .. .. .. .. ..@ ringDir: int 1
##   .. .. .. .. .. .. ..@ coords : num [1:73, 1:2] -86.6 -85.8 -84.9 -82.3 -80.6 ...
##   .. .. .. .. .. .. .. ..- attr(*, "dimnames")=List of 2
##   .. .. .. .. .. .. .. .. ..$ : NULL
##   .. .. .. .. .. .. .. .. ..$ : chr [1:2] "x" "y"
##   .. .. .. .. ..$ :Formal class 'Polygon' [package "sp"] with 5 slots
##   .. .. .. .. .. .. ..@ labpt  : num [1:2] -99.2 72.6
##   .. .. .. .. .. .. ..@ area   : num 9.4
##   .. .. .. .. .. .. ..@ hole   : logi FALSE
##   .. .. .. .. .. .. ..@ ringDir: int 1
##   .. .. .. .. .. .. ..@ coords : num [1:15, 1:2] -100.4 -99.2 -97.4 -97.1 -98.1 ...
##   .. .. .. .. .. .. .. ..- attr(*, "dimnames")=List of 2
##   .. .. .. .. .. .. .. .. ..$ : NULL
##   .. .. .. .. .. .. .. .. ..$ : chr [1:2] "x" "y"
##   .. .. .. .. ..$ :Formal class 'Polygon' [package "sp"] with 5 slots
##   .. .. .. .. .. .. ..@ labpt  : num [1:2] -93.8 73.3
##   .. .. .. .. .. .. ..@ area   : num 7.12
##   .. .. .. .. .. .. ..@ hole   : logi FALSE
##   .. .. .. .. .. .. ..@ ringDir: int 1
##   .. .. .. .. .. .. ..@ coords : num [1:11, 1:2] -93.2 -94.3 -95.4 -96 -96 ...
##   .. .. .. .. .. .. .. ..- attr(*, "dimnames")=List of 2
##   .. .. .. .. .. .. .. .. ..$ : NULL
##   .. .. .. .. .. .. .. .. ..$ : chr [1:2] "x" "y"
##   .. .. .. .. ..$ :Formal class 'Polygon' [package "sp"] with 5 slots
##   .. .. .. .. .. .. ..@ labpt  : num [1:2] -121.5 72.9
##   .. .. .. .. .. .. ..@ area   : num 20.1
##   .. .. .. .. .. .. ..@ hole   : logi FALSE
##   .. .. .. .. .. .. ..@ ringDir: int 1
##   .. .. .. .. .. .. ..@ coords : num [1:17, 1:2] -120 -123 -124 -126 -126 ...
##   .. .. .. .. .. .. .. ..- attr(*, "dimnames")=List of 2
##   .. .. .. .. .. .. .. .. ..$ : NULL
##   .. .. .. .. .. .. .. .. ..$ : chr [1:2] "x" "y"
##   .. .. .. .. ..$ :Formal class 'Polygon' [package "sp"] with 5 slots
##   .. .. .. .. .. .. ..@ labpt  : num [1:2] -95.1 75.1
##   .. .. .. .. .. .. ..@ area   : num 2.18
##   .. .. .. .. .. .. ..@ hole   : logi FALSE
##   .. .. .. .. .. .. ..@ ringDir: int 1
##   .. .. .. .. .. .. ..@ coords : num [1:8, 1:2] -93.6 -94.2 -95.6 -96.8 -96.3 ...
##   .. .. .. .. .. .. .. ..- attr(*, "dimnames")=List of 2
##   .. .. .. .. .. .. .. .. ..$ : NULL
##   .. .. .. .. .. .. .. .. ..$ : chr [1:2] "x" "y"
##   .. .. .. .. ..$ :Formal class 'Polygon' [package "sp"] with 5 slots
##   .. .. .. .. .. .. ..@ labpt  : num [1:2] -99.8 75.8
##   .. .. .. .. .. .. ..@ area   : num 6.11
##   .. .. .. .. .. .. ..@ hole   : logi FALSE
##   .. .. .. .. .. .. ..@ ringDir: int 1
##   .. .. .. .. .. .. ..@ coords : num [1:13, 1:2] -98.5 -97.7 -97.7 -98.2 -99.8 ...
##   .. .. .. .. .. .. .. ..- attr(*, "dimnames")=List of 2
##   .. .. .. .. .. .. .. .. ..$ : NULL
##   .. .. .. .. .. .. .. .. ..$ : chr [1:2] "x" "y"
##   .. .. .. .. ..$ :Formal class 'Polygon' [package "sp"] with 5 slots
##   .. .. .. .. .. .. ..@ labpt  : num [1:2] -111.6 75.5
##   .. .. .. .. .. .. ..@ area   : num 14.3
##   .. .. .. .. .. .. ..@ hole   : logi FALSE
##   .. .. .. .. .. .. ..@ ringDir: int 1
##   .. .. .. .. .. .. ..@ coords : num [1:22, 1:2] -108 -108 -107 -106 -106 ...
##   .. .. .. .. .. .. .. ..- attr(*, "dimnames")=List of 2
##   .. .. .. .. .. .. .. .. ..$ : NULL
##   .. .. .. .. .. .. .. .. ..$ : chr [1:2] "x" "y"
##   .. .. .. .. ..$ :Formal class 'Polygon' [package "sp"] with 5 slots
##   .. .. .. .. .. .. ..@ labpt  : num [1:2] -88.1 75.4
##   .. .. .. .. .. .. ..@ area   : num 19.2
##   .. .. .. .. .. .. ..@ hole   : logi FALSE
##   .. .. .. .. .. .. ..@ ringDir: int 1
##   .. .. .. .. .. .. ..@ coords : num [1:28, 1:2] -94.7 -93.6 -91.6 -90.7 -91 ...
##   .. .. .. .. .. .. .. ..- attr(*, "dimnames")=List of 2
##   .. .. .. .. .. .. .. .. ..$ : NULL
##   .. .. .. .. .. .. .. .. ..$ : chr [1:2] "x" "y"
##   .. .. .. .. ..$ :Formal class 'Polygon' [package "sp"] with 5 slots
##   .. .. .. .. .. .. ..@ labpt  : num [1:2] -119.2 76.8
##   .. .. .. .. .. .. ..@ area   : num 6.08
##   .. .. .. .. .. .. ..@ hole   : logi FALSE
##   .. .. .. .. .. .. ..@ ringDir: int 1
##   .. .. .. .. .. .. ..@ coords : num [1:12, 1:2] -116 -116 -117 -118 -120 ...
##   .. .. .. .. .. .. .. ..- attr(*, "dimnames")=List of 2
##   .. .. .. .. .. .. .. .. ..$ : NULL
##   .. .. .. .. .. .. .. .. ..$ : chr [1:2] "x" "y"
##   .. .. .. .. ..$ :Formal class 'Polygon' [package "sp"] with 5 slots
##   .. .. .. .. .. .. ..@ labpt  : num [1:2] -95.1 77.7
##   .. .. .. .. .. .. ..@ area   : num 0.711
##   .. .. .. .. .. .. ..@ hole   : logi FALSE
##   .. .. .. .. .. .. ..@ ringDir: int 1
##   .. .. .. .. .. .. ..@ coords : num [1:7, 1:2] -93.8 -94.3 -96.2 -96.4 -94.4 ...
##   .. .. .. .. .. .. .. ..- attr(*, "dimnames")=List of 2
##   .. .. .. .. .. .. .. .. ..$ : NULL
##   .. .. .. .. .. .. .. .. ..$ : chr [1:2] "x" "y"
##   .. .. .. .. ..$ :Formal class 'Polygon' [package "sp"] with 5 slots
##   .. .. .. .. .. .. ..@ labpt  : num [1:2] -111.7 77.8
##   .. .. .. .. .. .. ..@ area   : num 1.69
##   .. .. .. .. .. .. ..@ hole   : logi FALSE
##   .. .. .. .. .. .. ..@ ringDir: int 1
##   .. .. .. .. .. .. ..@ coords : num [1:7, 1:2] -110 -112 -114 -113 -111 ...
##   .. .. .. .. .. .. .. ..- attr(*, "dimnames")=List of 2
##   .. .. .. .. .. .. .. .. ..$ : NULL
##   .. .. .. .. .. .. .. .. ..$ : chr [1:2] "x" "y"
##   .. .. .. .. ..$ :Formal class 'Polygon' [package "sp"] with 5 slots
##   .. .. .. .. .. .. ..@ labpt  : num [1:2] -111.3 78.6
##   .. .. .. .. .. .. ..@ area   : num 0.793
##   .. .. .. .. .. .. ..@ hole   : logi FALSE
##   .. .. .. .. .. .. ..@ ringDir: int 1
##   .. .. .. .. .. .. ..@ coords : num [1:7, 1:2] -110 -111 -113 -113 -112 ...
##   .. .. .. .. .. .. .. ..- attr(*, "dimnames")=List of 2
##   .. .. .. .. .. .. .. .. ..$ : NULL
##   .. .. .. .. .. .. .. .. ..$ : chr [1:2] "x" "y"
##   .. .. .. .. ..$ :Formal class 'Polygon' [package "sp"] with 5 slots
##   .. .. .. .. .. .. ..@ labpt  : num [1:2] -97.2 78.4
##   .. .. .. .. .. .. ..@ area   : num 2.14
##   .. .. .. .. .. .. ..@ hole   : logi FALSE
##   .. .. .. .. .. .. ..@ ringDir: int 1
##   .. .. .. .. .. .. ..@ coords : num [1:9, 1:2] -95.8 -97.3 -98.1 -98.6 -98.6 ...
##   .. .. .. .. .. .. .. ..- attr(*, "dimnames")=List of 2
##   .. .. .. .. .. .. .. .. ..$ : NULL
##   .. .. .. .. .. .. .. .. ..$ : chr [1:2] "x" "y"
##   .. .. .. .. ..$ :Formal class 'Polygon' [package "sp"] with 5 slots
##   .. .. .. .. .. .. ..@ labpt  : num [1:2] -102.6 78.6
##   .. .. .. .. .. .. ..@ area   : num 3.98
##   .. .. .. .. .. .. ..@ hole   : logi FALSE
##   .. .. .. .. .. .. ..@ ringDir: int 1
##   .. .. .. .. .. .. ..@ coords : num [1:11, 1:2] -100.1 -99.7 -101.3 -102.9 -105.2 ...
##   .. .. .. .. .. .. .. ..- attr(*, "dimnames")=List of 2
##   .. .. .. .. .. .. .. .. ..$ : NULL
##   .. .. .. .. .. .. .. .. ..$ : chr [1:2] "x" "y"
##   .. .. .. .. ..$ :Formal class 'Polygon' [package "sp"] with 5 slots
##   .. .. .. .. .. .. ..@ labpt  : num [1:2] -91.6 79.7
##   .. .. .. .. .. .. ..@ area   : num 19.4
##   .. .. .. .. .. .. ..@ hole   : logi FALSE
##   .. .. .. .. .. .. ..@ ringDir: int 1
##   .. .. .. .. .. .. ..@ coords : num [1:21, 1:2] -87 -85.8 -87.2 -89 -90.8 ...
##   .. .. .. .. .. .. .. ..- attr(*, "dimnames")=List of 2
##   .. .. .. .. .. .. .. .. ..$ : NULL
##   .. .. .. .. .. .. .. .. ..$ : chr [1:2] "x" "y"
##   .. .. .. .. ..$ :Formal class 'Polygon' [package "sp"] with 5 slots
##   .. .. .. .. .. .. ..@ labpt  : num [1:2] -78.6 80.4
##   .. .. .. .. .. .. ..@ area   : num 105
##   .. .. .. .. .. .. ..@ hole   : logi FALSE
##   .. .. .. .. .. .. ..@ ringDir: int 1
##   .. .. .. .. .. .. ..@ coords : num [1:65, 1:2] -68.5 -65.8 -63.7 -61.9 -61.9 ...
##   .. .. .. .. .. .. .. ..- attr(*, "dimnames")=List of 2
##   .. .. .. .. .. .. .. .. ..$ : NULL
##   .. .. .. .. .. .. .. .. ..$ : chr [1:2] "x" "y"
##   .. .. .. ..@ plotOrder: int [1:30] 11 15 30 12 18 29 22 4 21 16 ...
##   .. .. .. ..@ labpt    : num [1:2] -101.6 57.7
##   .. .. .. ..@ ID       : chr "27"
##   .. .. .. ..@ area     : num 1713
##   .. ..$ :Formal class 'Polygons' [package "sp"] with 5 slots
##   .. .. .. ..@ Polygons :List of 1
##   .. .. .. .. ..$ :Formal class 'Polygon' [package "sp"] with 5 slots
##   .. .. .. .. .. .. ..@ labpt  : num [1:2] 8.12 46.79
##   .. .. .. .. .. .. ..@ area   : num 5.44
##   .. .. .. .. .. .. ..@ hole   : logi FALSE
##   .. .. .. .. .. .. ..@ ringDir: int 1
##   .. .. .. .. .. .. ..@ coords : num [1:24, 1:2] 9.59 9.63 9.48 9.93 10.44 ...
##   .. .. .. .. .. .. .. ..- attr(*, "dimnames")=List of 2
##   .. .. .. .. .. .. .. .. ..$ : NULL
##   .. .. .. .. .. .. .. .. ..$ : chr [1:2] "x" "y"
##   .. .. .. ..@ plotOrder: int 1
##   .. .. .. ..@ labpt    : num [1:2] 8.12 46.79
##   .. .. .. ..@ ID       : chr "28"
##   .. .. .. ..@ area     : num 5.44
##   .. ..$ :Formal class 'Polygons' [package "sp"] with 5 slots
##   .. .. .. ..@ Polygons :List of 2
##   .. .. .. .. ..$ :Formal class 'Polygon' [package "sp"] with 5 slots
##   .. .. .. .. .. .. ..@ labpt  : num [1:2] -70.2 -54.2
##   .. .. .. .. .. .. ..@ area   : num 8.76
##   .. .. .. .. .. .. ..@ hole   : logi FALSE
##   .. .. .. .. .. .. ..@ ringDir: int 1
##   .. .. .. .. .. .. ..@ coords : num [1:20, 1:2] -68.6 -68.6 -67.6 -67 -67.3 ...
##   .. .. .. .. .. .. .. ..- attr(*, "dimnames")=List of 2
##   .. .. .. .. .. .. .. .. ..$ : NULL
##   .. .. .. .. .. .. .. .. ..$ : chr [1:2] "x" "y"
##   .. .. .. .. ..$ :Formal class 'Polygon' [package "sp"] with 5 slots
##   .. .. .. .. .. .. ..@ labpt  : num [1:2] -71.7 -37.3
##   .. .. .. .. .. .. ..@ area   : num 77.7
##   .. .. .. .. .. .. ..@ hole   : logi FALSE
##   .. .. .. .. .. .. ..@ ringDir: int 1
##   .. .. .. .. .. .. ..@ coords : num [1:94, 1:2] -68.2 -67.8 -67.1 -67 -67.3 ...
##   .. .. .. .. .. .. .. ..- attr(*, "dimnames")=List of 2
##   .. .. .. .. .. .. .. .. ..$ : NULL
##   .. .. .. .. .. .. .. .. ..$ : chr [1:2] "x" "y"
##   .. .. .. ..@ plotOrder: int [1:2] 2 1
##   .. .. .. ..@ labpt    : num [1:2] -71.7 -37.3
##   .. .. .. ..@ ID       : chr "29"
##   .. .. .. ..@ area     : num 86.5
##   .. ..$ :Formal class 'Polygons' [package "sp"] with 5 slots
##   .. .. .. ..@ Polygons :List of 2
##   .. .. .. .. ..$ :Formal class 'Polygon' [package "sp"] with 5 slots
##   .. .. .. .. .. .. ..@ labpt  : num [1:2] 109.7 19.2
##   .. .. .. .. .. .. ..@ area   : num 2.98
##   .. .. .. .. .. .. ..@ hole   : logi FALSE
##   .. .. .. .. .. .. ..@ ringDir: int 1
##   .. .. .. .. .. .. ..@ coords : num [1:10, 1:2] 110 109 109 109 109 ...
##   .. .. .. .. .. .. .. ..- attr(*, "dimnames")=List of 2
##   .. .. .. .. .. .. .. .. ..$ : NULL
##   .. .. .. .. .. .. .. .. ..$ : chr [1:2] "x" "y"
##   .. .. .. .. ..$ :Formal class 'Polygon' [package "sp"] with 5 slots
##   .. .. .. .. .. .. ..@ labpt  : num [1:2] 103.9 36.6
##   .. .. .. .. .. .. ..@ area   : num 952
##   .. .. .. .. .. .. ..@ hole   : logi FALSE
##   .. .. .. .. .. .. ..@ ringDir: int 1
##   .. .. .. .. .. .. ..@ coords : num [1:230, 1:2] 128 129 131 131 133 ...
##   .. .. .. .. .. .. .. ..- attr(*, "dimnames")=List of 2
##   .. .. .. .. .. .. .. .. ..$ : NULL
##   .. .. .. .. .. .. .. .. ..$ : chr [1:2] "x" "y"
##   .. .. .. ..@ plotOrder: int [1:2] 2 1
##   .. .. .. ..@ labpt    : num [1:2] 103.9 36.6
##   .. .. .. ..@ ID       : chr "30"
##   .. .. .. ..@ area     : num 955
##   .. ..$ :Formal class 'Polygons' [package "sp"] with 5 slots
##   .. .. .. ..@ Polygons :List of 1
##   .. .. .. .. ..$ :Formal class 'Polygon' [package "sp"] with 5 slots
##   .. .. .. .. .. .. ..@ labpt  : num [1:2] -5.61 7.55
##   .. .. .. .. .. .. ..@ area   : num 27
##   .. .. .. .. .. .. ..@ hole   : logi FALSE
##   .. .. .. .. .. .. ..@ ringDir: int 1
##   .. .. .. .. .. .. ..@ coords : num [1:46, 1:2] -2.86 -3.31 -4.01 -4.65 -5.83 ...
##   .. .. .. .. .. .. .. ..- attr(*, "dimnames")=List of 2
##   .. .. .. .. .. .. .. .. ..$ : NULL
##   .. .. .. .. .. .. .. .. ..$ : chr [1:2] "x" "y"
##   .. .. .. ..@ plotOrder: int 1
##   .. .. .. ..@ labpt    : num [1:2] -5.61 7.55
##   .. .. .. ..@ ID       : chr "31"
##   .. .. .. ..@ area     : num 27
##   .. ..$ :Formal class 'Polygons' [package "sp"] with 5 slots
##   .. .. .. ..@ Polygons :List of 1
##   .. .. .. .. ..$ :Formal class 'Polygon' [package "sp"] with 5 slots
##   .. .. .. .. .. .. ..@ labpt  : num [1:2] 12.61 5.66
##   .. .. .. .. .. .. ..@ area   : num 37.6
##   .. .. .. .. .. .. ..@ hole   : logi FALSE
##   .. .. .. .. .. .. ..@ ringDir: int 1
##   .. .. .. .. .. .. ..@ coords : num [1:61, 1:2] 13.1 13 12.4 11.8 11.3 ...
##   .. .. .. .. .. .. .. ..- attr(*, "dimnames")=List of 2
##   .. .. .. .. .. .. .. .. ..$ : NULL
##   .. .. .. .. .. .. .. .. ..$ : chr [1:2] "x" "y"
##   .. .. .. ..@ plotOrder: int 1
##   .. .. .. ..@ labpt    : num [1:2] 12.61 5.66
##   .. .. .. ..@ ID       : chr "32"
##   .. .. .. ..@ area     : num 37.6
##   .. ..$ :Formal class 'Polygons' [package "sp"] with 5 slots
##   .. .. .. ..@ Polygons :List of 1
##   .. .. .. .. ..$ :Formal class 'Polygon' [package "sp"] with 5 slots
##   .. .. .. .. .. .. ..@ labpt  : num [1:2] 23.58 -2.85
##   .. .. .. .. .. .. ..@ area   : num 190
##   .. .. .. .. .. .. ..@ hole   : logi FALSE
##   .. .. .. .. .. .. ..@ ringDir: int 1
##   .. .. .. .. .. .. ..@ coords : num [1:122, 1:2] 30.8 30.8 31.2 30.9 30.5 ...
##   .. .. .. .. .. .. .. ..- attr(*, "dimnames")=List of 2
##   .. .. .. .. .. .. .. .. ..$ : NULL
##   .. .. .. .. .. .. .. .. ..$ : chr [1:2] "x" "y"
##   .. .. .. ..@ plotOrder: int 1
##   .. .. .. ..@ labpt    : num [1:2] 23.58 -2.85
##   .. .. .. ..@ ID       : chr "33"
##   .. .. .. ..@ area     : num 190
##   .. ..$ :Formal class 'Polygons' [package "sp"] with 5 slots
##   .. .. .. ..@ Polygons :List of 1
##   .. .. .. .. ..$ :Formal class 'Polygon' [package "sp"] with 5 slots
##   .. .. .. .. .. .. ..@ labpt  : num [1:2] 15.134 -0.838
##   .. .. .. .. .. .. ..@ area   : num 27.6
##   .. .. .. .. .. .. ..@ hole   : logi FALSE
##   .. .. .. .. .. .. ..@ ringDir: int 1
##   .. .. .. .. .. .. ..@ coords : num [1:49, 1:2] 13 12.6 12.3 11.9 11.1 ...
##   .. .. .. .. .. .. .. ..- attr(*, "dimnames")=List of 2
##   .. .. .. .. .. .. .. .. ..$ : NULL
##   .. .. .. .. .. .. .. .. ..$ : chr [1:2] "x" "y"
##   .. .. .. ..@ plotOrder: int 1
##   .. .. .. ..@ labpt    : num [1:2] 15.134 -0.838
##   .. .. .. ..@ ID       : chr "34"
##   .. .. .. ..@ area     : num 27.6
##   .. ..$ :Formal class 'Polygons' [package "sp"] with 5 slots
##   .. .. .. ..@ Polygons :List of 1
##   .. .. .. .. ..$ :Formal class 'Polygon' [package "sp"] with 5 slots
##   .. .. .. .. .. .. ..@ labpt  : num [1:2] -73.08 3.93
##   .. .. .. .. .. .. ..@ area   : num 93.9
##   .. .. .. .. .. .. ..@ hole   : logi FALSE
##   .. .. .. .. .. .. ..@ ringDir: int 1
##   .. .. .. .. .. .. ..@ coords : num [1:100, 1:2] -75.4 -75.8 -76.3 -76.6 -77.4 ...
##   .. .. .. .. .. .. .. ..- attr(*, "dimnames")=List of 2
##   .. .. .. .. .. .. .. .. ..$ : NULL
##   .. .. .. .. .. .. .. .. ..$ : chr [1:2] "x" "y"
##   .. .. .. ..@ plotOrder: int 1
##   .. .. .. ..@ labpt    : num [1:2] -73.08 3.93
##   .. .. .. ..@ ID       : chr "35"
##   .. .. .. ..@ area     : num 93.9
##   .. ..$ :Formal class 'Polygons' [package "sp"] with 5 slots
##   .. .. .. ..@ Polygons :List of 1
##   .. .. .. .. ..$ :Formal class 'Polygon' [package "sp"] with 5 slots
##   .. .. .. .. .. .. ..@ labpt  : num [1:2] -84.18 9.97
##   .. .. .. .. .. .. ..@ area   : num 4.44
##   .. .. .. .. .. .. ..@ hole   : logi FALSE
##   .. .. .. .. .. .. ..@ ringDir: int 1
##   .. .. .. .. .. .. ..@ coords : num [1:36, 1:2] -83 -83.5 -83.7 -83.6 -83.6 ...
##   .. .. .. .. .. .. .. ..- attr(*, "dimnames")=List of 2
##   .. .. .. .. .. .. .. .. ..$ : NULL
##   .. .. .. .. .. .. .. .. ..$ : chr [1:2] "x" "y"
##   .. .. .. ..@ plotOrder: int 1
##   .. .. .. ..@ labpt    : num [1:2] -84.18 9.97
##   .. .. .. ..@ ID       : chr "36"
##   .. .. .. ..@ area     : num 4.44
##   .. ..$ :Formal class 'Polygons' [package "sp"] with 5 slots
##   .. .. .. ..@ Polygons :List of 1
##   .. .. .. .. ..$ :Formal class 'Polygon' [package "sp"] with 5 slots
##   .. .. .. .. .. .. ..@ labpt  : num [1:2] -79 21.6
##   .. .. .. .. .. .. ..@ area   : num 10
##   .. .. .. .. .. .. ..@ hole   : logi FALSE
##   .. .. .. .. .. .. ..@ ringDir: int 1
##   .. .. .. .. .. .. ..@ coords : num [1:42, 1:2] -82.3 -81.4 -80.6 -79.7 -79.3 ...
##   .. .. .. .. .. .. .. ..- attr(*, "dimnames")=List of 2
##   .. .. .. .. .. .. .. .. ..$ : NULL
##   .. .. .. .. .. .. .. .. ..$ : chr [1:2] "x" "y"
##   .. .. .. ..@ plotOrder: int 1
##   .. .. .. ..@ labpt    : num [1:2] -79 21.6
##   .. .. .. ..@ ID       : chr "37"
##   .. .. .. ..@ area     : num 10
##   .. ..$ :Formal class 'Polygons' [package "sp"] with 5 slots
##   .. .. .. ..@ Polygons :List of 1
##   .. .. .. .. ..$ :Formal class 'Polygon' [package "sp"] with 5 slots
##   .. .. .. .. .. .. ..@ labpt  : num [1:2] 33.6 35.3
##   .. .. .. .. .. .. ..@ area   : num 0.375
##   .. .. .. .. .. .. ..@ hole   : logi FALSE
##   .. .. .. .. .. .. ..@ ringDir: int 1
##   .. .. .. .. .. .. ..@ coords : num [1:16, 1:2] 32.7 32.8 32.9 33.7 34.6 ...
##   .. .. .. .. .. .. .. ..- attr(*, "dimnames")=List of 2
##   .. .. .. .. .. .. .. .. ..$ : NULL
##   .. .. .. .. .. .. .. .. ..$ : chr [1:2] "x" "y"
##   .. .. .. ..@ plotOrder: int 1
##   .. .. .. ..@ labpt    : num [1:2] 33.6 35.3
##   .. .. .. ..@ ID       : chr "38"
##   .. .. .. ..@ area     : num 0.375
##   .. ..$ :Formal class 'Polygons' [package "sp"] with 5 slots
##   .. .. .. ..@ Polygons :List of 1
##   .. .. .. .. ..$ :Formal class 'Polygon' [package "sp"] with 5 slots
##   .. .. .. .. .. .. ..@ labpt  : num [1:2] 33 34.9
##   .. .. .. .. .. .. ..@ area   : num 0.613
##   .. .. .. .. .. .. ..@ hole   : logi FALSE
##   .. .. .. .. .. .. ..@ ringDir: int 1
##   .. .. .. .. .. .. ..@ coords : num [1:15, 1:2] 34 34 33 32.5 32.3 ...
##   .. .. .. .. .. .. .. ..- attr(*, "dimnames")=List of 2
##   .. .. .. .. .. .. .. .. ..$ : NULL
##   .. .. .. .. .. .. .. .. ..$ : chr [1:2] "x" "y"
##   .. .. .. ..@ plotOrder: int 1
##   .. .. .. ..@ labpt    : num [1:2] 33 34.9
##   .. .. .. ..@ ID       : chr "39"
##   .. .. .. ..@ area     : num 0.613
##   .. ..$ :Formal class 'Polygons' [package "sp"] with 5 slots
##   .. .. .. ..@ Polygons :List of 1
##   .. .. .. .. ..$ :Formal class 'Polygon' [package "sp"] with 5 slots
##   .. .. .. .. .. .. ..@ labpt  : num [1:2] 15.3 49.8
##   .. .. .. .. .. .. ..@ area   : num 10.1
##   .. .. .. .. .. .. ..@ hole   : logi FALSE
##   .. .. .. .. .. .. ..@ ringDir: int 1
##   .. .. .. .. .. .. ..@ coords : num [1:35, 1:2] 17 16.5 16 15.3 14.9 ...
##   .. .. .. .. .. .. .. ..- attr(*, "dimnames")=List of 2
##   .. .. .. .. .. .. .. .. ..$ : NULL
##   .. .. .. .. .. .. .. .. ..$ : chr [1:2] "x" "y"
##   .. .. .. ..@ plotOrder: int 1
##   .. .. .. ..@ labpt    : num [1:2] 15.3 49.8
##   .. .. .. ..@ ID       : chr "40"
##   .. .. .. ..@ area     : num 10.1
##   .. ..$ :Formal class 'Polygons' [package "sp"] with 5 slots
##   .. .. .. ..@ Polygons :List of 1
##   .. .. .. .. ..$ :Formal class 'Polygon' [package "sp"] with 5 slots
##   .. .. .. .. .. .. ..@ labpt  : num [1:2] 10.3 51.1
##   .. .. .. .. .. .. ..@ area   : num 45.9
##   .. .. .. .. .. .. ..@ hole   : logi FALSE
##   .. .. .. .. .. .. ..@ ringDir: int 1
##   .. .. .. .. .. .. ..@ coords : num [1:58, 1:2] 9.92 9.94 10.95 10.94 11.96 ...
##   .. .. .. .. .. .. .. ..- attr(*, "dimnames")=List of 2
##   .. .. .. .. .. .. .. .. ..$ : NULL
##   .. .. .. .. .. .. .. .. ..$ : chr [1:2] "x" "y"
##   .. .. .. ..@ plotOrder: int 1
##   .. .. .. ..@ labpt    : num [1:2] 10.3 51.1
##   .. .. .. ..@ ID       : chr "41"
##   .. .. .. ..@ area     : num 45.9
##   .. ..$ :Formal class 'Polygons' [package "sp"] with 5 slots
##   .. .. .. ..@ Polygons :List of 1
##   .. .. .. .. ..$ :Formal class 'Polygon' [package "sp"] with 5 slots
##   .. .. .. .. .. .. ..@ labpt  : num [1:2] 42.5 11.8
##   .. .. .. .. .. .. ..@ area   : num 1.81
##   .. .. .. .. .. .. ..@ hole   : logi FALSE
##   .. .. .. .. .. .. ..@ ringDir: int 1
##   .. .. .. .. .. .. ..@ coords : num [1:15, 1:2] 43.1 43.3 43.3 42.7 43.1 ...
##   .. .. .. .. .. .. .. ..- attr(*, "dimnames")=List of 2
##   .. .. .. .. .. .. .. .. ..$ : NULL
##   .. .. .. .. .. .. .. .. ..$ : chr [1:2] "x" "y"
##   .. .. .. ..@ plotOrder: int 1
##   .. .. .. ..@ labpt    : num [1:2] 42.5 11.8
##   .. .. .. ..@ ID       : chr "42"
##   .. .. .. ..@ area     : num 1.81
##   .. ..$ :Formal class 'Polygons' [package "sp"] with 5 slots
##   .. .. .. ..@ Polygons :List of 2
##   .. .. .. .. ..$ :Formal class 'Polygon' [package "sp"] with 5 slots
##   .. .. .. .. .. .. ..@ labpt  : num [1:2] 11.9 55.5
##   .. .. .. .. .. .. ..@ area   : num 1.37
##   .. .. .. .. .. .. ..@ hole   : logi FALSE
##   .. .. .. .. .. .. ..@ ringDir: int 1
##   .. .. .. .. .. .. ..@ coords : num [1:6, 1:2] 12.7 12.1 11 10.9 12.4 ...
##   .. .. .. .. .. .. .. ..- attr(*, "dimnames")=List of 2
##   .. .. .. .. .. .. .. .. ..$ : NULL
##   .. .. .. .. .. .. .. .. ..$ : chr [1:2] "x" "y"
##   .. .. .. .. ..$ :Formal class 'Polygon' [package "sp"] with 5 slots
##   .. .. .. .. .. .. ..@ labpt  : num [1:2] 9.31 56.22
##   .. .. .. .. .. .. ..@ area   : num 4.8
##   .. .. .. .. .. .. ..@ hole   : logi FALSE
##   .. .. .. .. .. .. ..@ ringDir: int 1
##   .. .. .. .. .. .. ..@ coords : num [1:18, 1:2] 10.91 10.67 10.37 9.65 9.92 ...
##   .. .. .. .. .. .. .. ..- attr(*, "dimnames")=List of 2
##   .. .. .. .. .. .. .. .. ..$ : NULL
##   .. .. .. .. .. .. .. .. ..$ : chr [1:2] "x" "y"
##   .. .. .. ..@ plotOrder: int [1:2] 2 1
##   .. .. .. ..@ labpt    : num [1:2] 9.31 56.22
##   .. .. .. ..@ ID       : chr "43"
##   .. .. .. ..@ area     : num 6.17
##   .. ..$ :Formal class 'Polygons' [package "sp"] with 5 slots
##   .. .. .. ..@ Polygons :List of 1
##   .. .. .. .. ..$ :Formal class 'Polygon' [package "sp"] with 5 slots
##   .. .. .. .. .. .. ..@ labpt  : num [1:2] -70.5 18.9
##   .. .. .. .. .. .. ..@ area   : num 4.13
##   .. .. .. .. .. .. ..@ hole   : logi FALSE
##   .. .. .. .. .. .. ..@ ringDir: int 1
##   .. .. .. .. .. .. ..@ coords : num [1:26, 1:2] -71.7 -71.6 -70.8 -70.2 -70 ...
##   .. .. .. .. .. .. .. ..- attr(*, "dimnames")=List of 2
##   .. .. .. .. .. .. .. .. ..$ : NULL
##   .. .. .. .. .. .. .. .. ..$ : chr [1:2] "x" "y"
##   .. .. .. ..@ plotOrder: int 1
##   .. .. .. ..@ labpt    : num [1:2] -70.5 18.9
##   .. .. .. ..@ ID       : chr "44"
##   .. .. .. ..@ area     : num 4.13
##   .. ..$ :Formal class 'Polygons' [package "sp"] with 5 slots
##   .. .. .. ..@ Polygons :List of 1
##   .. .. .. .. ..$ :Formal class 'Polygon' [package "sp"] with 5 slots
##   .. .. .. .. .. .. ..@ labpt  : num [1:2] 2.6 28.2
##   .. .. .. .. .. .. ..@ area   : num 214
##   .. .. .. .. .. .. ..@ hole   : logi FALSE
##   .. .. .. .. .. .. ..@ ringDir: int 1
##   .. .. .. .. .. .. ..@ coords : num [1:62, 1:2] 12 8.57 5.68 4.27 3.16 ...
##   .. .. .. .. .. .. .. ..- attr(*, "dimnames")=List of 2
##   .. .. .. .. .. .. .. .. ..$ : NULL
##   .. .. .. .. .. .. .. .. ..$ : chr [1:2] "x" "y"
##   .. .. .. ..@ plotOrder: int 1
##   .. .. .. ..@ labpt    : num [1:2] 2.6 28.2
##   .. .. .. ..@ ID       : chr "45"
##   .. .. .. ..@ area     : num 214
##   .. ..$ :Formal class 'Polygons' [package "sp"] with 5 slots
##   .. .. .. ..@ Polygons :List of 1
##   .. .. .. .. ..$ :Formal class 'Polygon' [package "sp"] with 5 slots
##   .. .. .. .. .. .. ..@ labpt  : num [1:2] -78.38 -1.45
##   .. .. .. .. .. .. ..@ area   : num 20.4
##   .. .. .. .. .. .. ..@ hole   : logi FALSE
##   .. .. .. .. .. .. ..@ ringDir: int 1
##   .. .. .. .. .. .. ..@ coords : num [1:33, 1:2] -80.3 -79.8 -80 -80.4 -81 ...
##   .. .. .. .. .. .. .. ..- attr(*, "dimnames")=List of 2
##   .. .. .. .. .. .. .. .. ..$ : NULL
##   .. .. .. .. .. .. .. .. ..$ : chr [1:2] "x" "y"
##   .. .. .. ..@ plotOrder: int 1
##   .. .. .. ..@ labpt    : num [1:2] -78.38 -1.45
##   .. .. .. ..@ ID       : chr "46"
##   .. .. .. ..@ area     : num 20.4
##   .. ..$ :Formal class 'Polygons' [package "sp"] with 5 slots
##   .. .. .. ..@ Polygons :List of 1
##   .. .. .. .. ..$ :Formal class 'Polygon' [package "sp"] with 5 slots
##   .. .. .. .. .. .. ..@ labpt  : num [1:2] 29.8 26.5
##   .. .. .. .. .. .. ..@ area   : num 90.4
##   .. .. .. .. .. .. ..@ hole   : logi FALSE
##   .. .. .. .. .. .. ..@ ringDir: int 1
##   .. .. .. .. .. .. ..@ coords : num [1:42, 1:2] 34.9 34.6 34.4 34.2 33.9 ...
##   .. .. .. .. .. .. .. ..- attr(*, "dimnames")=List of 2
##   .. .. .. .. .. .. .. .. ..$ : NULL
##   .. .. .. .. .. .. .. .. ..$ : chr [1:2] "x" "y"
##   .. .. .. ..@ plotOrder: int 1
##   .. .. .. ..@ labpt    : num [1:2] 29.8 26.5
##   .. .. .. ..@ ID       : chr "47"
##   .. .. .. ..@ area     : num 90.4
##   .. ..$ :Formal class 'Polygons' [package "sp"] with 5 slots
##   .. .. .. ..@ Polygons :List of 1
##   .. .. .. .. ..$ :Formal class 'Polygon' [package "sp"] with 5 slots
##   .. .. .. .. .. .. ..@ labpt  : num [1:2] 38.7 15.4
##   .. .. .. .. .. .. ..@ area   : num 10
##   .. .. .. .. .. .. ..@ hole   : logi FALSE
##   .. .. .. .. .. .. ..@ ringDir: int 1
##   .. .. .. .. .. .. ..@ coords : num [1:28, 1:2] 42.4 42 41.6 41.2 40.9 ...
##   .. .. .. .. .. .. .. ..- attr(*, "dimnames")=List of 2
##   .. .. .. .. .. .. .. .. ..$ : NULL
##   .. .. .. .. .. .. .. .. ..$ : chr [1:2] "x" "y"
##   .. .. .. ..@ plotOrder: int 1
##   .. .. .. ..@ labpt    : num [1:2] 38.7 15.4
##   .. .. .. ..@ ID       : chr "48"
##   .. .. .. ..@ area     : num 10
##   .. ..$ :Formal class 'Polygons' [package "sp"] with 5 slots
##   .. .. .. ..@ Polygons :List of 1
##   .. .. .. .. ..$ :Formal class 'Polygon' [package "sp"] with 5 slots
##   .. .. .. .. .. .. ..@ labpt  : num [1:2] -3.62 40.35
##   .. .. .. .. .. .. ..@ area   : num 53.3
##   .. .. .. .. .. .. ..@ hole   : logi FALSE
##   .. .. .. .. .. .. ..@ ringDir: int 1
##   .. .. .. .. .. .. ..@ coords : num [1:51, 1:2] -9.03 -8.98 -9.39 -7.98 -6.75 ...
##   .. .. .. .. .. .. .. ..- attr(*, "dimnames")=List of 2
##   .. .. .. .. .. .. .. .. ..$ : NULL
##   .. .. .. .. .. .. .. .. ..$ : chr [1:2] "x" "y"
##   .. .. .. ..@ plotOrder: int 1
##   .. .. .. ..@ labpt    : num [1:2] -3.62 40.35
##   .. .. .. ..@ ID       : chr "49"
##   .. .. .. ..@ area     : num 53.3
##   .. ..$ :Formal class 'Polygons' [package "sp"] with 5 slots
##   .. .. .. ..@ Polygons :List of 1
##   .. .. .. .. ..$ :Formal class 'Polygon' [package "sp"] with 5 slots
##   .. .. .. .. .. .. ..@ labpt  : num [1:2] 25.8 58.6
##   .. .. .. .. .. .. ..@ area   : num 6.91
##   .. .. .. .. .. .. ..@ hole   : logi FALSE
##   .. .. .. .. .. .. ..@ ringDir: int 1
##   .. .. .. .. .. .. ..@ coords : num [1:17, 1:2] 24.3 24.4 24.1 23.4 23.3 ...
##   .. .. .. .. .. .. .. ..- attr(*, "dimnames")=List of 2
##   .. .. .. .. .. .. .. .. ..$ : NULL
##   .. .. .. .. .. .. .. .. ..$ : chr [1:2] "x" "y"
##   .. .. .. ..@ plotOrder: int 1
##   .. .. .. ..@ labpt    : num [1:2] 25.8 58.6
##   .. .. .. ..@ ID       : chr "50"
##   .. .. .. ..@ area     : num 6.91
##   .. ..$ :Formal class 'Polygons' [package "sp"] with 5 slots
##   .. .. .. ..@ Polygons :List of 1
##   .. .. .. .. ..$ :Formal class 'Polygon' [package "sp"] with 5 slots
##   .. .. .. .. .. .. ..@ labpt  : num [1:2] 39.55 8.65
##   .. .. .. .. .. .. ..@ area   : num 93.1
##   .. .. .. .. .. .. ..@ hole   : logi FALSE
##   .. .. .. .. .. .. ..@ ringDir: int 1
##   .. .. .. .. .. .. ..@ coords : num [1:59, 1:2] 37.9 38.5 39.1 39.3 40 ...
##   .. .. .. .. .. .. .. ..- attr(*, "dimnames")=List of 2
##   .. .. .. .. .. .. .. .. ..$ : NULL
##   .. .. .. .. .. .. .. .. ..$ : chr [1:2] "x" "y"
##   .. .. .. ..@ plotOrder: int 1
##   .. .. .. ..@ labpt    : num [1:2] 39.55 8.65
##   .. .. .. ..@ ID       : chr "51"
##   .. .. .. ..@ area     : num 93.1
##   .. ..$ :Formal class 'Polygons' [package "sp"] with 5 slots
##   .. .. .. ..@ Polygons :List of 1
##   .. .. .. .. ..$ :Formal class 'Polygon' [package "sp"] with 5 slots
##   .. .. .. .. .. .. ..@ labpt  : num [1:2] 26.2 64.5
##   .. .. .. .. .. .. ..@ area   : num 63.8
##   .. .. .. .. .. .. ..@ hole   : logi FALSE
##   .. .. .. .. .. .. ..@ ringDir: int 1
##   .. .. .. .. .. .. ..@ coords : num [1:38, 1:2] 28.6 28.4 30 29.1 30.2 ...
##   .. .. .. .. .. .. .. ..- attr(*, "dimnames")=List of 2
##   .. .. .. .. .. .. .. .. ..$ : NULL
##   .. .. .. .. .. .. .. .. ..$ : chr [1:2] "x" "y"
##   .. .. .. ..@ plotOrder: int 1
##   .. .. .. ..@ labpt    : num [1:2] 26.2 64.5
##   .. .. .. ..@ ID       : chr "52"
##   .. .. .. ..@ area     : num 63.8
##   .. ..$ :Formal class 'Polygons' [package "sp"] with 5 slots
##   .. .. .. ..@ Polygons :List of 3
##   .. .. .. .. ..$ :Formal class 'Polygon' [package "sp"] with 5 slots
##   .. .. .. .. .. .. ..@ labpt  : num [1:2] 178 -17.8
##   .. .. .. .. .. .. ..@ area   : num 0.984
##   .. .. .. .. .. .. ..@ hole   : logi FALSE
##   .. .. .. .. .. .. ..@ ringDir: int 1
##   .. .. .. .. .. .. ..@ coords : num [1:9, 1:2] 178 179 179 178 177 ...
##   .. .. .. .. .. .. .. ..- attr(*, "dimnames")=List of 2
##   .. .. .. .. .. .. .. .. ..$ : NULL
##   .. .. .. .. .. .. .. .. ..$ : chr [1:2] "x" "y"
##   .. .. .. .. ..$ :Formal class 'Polygon' [package "sp"] with 5 slots
##   .. .. .. .. .. .. ..@ labpt  : num [1:2] 179.3 -16.6
##   .. .. .. .. .. .. ..@ area   : num 0.589
##   .. .. .. .. .. .. ..@ hole   : logi FALSE
##   .. .. .. .. .. .. ..@ ringDir: int 1
##   .. .. .. .. .. .. ..@ coords : num [1:8, 1:2] 179 179 179 179 179 ...
##   .. .. .. .. .. .. .. ..- attr(*, "dimnames")=List of 2
##   .. .. .. .. .. .. .. .. ..$ : NULL
##   .. .. .. .. .. .. .. .. ..$ : chr [1:2] "x" "y"
##   .. .. .. .. ..$ :Formal class 'Polygon' [package "sp"] with 5 slots
##   .. .. .. .. .. .. ..@ labpt  : num [1:2] -179.9 -16.3
##   .. .. .. .. .. .. ..@ area   : num 0.067
##   .. .. .. .. .. .. ..@ hole   : logi FALSE
##   .. .. .. .. .. .. ..@ ringDir: int 1
##   .. .. .. .. .. .. ..@ coords : num [1:5, 1:2] -180 -180 -180 -180 -180 ...
##   .. .. .. .. .. .. .. ..- attr(*, "dimnames")=List of 2
##   .. .. .. .. .. .. .. .. ..$ : NULL
##   .. .. .. .. .. .. .. .. ..$ : chr [1:2] "x" "y"
##   .. .. .. ..@ plotOrder: int [1:3] 1 2 3
##   .. .. .. ..@ labpt    : num [1:2] 178 -17.8
##   .. .. .. ..@ ID       : chr "53"
##   .. .. .. ..@ area     : num 1.64
##   .. ..$ :Formal class 'Polygons' [package "sp"] with 5 slots
##   .. .. .. ..@ Polygons :List of 1
##   .. .. .. .. ..$ :Formal class 'Polygon' [package "sp"] with 5 slots
##   .. .. .. .. .. .. ..@ labpt  : num [1:2] -59.4 -51.7
##   .. .. .. .. .. .. ..@ area   : num 2.13
##   .. .. .. .. .. .. ..@ hole   : logi FALSE
##   .. .. .. .. .. .. ..@ ringDir: int 1
##   .. .. .. .. .. .. ..@ coords : num [1:10, 1:2] -61.2 -60 -59.1 -58.5 -57.8 ...
##   .. .. .. .. .. .. .. ..- attr(*, "dimnames")=List of 2
##   .. .. .. .. .. .. .. .. ..$ : NULL
##   .. .. .. .. .. .. .. .. ..$ : chr [1:2] "x" "y"
##   .. .. .. ..@ plotOrder: int 1
##   .. .. .. ..@ labpt    : num [1:2] -59.4 -51.7
##   .. .. .. ..@ ID       : chr "54"
##   .. .. .. ..@ area     : num 2.13
##   .. ..$ :Formal class 'Polygons' [package "sp"] with 5 slots
##   .. .. .. ..@ Polygons :List of 3
##   .. .. .. .. ..$ :Formal class 'Polygon' [package "sp"] with 5 slots
##   .. .. .. .. .. .. ..@ labpt  : num [1:2] -53.24 3.91
##   .. .. .. .. .. .. ..@ area   : num 6.95
##   .. .. .. .. .. .. ..@ hole   : logi FALSE
##   .. .. .. .. .. .. ..@ ringDir: int 1
##   .. .. .. .. .. .. ..@ coords : num [1:19, 1:2] -52.6 -52.9 -53.4 -53.6 -53.8 ...
##   .. .. .. .. .. .. .. ..- attr(*, "dimnames")=List of 2
##   .. .. .. .. .. .. .. .. ..$ : NULL
##   .. .. .. .. .. .. .. .. ..$ : chr [1:2] "x" "y"
##   .. .. .. .. ..$ :Formal class 'Polygon' [package "sp"] with 5 slots
##   .. .. .. .. .. .. ..@ labpt  : num [1:2] 9.08 42.18
##   .. .. .. .. .. .. ..@ area   : num 1.05
##   .. .. .. .. .. .. ..@ hole   : logi FALSE
##   .. .. .. .. .. .. ..@ ringDir: int 1
##   .. .. .. .. .. .. ..@ coords : num [1:7, 1:2] 9.56 9.23 8.78 8.54 8.75 ...
##   .. .. .. .. .. .. .. ..- attr(*, "dimnames")=List of 2
##   .. .. .. .. .. .. .. .. ..$ : NULL
##   .. .. .. .. .. .. .. .. ..$ : chr [1:2] "x" "y"
##   .. .. .. .. ..$ :Formal class 'Polygon' [package "sp"] with 5 slots
##   .. .. .. .. .. .. ..@ labpt  : num [1:2] 2.34 46.61
##   .. .. .. .. .. .. ..@ area   : num 64.6
##   .. .. .. .. .. .. ..@ hole   : logi FALSE
##   .. .. .. .. .. .. ..@ ringDir: int 1
##   .. .. .. .. .. .. ..@ coords : num [1:48, 1:2] 3.59 4.29 4.8 5.67 5.9 ...
##   .. .. .. .. .. .. .. ..- attr(*, "dimnames")=List of 2
##   .. .. .. .. .. .. .. .. ..$ : NULL
##   .. .. .. .. .. .. .. .. ..$ : chr [1:2] "x" "y"
##   .. .. .. ..@ plotOrder: int [1:3] 3 1 2
##   .. .. .. ..@ labpt    : num [1:2] 2.34 46.61
##   .. .. .. ..@ ID       : chr "55"
##   .. .. .. ..@ area     : num 72.6
##   .. ..$ :Formal class 'Polygons' [package "sp"] with 5 slots
##   .. .. .. ..@ Polygons :List of 1
##   .. .. .. .. ..$ :Formal class 'Polygon' [package "sp"] with 5 slots
##   .. .. .. .. .. .. ..@ labpt  : num [1:2] 11.688 -0.647
##   .. .. .. .. .. .. ..@ area   : num 21.9
##   .. .. .. .. .. .. ..@ hole   : logi FALSE
##   .. .. .. .. .. .. ..@ ringDir: int 1
##   .. .. .. .. .. .. ..@ coords : num [1:31, 1:2] 11.09 10.07 9.41 8.8 8.83 ...
##   .. .. .. .. .. .. .. ..- attr(*, "dimnames")=List of 2
##   .. .. .. .. .. .. .. .. ..$ : NULL
##   .. .. .. .. .. .. .. .. ..$ : chr [1:2] "x" "y"
##   .. .. .. ..@ plotOrder: int 1
##   .. .. .. ..@ labpt    : num [1:2] 11.688 -0.647
##   .. .. .. ..@ ID       : chr "56"
##   .. .. .. ..@ area     : num 21.9
##   .. ..$ :Formal class 'Polygons' [package "sp"] with 5 slots
##   .. .. .. ..@ Polygons :List of 2
##   .. .. .. .. ..$ :Formal class 'Polygon' [package "sp"] with 5 slots
##   .. .. .. .. .. .. ..@ labpt  : num [1:2] -6.71 54.53
##   .. .. .. .. .. .. ..@ area   : num 1.65
##   .. .. .. .. .. .. ..@ hole   : logi FALSE
##   .. .. .. .. .. .. ..@ ringDir: int 1
##   .. .. .. .. .. .. ..@ coords : num [1:8, 1:2] -5.66 -6.2 -6.95 -7.57 -7.37 ...
##   .. .. .. .. .. .. .. ..- attr(*, "dimnames")=List of 2
##   .. .. .. .. .. .. .. .. ..$ : NULL
##   .. .. .. .. .. .. .. .. ..$ : chr [1:2] "x" "y"
##   .. .. .. .. ..$ :Formal class 'Polygon' [package "sp"] with 5 slots
##   .. .. .. .. .. .. ..@ labpt  : num [1:2] -2.66 53.88
##   .. .. .. .. .. .. ..@ area   : num 32.6
##   .. .. .. .. .. .. ..@ hole   : logi FALSE
##   .. .. .. .. .. .. ..@ ringDir: int 1
##   .. .. .. .. .. .. ..@ coords : num [1:48, 1:2] -3.01 -4.07 -3.06 -1.96 -2.22 ...
##   .. .. .. .. .. .. .. ..- attr(*, "dimnames")=List of 2
##   .. .. .. .. .. .. .. .. ..$ : NULL
##   .. .. .. .. .. .. .. .. ..$ : chr [1:2] "x" "y"
##   .. .. .. ..@ plotOrder: int [1:2] 2 1
##   .. .. .. ..@ labpt    : num [1:2] -2.66 53.88
##   .. .. .. ..@ ID       : chr "57"
##   .. .. .. ..@ area     : num 34.2
##   .. ..$ :Formal class 'Polygons' [package "sp"] with 5 slots
##   .. .. .. ..@ Polygons :List of 1
##   .. .. .. .. ..$ :Formal class 'Polygon' [package "sp"] with 5 slots
##   .. .. .. .. .. .. ..@ labpt  : num [1:2] 43.5 42.2
##   .. .. .. .. .. .. ..@ area   : num 7.52
##   .. .. .. .. .. .. ..@ hole   : logi FALSE
##   .. .. .. .. .. .. ..@ ringDir: int 1
##   .. .. .. .. .. .. ..@ coords : num [1:24, 1:2] 41.6 41.7 41.5 40.9 40.3 ...
##   .. .. .. .. .. .. .. ..- attr(*, "dimnames")=List of 2
##   .. .. .. .. .. .. .. .. ..$ : NULL
##   .. .. .. .. .. .. .. .. ..$ : chr [1:2] "x" "y"
##   .. .. .. ..@ plotOrder: int 1
##   .. .. .. ..@ labpt    : num [1:2] 43.5 42.2
##   .. .. .. ..@ ID       : chr "58"
##   .. .. .. ..@ area     : num 7.52
##   .. ..$ :Formal class 'Polygons' [package "sp"] with 5 slots
##   .. .. .. ..@ Polygons :List of 1
##   .. .. .. .. ..$ :Formal class 'Polygon' [package "sp"] with 5 slots
##   .. .. .. .. .. .. ..@ labpt  : num [1:2] -1.24 7.93
##   .. .. .. .. .. .. ..@ area   : num 20
##   .. .. .. .. .. .. ..@ hole   : logi FALSE
##   .. .. .. .. .. .. ..@ ringDir: int 1
##   .. .. .. .. .. .. ..@ coords : num [1:25, 1:2] 1.06 -0.508 -1.064 -1.965 -2.856 ...
##   .. .. .. .. .. .. .. ..- attr(*, "dimnames")=List of 2
##   .. .. .. .. .. .. .. .. ..$ : NULL
##   .. .. .. .. .. .. .. .. ..$ : chr [1:2] "x" "y"
##   .. .. .. ..@ plotOrder: int 1
##   .. .. .. ..@ labpt    : num [1:2] -1.24 7.93
##   .. .. .. ..@ ID       : chr "59"
##   .. .. .. ..@ area     : num 20
##   .. ..$ :Formal class 'Polygons' [package "sp"] with 5 slots
##   .. .. .. ..@ Polygons :List of 1
##   .. .. .. .. ..$ :Formal class 'Polygon' [package "sp"] with 5 slots
##   .. .. .. .. .. .. ..@ labpt  : num [1:2] -11.1 10.4
##   .. .. .. .. .. .. ..@ area   : num 19.8
##   .. .. .. .. .. .. ..@ hole   : logi FALSE
##   .. .. .. .. .. .. ..@ ringDir: int 1
##   .. .. .. .. .. .. ..@ coords : num [1:70, 1:2] -8.44 -8.72 -8.93 -9.21 -9.4 ...
##   .. .. .. .. .. .. .. ..- attr(*, "dimnames")=List of 2
##   .. .. .. .. .. .. .. .. ..$ : NULL
##   .. .. .. .. .. .. .. .. ..$ : chr [1:2] "x" "y"
##   .. .. .. ..@ plotOrder: int 1
##   .. .. .. ..@ labpt    : num [1:2] -11.1 10.4
##   .. .. .. ..@ ID       : chr "60"
##   .. .. .. ..@ area     : num 19.8
##   .. ..$ :Formal class 'Polygons' [package "sp"] with 5 slots
##   .. .. .. ..@ Polygons :List of 1
##   .. .. .. .. ..$ :Formal class 'Polygon' [package "sp"] with 5 slots
##   .. .. .. .. .. .. ..@ labpt  : num [1:2] -15.4 13.5
##   .. .. .. .. .. .. ..@ area   : num 1.17
##   .. .. .. .. .. .. ..@ hole   : logi FALSE
##   .. .. .. .. .. .. ..@ ringDir: int 1
##   .. .. .. .. .. .. ..@ coords : num [1:16, 1:2] -16.8 -16.7 -15.6 -15.4 -15.1 ...
##   .. .. .. .. .. .. .. ..- attr(*, "dimnames")=List of 2
##   .. .. .. .. .. .. .. .. ..$ : NULL
##   .. .. .. .. .. .. .. .. ..$ : chr [1:2] "x" "y"
##   .. .. .. ..@ plotOrder: int 1
##   .. .. .. ..@ labpt    : num [1:2] -15.4 13.5
##   .. .. .. ..@ ID       : chr "61"
##   .. .. .. ..@ area     : num 1.17
##   .. ..$ :Formal class 'Polygons' [package "sp"] with 5 slots
##   .. .. .. ..@ Polygons :List of 1
##   .. .. .. .. ..$ :Formal class 'Polygon' [package "sp"] with 5 slots
##   .. .. .. .. .. .. ..@ labpt  : num [1:2] -15.1 12
##   .. .. .. .. .. .. ..@ area   : num 3
##   .. .. .. .. .. .. ..@ hole   : logi FALSE
##   .. .. .. .. .. .. ..@ ringDir: int 1
##   .. .. .. .. .. .. ..@ coords : num [1:19, 1:2] -15.1 -15.7 -16.1 -16.3 -16.3 ...
##   .. .. .. .. .. .. .. ..- attr(*, "dimnames")=List of 2
##   .. .. .. .. .. .. .. .. ..$ : NULL
##   .. .. .. .. .. .. .. .. ..$ : chr [1:2] "x" "y"
##   .. .. .. ..@ plotOrder: int 1
##   .. .. .. ..@ labpt    : num [1:2] -15.1 12
##   .. .. .. ..@ ID       : chr "62"
##   .. .. .. ..@ area     : num 3
##   .. ..$ :Formal class 'Polygons' [package "sp"] with 5 slots
##   .. .. .. ..@ Polygons :List of 1
##   .. .. .. .. ..$ :Formal class 'Polygon' [package "sp"] with 5 slots
##   .. .. .. .. .. .. ..@ labpt  : num [1:2] 10.37 1.65
##   .. .. .. .. .. .. ..@ area   : num 2.2
##   .. .. .. .. .. .. ..@ hole   : logi FALSE
##   .. .. .. .. .. .. ..@ ringDir: int 1
##   .. .. .. .. .. .. ..@ coords : num [1:7, 1:2] 9.49 9.31 9.65 11.28 11.29 ...
##   .. .. .. .. .. .. .. ..- attr(*, "dimnames")=List of 2
##   .. .. .. .. .. .. .. .. ..$ : NULL
##   .. .. .. .. .. .. .. .. ..$ : chr [1:2] "x" "y"
##   .. .. .. ..@ plotOrder: int 1
##   .. .. .. ..@ labpt    : num [1:2] 10.37 1.65
##   .. .. .. ..@ ID       : chr "63"
##   .. .. .. ..@ area     : num 2.2
##   .. ..$ :Formal class 'Polygons' [package "sp"] with 5 slots
##   .. .. .. ..@ Polygons :List of 2
##   .. .. .. .. ..$ :Formal class 'Polygon' [package "sp"] with 5 slots
##   .. .. .. .. .. .. ..@ labpt  : num [1:2] 24.9 35.2
##   .. .. .. .. .. .. ..@ area   : num 0.92
##   .. .. .. .. .. .. ..@ hole   : logi FALSE
##   .. .. .. .. .. .. ..@ ringDir: int 1
##   .. .. .. .. .. .. ..@ coords : num [1:11, 1:2] 23.7 24.2 25 25.8 25.7 ...
##   .. .. .. .. .. .. .. ..- attr(*, "dimnames")=List of 2
##   .. .. .. .. .. .. .. .. ..$ : NULL
##   .. .. .. .. .. .. .. .. ..$ : chr [1:2] "x" "y"
##   .. .. .. .. ..$ :Formal class 'Polygon' [package "sp"] with 5 slots
##   .. .. .. .. .. .. ..@ labpt  : num [1:2] 22.6 39.3
##   .. .. .. .. .. .. ..@ area   : num 12.8
##   .. .. .. .. .. .. ..@ hole   : logi FALSE
##   .. .. .. .. .. .. ..@ ringDir: int 1
##   .. .. .. .. .. .. ..@ coords : num [1:43, 1:2] 26.6 26.3 26.1 25.4 24.9 ...
##   .. .. .. .. .. .. .. ..- attr(*, "dimnames")=List of 2
##   .. .. .. .. .. .. .. .. ..$ : NULL
##   .. .. .. .. .. .. .. .. ..$ : chr [1:2] "x" "y"
##   .. .. .. ..@ plotOrder: int [1:2] 2 1
##   .. .. .. ..@ labpt    : num [1:2] 22.6 39.3
##   .. .. .. ..@ ID       : chr "64"
##   .. .. .. ..@ area     : num 13.7
##   .. ..$ :Formal class 'Polygons' [package "sp"] with 5 slots
##   .. .. .. ..@ Polygons :List of 1
##   .. .. .. .. ..$ :Formal class 'Polygon' [package "sp"] with 5 slots
##   .. .. .. .. .. .. ..@ labpt  : num [1:2] -41.5 74.8
##   .. .. .. .. .. .. ..@ area   : num 678
##   .. .. .. .. .. .. ..@ hole   : logi FALSE
##   .. .. .. .. .. .. ..@ ringDir: int 1
##   .. .. .. .. .. .. ..@ coords : num [1:132, 1:2] -46.8 -43.4 -39.9 -38.6 -35.1 ...
##   .. .. .. .. .. .. .. ..- attr(*, "dimnames")=List of 2
##   .. .. .. .. .. .. .. .. ..$ : NULL
##   .. .. .. .. .. .. .. .. ..$ : chr [1:2] "x" "y"
##   .. .. .. ..@ plotOrder: int 1
##   .. .. .. ..@ labpt    : num [1:2] -41.5 74.8
##   .. .. .. ..@ ID       : chr "65"
##   .. .. .. ..@ area     : num 678
##   .. ..$ :Formal class 'Polygons' [package "sp"] with 5 slots
##   .. .. .. ..@ Polygons :List of 1
##   .. .. .. .. ..$ :Formal class 'Polygon' [package "sp"] with 5 slots
##   .. .. .. .. .. .. ..@ labpt  : num [1:2] -90.4 15.7
##   .. .. .. .. .. .. ..@ area   : num 9.23
##   .. .. .. .. .. .. ..@ hole   : logi FALSE
##   .. .. .. .. .. .. ..@ ringDir: int 1
##   .. .. .. .. .. .. ..@ coords : num [1:35, 1:2] -90.1 -90.6 -91.2 -91.7 -92.2 ...
##   .. .. .. .. .. .. .. ..- attr(*, "dimnames")=List of 2
##   .. .. .. .. .. .. .. .. ..$ : NULL
##   .. .. .. .. .. .. .. .. ..$ : chr [1:2] "x" "y"
##   .. .. .. ..@ plotOrder: int 1
##   .. .. .. ..@ labpt    : num [1:2] -90.4 15.7
##   .. .. .. ..@ ID       : chr "66"
##   .. .. .. ..@ area     : num 9.23
##   .. ..$ :Formal class 'Polygons' [package "sp"] with 5 slots
##   .. .. .. ..@ Polygons :List of 1
##   .. .. .. .. ..$ :Formal class 'Polygon' [package "sp"] with 5 slots
##   .. .. .. .. .. .. ..@ labpt  : num [1:2] -58.97 4.79
##   .. .. .. .. .. .. ..@ area   : num 17.1
##   .. .. .. .. .. .. ..@ hole   : logi FALSE
##   .. .. .. .. .. .. ..@ ringDir: int 1
##   .. .. .. .. .. .. ..@ coords : num [1:40, 1:2] -59.8 -59.1 -58.5 -58.5 -58.1 ...
##   .. .. .. .. .. .. .. ..- attr(*, "dimnames")=List of 2
##   .. .. .. .. .. .. .. .. ..$ : NULL
##   .. .. .. .. .. .. .. .. ..$ : chr [1:2] "x" "y"
##   .. .. .. ..@ plotOrder: int 1
##   .. .. .. ..@ labpt    : num [1:2] -58.97 4.79
##   .. .. .. ..@ ID       : chr "67"
##   .. .. .. ..@ area     : num 17.1
##   .. ..$ :Formal class 'Polygons' [package "sp"] with 5 slots
##   .. .. .. ..@ Polygons :List of 1
##   .. .. .. .. ..$ :Formal class 'Polygon' [package "sp"] with 5 slots
##   .. .. .. .. .. .. ..@ labpt  : num [1:2] -86.6 14.8
##   .. .. .. .. .. .. ..@ area   : num 9.55
##   .. .. .. .. .. .. ..@ hole   : logi FALSE
##   .. .. .. .. .. .. ..@ ringDir: int 1
##   .. .. .. .. .. .. ..@ coords : num [1:57, 1:2] -87.3 -87.5 -87.8 -87.7 -87.9 ...
##   .. .. .. .. .. .. .. ..- attr(*, "dimnames")=List of 2
##   .. .. .. .. .. .. .. .. ..$ : NULL
##   .. .. .. .. .. .. .. .. ..$ : chr [1:2] "x" "y"
##   .. .. .. ..@ plotOrder: int 1
##   .. .. .. ..@ labpt    : num [1:2] -86.6 14.8
##   .. .. .. ..@ ID       : chr "68"
##   .. .. .. ..@ area     : num 9.55
##   .. ..$ :Formal class 'Polygons' [package "sp"] with 5 slots
##   .. .. .. ..@ Polygons :List of 1
##   .. .. .. .. ..$ :Formal class 'Polygon' [package "sp"] with 5 slots
##   .. .. .. .. .. .. ..@ labpt  : num [1:2] 16.6 45
##   .. .. .. .. .. .. ..@ area   : num 6.57
##   .. .. .. .. .. .. ..@ hole   : logi FALSE
##   .. .. .. .. .. .. ..@ ringDir: int 1
##   .. .. .. .. .. .. ..@ coords : num [1:42, 1:2] 18.8 19.1 19.4 19 18.6 ...
##   .. .. .. .. .. .. .. ..- attr(*, "dimnames")=List of 2
##   .. .. .. .. .. .. .. .. ..$ : NULL
##   .. .. .. .. .. .. .. .. ..$ : chr [1:2] "x" "y"
##   .. .. .. ..@ plotOrder: int 1
##   .. .. .. ..@ labpt    : num [1:2] 16.6 45
##   .. .. .. ..@ ID       : chr "69"
##   .. .. .. ..@ area     : num 6.57
##   .. ..$ :Formal class 'Polygons' [package "sp"] with 5 slots
##   .. .. .. ..@ Polygons :List of 1
##   .. .. .. .. ..$ :Formal class 'Polygon' [package "sp"] with 5 slots
##   .. .. .. .. .. .. ..@ labpt  : num [1:2] -72.7 18.9
##   .. .. .. .. .. .. ..@ area   : num 2.45
##   .. .. .. .. .. .. ..@ hole   : logi FALSE
##   .. .. .. .. .. .. ..@ ringDir: int 1
##   .. .. .. .. .. .. ..@ coords : num [1:21, 1:2] -73.2 -72.6 -71.7 -71.6 -71.7 ...
##   .. .. .. .. .. .. .. ..- attr(*, "dimnames")=List of 2
##   .. .. .. .. .. .. .. .. ..$ : NULL
##   .. .. .. .. .. .. .. .. ..$ : chr [1:2] "x" "y"
##   .. .. .. ..@ plotOrder: int 1
##   .. .. .. ..@ labpt    : num [1:2] -72.7 18.9
##   .. .. .. ..@ ID       : chr "70"
##   .. .. .. ..@ area     : num 2.45
##   .. ..$ :Formal class 'Polygons' [package "sp"] with 5 slots
##   .. .. .. ..@ Polygons :List of 1
##   .. .. .. .. ..$ :Formal class 'Polygon' [package "sp"] with 5 slots
##   .. .. .. .. .. .. ..@ labpt  : num [1:2] 19.4 47.2
##   .. .. .. .. .. .. ..@ area   : num 11
##   .. .. .. .. .. .. ..@ hole   : logi FALSE
##   .. .. .. .. .. .. ..@ ringDir: int 1
##   .. .. .. .. .. .. ..@ coords : num [1:31, 1:2] 16.2 16.5 16.3 16.9 17 ...
##   .. .. .. .. .. .. .. ..- attr(*, "dimnames")=List of 2
##   .. .. .. .. .. .. .. .. ..$ : NULL
##   .. .. .. .. .. .. .. .. ..$ : chr [1:2] "x" "y"
##   .. .. .. ..@ plotOrder: int 1
##   .. .. .. ..@ labpt    : num [1:2] 19.4 47.2
##   .. .. .. ..@ ID       : chr "71"
##   .. .. .. ..@ area     : num 11
##   .. ..$ :Formal class 'Polygons' [package "sp"] with 5 slots
##   .. .. .. ..@ Polygons :List of 13
##   .. .. .. .. ..$ :Formal class 'Polygon' [package "sp"] with 5 slots
##   .. .. .. .. .. .. ..@ labpt  : num [1:2] 120 -9.8
##   .. .. .. .. .. .. ..@ area   : num 0.813
##   .. .. .. .. .. .. ..@ hole   : logi FALSE
##   .. .. .. .. .. .. ..@ ringDir: int 1
##   .. .. .. .. .. .. ..@ coords : num [1:7, 1:2] 121 120 119 120 120 ...
##   .. .. .. .. .. .. .. ..- attr(*, "dimnames")=List of 2
##   .. .. .. .. .. .. .. .. ..$ : NULL
##   .. .. .. .. .. .. .. .. ..$ : chr [1:2] "x" "y"
##   .. .. .. .. ..$ :Formal class 'Polygon' [package "sp"] with 5 slots
##   .. .. .. .. .. .. ..@ labpt  : num [1:2] 124.3 -9.65
##   .. .. .. .. .. .. ..@ area   : num 1.28
##   .. .. .. .. .. .. ..@ hole   : logi FALSE
##   .. .. .. .. .. .. ..@ ringDir: int 1
##   .. .. .. .. .. .. ..@ coords : num [1:9, 1:2] 124 124 123 124 124 ...
##   .. .. .. .. .. .. .. ..- attr(*, "dimnames")=List of 2
##   .. .. .. .. .. .. .. .. ..$ : NULL
##   .. .. .. .. .. .. .. .. ..$ : chr [1:2] "x" "y"
##   .. .. .. .. ..$ :Formal class 'Polygon' [package "sp"] with 5 slots
##   .. .. .. .. .. .. ..@ labpt  : num [1:2] 117.91 -8.64
##   .. .. .. .. .. .. ..@ area   : num 1.18
##   .. .. .. .. .. .. ..@ hole   : logi FALSE
##   .. .. .. .. .. .. ..@ ringDir: int 1
##   .. .. .. .. .. .. ..@ coords : num [1:10, 1:2] 118 118 119 119 118 ...
##   .. .. .. .. .. .. .. ..- attr(*, "dimnames")=List of 2
##   .. .. .. .. .. .. .. .. ..$ : NULL
##   .. .. .. .. .. .. .. .. ..$ : chr [1:2] "x" "y"
##   .. .. .. .. ..$ :Formal class 'Polygon' [package "sp"] with 5 slots
##   .. .. .. .. .. .. ..@ labpt  : num [1:2] 121.31 -8.59
##   .. .. .. .. .. .. ..@ area   : num 1.31
##   .. .. .. .. .. .. ..@ hole   : logi FALSE
##   .. .. .. .. .. .. ..@ ringDir: int 1
##   .. .. .. .. .. .. ..@ coords : num [1:9, 1:2] 123 123 121 120 120 ...
##   .. .. .. .. .. .. .. ..- attr(*, "dimnames")=List of 2
##   .. .. .. .. .. .. .. .. ..$ : NULL
##   .. .. .. .. .. .. .. .. ..$ : chr [1:2] "x" "y"
##   .. .. .. .. ..$ :Formal class 'Polygon' [package "sp"] with 5 slots
##   .. .. .. .. .. .. ..@ labpt  : num [1:2] 110.18 -7.34
##   .. .. .. .. .. .. ..@ area   : num 11.3
##   .. .. .. .. .. .. ..@ hole   : logi FALSE
##   .. .. .. .. .. .. ..@ ringDir: int 1
##   .. .. .. .. .. .. ..@ coords : num [1:23, 1:2] 109 111 111 113 113 ...
##   .. .. .. .. .. .. .. ..- attr(*, "dimnames")=List of 2
##   .. .. .. .. .. .. .. .. ..$ : NULL
##   .. .. .. .. .. .. .. .. ..$ : chr [1:2] "x" "y"
##   .. .. .. .. ..$ :Formal class 'Polygon' [package "sp"] with 5 slots
##   .. .. .. .. .. .. ..@ labpt  : num [1:2] 134.42 -6.13
##   .. .. .. .. .. .. ..@ area   : num 0.516
##   .. .. .. .. .. .. ..@ hole   : logi FALSE
##   .. .. .. .. .. .. ..@ ringDir: int 1
##   .. .. .. .. .. .. ..@ coords : num [1:7, 1:2] 135 134 134 134 134 ...
##   .. .. .. .. .. .. .. ..- attr(*, "dimnames")=List of 2
##   .. .. .. .. .. .. .. .. ..$ : NULL
##   .. .. .. .. .. .. .. .. ..$ : chr [1:2] "x" "y"
##   .. .. .. .. ..$ :Formal class 'Polygon' [package "sp"] with 5 slots
##   .. .. .. .. .. .. ..@ labpt  : num [1:2] 126.64 -3.42
##   .. .. .. .. .. .. ..@ area   : num 0.565
##   .. .. .. .. .. .. ..@ hole   : logi FALSE
##   .. .. .. .. .. .. ..@ ringDir: int 1
##   .. .. .. .. .. .. ..@ coords : num [1:6, 1:2] 127 127 126 126 127 ...
##   .. .. .. .. .. .. .. ..- attr(*, "dimnames")=List of 2
##   .. .. .. .. .. .. .. .. ..$ : NULL
##   .. .. .. .. .. .. .. .. ..$ : chr [1:2] "x" "y"
##   .. .. .. .. ..$ :Formal class 'Polygon' [package "sp"] with 5 slots
##   .. .. .. .. .. .. ..@ labpt  : num [1:2] 129.32 -3.19
##   .. .. .. .. .. .. ..@ area   : num 1.47
##   .. .. .. .. .. .. ..@ hole   : logi FALSE
##   .. .. .. .. .. .. ..@ ringDir: int 1
##   .. .. .. .. .. .. ..@ coords : num [1:9, 1:2] 130 131 130 129 129 ...
##   .. .. .. .. .. .. .. ..- attr(*, "dimnames")=List of 2
##   .. .. .. .. .. .. .. .. ..$ : NULL
##   .. .. .. .. .. .. .. .. ..$ : chr [1:2] "x" "y"
##   .. .. .. .. ..$ :Formal class 'Polygon' [package "sp"] with 5 slots
##   .. .. .. .. .. .. ..@ labpt  : num [1:2] 137.41 -4.13
##   .. .. .. .. .. .. ..@ area   : num 33.4
##   .. .. .. .. .. .. ..@ hole   : logi FALSE
##   .. .. .. .. .. .. ..@ ringDir: int 1
##   .. .. .. .. .. .. ..@ coords : num [1:38, 1:2] 134 134 135 136 137 ...
##   .. .. .. .. .. .. .. ..- attr(*, "dimnames")=List of 2
##   .. .. .. .. .. .. .. .. ..$ : NULL
##   .. .. .. .. .. .. .. .. ..$ : chr [1:2] "x" "y"
##   .. .. .. .. ..$ :Formal class 'Polygon' [package "sp"] with 5 slots
##   .. .. .. .. .. .. ..@ labpt  : num [1:2] 121.2 -2.11
##   .. .. .. .. .. .. ..@ area   : num 15.3
##   .. .. .. .. .. .. ..@ hole   : logi FALSE
##   .. .. .. .. .. .. ..@ ringDir: int 1
##   .. .. .. .. .. .. ..@ coords : num [1:45, 1:2] 125 124 124 123 121 ...
##   .. .. .. .. .. .. .. ..- attr(*, "dimnames")=List of 2
##   .. .. .. .. .. .. .. .. ..$ : NULL
##   .. .. .. .. .. .. .. .. ..$ : chr [1:2] "x" "y"
##   .. .. .. .. ..$ :Formal class 'Polygon' [package "sp"] with 5 slots
##   .. .. .. .. .. .. ..@ labpt  : num [1:2] 128.016 0.779
##   .. .. .. .. .. .. ..@ area   : num 2.09
##   .. .. .. .. .. .. ..@ hole   : logi FALSE
##   .. .. .. .. .. .. ..@ ringDir: int 1
##   .. .. .. .. .. .. ..@ coords : num [1:13, 1:2] 129 129 128 128 128 ...
##   .. .. .. .. .. .. .. ..- attr(*, "dimnames")=List of 2
##   .. .. .. .. .. .. .. .. ..$ : NULL
##   .. .. .. .. .. .. .. .. ..$ : chr [1:2] "x" "y"
##   .. .. .. .. ..$ :Formal class 'Polygon' [package "sp"] with 5 slots
##   .. .. .. .. .. .. ..@ labpt  : num [1:2] 114.023 -0.254
##   .. .. .. .. .. .. ..@ area   : num 43.1
##   .. .. .. .. .. .. ..@ hole   : logi FALSE
##   .. .. .. .. .. .. ..@ ringDir: int 1
##   .. .. .. .. .. .. ..@ coords : num [1:39, 1:2] 118 119 118 117 118 ...
##   .. .. .. .. .. .. .. ..- attr(*, "dimnames")=List of 2
##   .. .. .. .. .. .. .. .. ..$ : NULL
##   .. .. .. .. .. .. .. .. ..$ : chr [1:2] "x" "y"
##   .. .. .. .. ..$ :Formal class 'Polygon' [package "sp"] with 5 slots
##   .. .. .. .. .. .. ..@ labpt  : num [1:2] 101.554 -0.413
##   .. .. .. .. .. .. ..@ area   : num 35.8
##   .. .. .. .. .. .. ..@ hole   : logi FALSE
##   .. .. .. .. .. .. ..@ ringDir: int 1
##   .. .. .. .. .. .. ..@ coords : num [1:35, 1:2] 106 105 104 103 102 ...
##   .. .. .. .. .. .. .. ..- attr(*, "dimnames")=List of 2
##   .. .. .. .. .. .. .. .. ..$ : NULL
##   .. .. .. .. .. .. .. .. ..$ : chr [1:2] "x" "y"
##   .. .. .. ..@ plotOrder: int [1:13] 12 13 9 10 5 11 8 4 2 3 ...
##   .. .. .. ..@ labpt    : num [1:2] 114.023 -0.254
##   .. .. .. ..@ ID       : chr "72"
##   .. .. .. ..@ area     : num 148
##   .. ..$ :Formal class 'Polygons' [package "sp"] with 5 slots
##   .. .. .. ..@ Polygons :List of 1
##   .. .. .. .. ..$ :Formal class 'Polygon' [package "sp"] with 5 slots
##   .. .. .. .. .. .. ..@ labpt  : num [1:2] 79.6 22.9
##   .. .. .. .. .. .. ..@ area   : num 278
##   .. .. .. .. .. .. ..@ hole   : logi FALSE
##   .. .. .. .. .. .. ..@ ringDir: int 1
##   .. .. .. .. .. .. ..@ coords : num [1:136, 1:2] 77.8 78.9 78.8 79.2 79.2 ...
##   .. .. .. .. .. .. .. ..- attr(*, "dimnames")=List of 2
##   .. .. .. .. .. .. .. .. ..$ : NULL
##   .. .. .. .. .. .. .. .. ..$ : chr [1:2] "x" "y"
##   .. .. .. ..@ plotOrder: int 1
##   .. .. .. ..@ labpt    : num [1:2] 79.6 22.9
##   .. .. .. ..@ ID       : chr "73"
##   .. .. .. ..@ area     : num 278
##   .. ..$ :Formal class 'Polygons' [package "sp"] with 5 slots
##   .. .. .. ..@ Polygons :List of 1
##   .. .. .. .. ..$ :Formal class 'Polygon' [package "sp"] with 5 slots
##   .. .. .. .. .. .. ..@ labpt  : num [1:2] -8.01 53.18
##   .. .. .. .. .. .. ..@ area   : num 7.86
##   .. .. .. .. .. .. ..@ hole   : logi FALSE
##   .. .. .. .. .. .. ..@ ringDir: int 1
##   .. .. .. .. .. .. ..@ coords : num [1:13, 1:2] -6.2 -6.03 -6.79 -8.56 -9.98 ...
##   .. .. .. .. .. .. .. ..- attr(*, "dimnames")=List of 2
##   .. .. .. .. .. .. .. .. ..$ : NULL
##   .. .. .. .. .. .. .. .. ..$ : chr [1:2] "x" "y"
##   .. .. .. ..@ plotOrder: int 1
##   .. .. .. ..@ labpt    : num [1:2] -8.01 53.18
##   .. .. .. ..@ ID       : chr "74"
##   .. .. .. ..@ area     : num 7.86
##   .. ..$ :Formal class 'Polygons' [package "sp"] with 5 slots
##   .. .. .. ..@ Polygons :List of 1
##   .. .. .. .. ..$ :Formal class 'Polygon' [package "sp"] with 5 slots
##   .. .. .. .. .. .. ..@ labpt  : num [1:2] 54.3 32.5
##   .. .. .. .. .. .. ..@ area   : num 156
##   .. .. .. .. .. .. ..@ hole   : logi FALSE
##   .. .. .. .. .. .. ..@ ringDir: int 1
##   .. .. .. .. .. .. ..@ coords : num [1:75, 1:2] 53.9 54.8 55.5 56.2 56.6 ...
##   .. .. .. .. .. .. .. ..- attr(*, "dimnames")=List of 2
##   .. .. .. .. .. .. .. .. ..$ : NULL
##   .. .. .. .. .. .. .. .. ..$ : chr [1:2] "x" "y"
##   .. .. .. ..@ plotOrder: int 1
##   .. .. .. ..@ labpt    : num [1:2] 54.3 32.5
##   .. .. .. ..@ ID       : chr "75"
##   .. .. .. ..@ area     : num 156
##   .. ..$ :Formal class 'Polygons' [package "sp"] with 5 slots
##   .. .. .. ..@ Polygons :List of 1
##   .. .. .. .. ..$ :Formal class 'Polygon' [package "sp"] with 5 slots
##   .. .. .. .. .. .. ..@ labpt  : num [1:2] 43.8 33
##   .. .. .. .. .. .. ..@ area   : num 42.2
##   .. .. .. .. .. .. ..@ hole   : logi FALSE
##   .. .. .. .. .. .. ..@ ringDir: int 1
##   .. .. .. .. .. .. ..@ coords : num [1:30, 1:2] 45.4 46.1 46.2 45.6 45.4 ...
##   .. .. .. .. .. .. .. ..- attr(*, "dimnames")=List of 2
##   .. .. .. .. .. .. .. .. ..$ : NULL
##   .. .. .. .. .. .. .. .. ..$ : chr [1:2] "x" "y"
##   .. .. .. ..@ plotOrder: int 1
##   .. .. .. ..@ labpt    : num [1:2] 43.8 33
##   .. .. .. ..@ ID       : chr "76"
##   .. .. .. ..@ area     : num 42.2
##   .. ..$ :Formal class 'Polygons' [package "sp"] with 5 slots
##   .. .. .. ..@ Polygons :List of 1
##   .. .. .. .. ..$ :Formal class 'Polygon' [package "sp"] with 5 slots
##   .. .. .. .. .. .. ..@ labpt  : num [1:2] -18.8 65.1
##   .. .. .. .. .. .. ..@ area   : num 20.6
##   .. .. .. .. .. .. ..@ hole   : logi FALSE
##   .. .. .. .. .. .. ..@ ringDir: int 1
##   .. .. .. .. .. .. ..@ coords : num [1:20, 1:2] -14.5 -14.7 -13.6 -14.9 -17.8 ...
##   .. .. .. .. .. .. .. ..- attr(*, "dimnames")=List of 2
##   .. .. .. .. .. .. .. .. ..$ : NULL
##   .. .. .. .. .. .. .. .. ..$ : chr [1:2] "x" "y"
##   .. .. .. ..@ plotOrder: int 1
##   .. .. .. ..@ labpt    : num [1:2] -18.8 65.1
##   .. .. .. ..@ ID       : chr "77"
##   .. .. .. ..@ area     : num 20.6
##   .. ..$ :Formal class 'Polygons' [package "sp"] with 5 slots
##   .. .. .. ..@ Polygons :List of 1
##   .. .. .. .. ..$ :Formal class 'Polygon' [package "sp"] with 5 slots
##   .. .. .. .. .. .. ..@ labpt  : num [1:2] 35 31.5
##   .. .. .. .. .. .. ..@ area   : num 2.19
##   .. .. .. .. .. .. ..@ hole   : logi FALSE
##   .. .. .. .. .. .. ..@ ringDir: int 1
##   .. .. .. .. .. .. ..@ coords : num [1:23, 1:2] 35.7 35.5 35.2 35 35.2 ...
##   .. .. .. .. .. .. .. ..- attr(*, "dimnames")=List of 2
##   .. .. .. .. .. .. .. .. ..$ : NULL
##   .. .. .. .. .. .. .. .. ..$ : chr [1:2] "x" "y"
##   .. .. .. ..@ plotOrder: int 1
##   .. .. .. ..@ labpt    : num [1:2] 35 31.5
##   .. .. .. ..@ ID       : chr "78"
##   .. .. .. ..@ area     : num 2.19
##   .. ..$ :Formal class 'Polygons' [package "sp"] with 5 slots
##   .. .. .. ..@ Polygons :List of 3
##   .. .. .. .. ..$ :Formal class 'Polygon' [package "sp"] with 5 slots
##   .. .. .. .. .. .. ..@ labpt  : num [1:2] 14.1 37.6
##   .. .. .. .. .. .. ..@ area   : num 2.79
##   .. .. .. .. .. .. ..@ hole   : logi FALSE
##   .. .. .. .. .. .. ..@ ringDir: int 1
##   .. .. .. .. .. .. ..@ coords : num [1:11, 1:2] 15.5 15.2 15.3 15.1 14.3 ...
##   .. .. .. .. .. .. .. ..- attr(*, "dimnames")=List of 2
##   .. .. .. .. .. .. .. .. ..$ : NULL
##   .. .. .. .. .. .. .. .. ..$ : chr [1:2] "x" "y"
##   .. .. .. .. ..$ :Formal class 'Polygon' [package "sp"] with 5 slots
##   .. .. .. .. .. .. ..@ labpt  : num [1:2] 9.03 40.07
##   .. .. .. .. .. .. ..@ area   : num 2.53
##   .. .. .. .. .. .. ..@ hole   : logi FALSE
##   .. .. .. .. .. .. ..@ ringDir: int 1
##   .. .. .. .. .. .. ..@ coords : num [1:10, 1:2] 9.21 9.81 9.67 9.21 8.81 ...
##   .. .. .. .. .. .. .. ..- attr(*, "dimnames")=List of 2
##   .. .. .. .. .. .. .. .. ..$ : NULL
##   .. .. .. .. .. .. .. .. ..$ : chr [1:2] "x" "y"
##   .. .. .. .. ..$ :Formal class 'Polygon' [package "sp"] with 5 slots
##   .. .. .. .. .. .. ..@ labpt  : num [1:2] 12.2 43.5
##   .. .. .. .. .. .. ..@ area   : num 29.4
##   .. .. .. .. .. .. ..@ hole   : logi FALSE
##   .. .. .. .. .. .. ..@ ringDir: int 1
##   .. .. .. .. .. .. ..@ coords : num [1:66, 1:2] 12.4 13.8 13.7 13.9 13.1 ...
##   .. .. .. .. .. .. .. ..- attr(*, "dimnames")=List of 2
##   .. .. .. .. .. .. .. .. ..$ : NULL
##   .. .. .. .. .. .. .. .. ..$ : chr [1:2] "x" "y"
##   .. .. .. ..@ plotOrder: int [1:3] 3 1 2
##   .. .. .. ..@ labpt    : num [1:2] 12.2 43.5
##   .. .. .. ..@ ID       : chr "79"
##   .. .. .. ..@ area     : num 34.7
##   .. ..$ :Formal class 'Polygons' [package "sp"] with 5 slots
##   .. .. .. ..@ Polygons :List of 1
##   .. .. .. .. ..$ :Formal class 'Polygon' [package "sp"] with 5 slots
##   .. .. .. .. .. .. ..@ labpt  : num [1:2] -77.3 18.1
##   .. .. .. .. .. .. ..@ area   : num 1.06
##   .. .. .. .. .. .. ..@ hole   : logi FALSE
##   .. .. .. .. .. .. ..@ ringDir: int 1
##   .. .. .. .. .. .. ..@ coords : num [1:11, 1:2] -77.6 -76.9 -76.4 -76.2 -76.9 ...
##   .. .. .. .. .. .. .. ..- attr(*, "dimnames")=List of 2
##   .. .. .. .. .. .. .. .. ..$ : NULL
##   .. .. .. .. .. .. .. .. ..$ : chr [1:2] "x" "y"
##   .. .. .. ..@ plotOrder: int 1
##   .. .. .. ..@ labpt    : num [1:2] -77.3 18.1
##   .. .. .. ..@ ID       : chr "80"
##   .. .. .. ..@ area     : num 1.06
##   .. ..$ :Formal class 'Polygons' [package "sp"] with 5 slots
##   .. .. .. ..@ Polygons :List of 1
##   .. .. .. .. ..$ :Formal class 'Polygon' [package "sp"] with 5 slots
##   .. .. .. .. .. .. ..@ labpt  : num [1:2] 36.8 31.2
##   .. .. .. .. .. .. ..@ area   : num 8.44
##   .. .. .. .. .. .. ..@ hole   : logi FALSE
##   .. .. .. .. .. .. ..@ ringDir: int 1
##   .. .. .. .. .. .. ..@ coords : num [1:19, 1:2] 35.5 35.7 36.8 38.8 39.2 ...
##   .. .. .. .. .. .. .. ..- attr(*, "dimnames")=List of 2
##   .. .. .. .. .. .. .. .. ..$ : NULL
##   .. .. .. .. .. .. .. .. ..$ : chr [1:2] "x" "y"
##   .. .. .. ..@ plotOrder: int 1
##   .. .. .. ..@ labpt    : num [1:2] 36.8 31.2
##   .. .. .. ..@ ID       : chr "81"
##   .. .. .. ..@ area     : num 8.44
##   .. ..$ :Formal class 'Polygons' [package "sp"] with 5 slots
##   .. .. .. ..@ Polygons :List of 3
##   .. .. .. .. ..$ :Formal class 'Polygon' [package "sp"] with 5 slots
##   .. .. .. .. .. .. ..@ labpt  : num [1:2] 133.5 33.6
##   .. .. .. .. .. .. ..@ area   : num 1.98
##   .. .. .. .. .. .. ..@ hole   : logi FALSE
##   .. .. .. .. .. .. ..@ ringDir: int 1
##   .. .. .. .. .. .. ..@ coords : num [1:12, 1:2] 135 135 134 134 133 ...
##   .. .. .. .. .. .. .. ..- attr(*, "dimnames")=List of 2
##   .. .. .. .. .. .. .. .. ..$ : NULL
##   .. .. .. .. .. .. .. .. ..$ : chr [1:2] "x" "y"
##   .. .. .. .. ..$ :Formal class 'Polygon' [package "sp"] with 5 slots
##   .. .. .. .. .. .. ..@ labpt  : num [1:2] 137 36
##   .. .. .. .. .. .. ..@ area   : num 29.4
##   .. .. .. .. .. .. ..@ hole   : logi FALSE
##   .. .. .. .. .. .. ..@ ringDir: int 1
##   .. .. .. .. .. .. ..@ coords : num [1:37, 1:2] 141 141 141 140 139 ...
##   .. .. .. .. .. .. .. ..- attr(*, "dimnames")=List of 2
##   .. .. .. .. .. .. .. .. ..$ : NULL
##   .. .. .. .. .. .. .. .. ..$ : chr [1:2] "x" "y"
##   .. .. .. .. ..$ :Formal class 'Polygon' [package "sp"] with 5 slots
##   .. .. .. .. .. .. ..@ labpt  : num [1:2] 142.5 43.3
##   .. .. .. .. .. .. ..@ area   : num 9.98
##   .. .. .. .. .. .. ..@ hole   : logi FALSE
##   .. .. .. .. .. .. ..@ ringDir: int 1
##   .. .. .. .. .. .. ..@ coords : num [1:16, 1:2] 144 145 145 146 144 ...
##   .. .. .. .. .. .. .. ..- attr(*, "dimnames")=List of 2
##   .. .. .. .. .. .. .. .. ..$ : NULL
##   .. .. .. .. .. .. .. .. ..$ : chr [1:2] "x" "y"
##   .. .. .. ..@ plotOrder: int [1:3] 2 3 1
##   .. .. .. ..@ labpt    : num [1:2] 137 36
##   .. .. .. ..@ ID       : chr "82"
##   .. .. .. ..@ area     : num 41.4
##   .. ..$ :Formal class 'Polygons' [package "sp"] with 5 slots
##   .. .. .. ..@ Polygons :List of 1
##   .. .. .. .. ..$ :Formal class 'Polygon' [package "sp"] with 5 slots
##   .. .. .. .. .. .. ..@ labpt  : num [1:2] 67.3 48.2
##   .. .. .. .. .. .. ..@ area   : num 331
##   .. .. .. .. .. .. ..@ hole   : logi FALSE
##   .. .. .. .. .. .. ..@ ringDir: int 1
##   .. .. .. .. .. .. ..@ coords : num [1:112, 1:2] 71 70.4 69.1 68.6 68.3 ...
##   .. .. .. .. .. .. .. ..- attr(*, "dimnames")=List of 2
##   .. .. .. .. .. .. .. .. ..$ : NULL
##   .. .. .. .. .. .. .. .. ..$ : chr [1:2] "x" "y"
##   .. .. .. ..@ plotOrder: int 1
##   .. .. .. ..@ labpt    : num [1:2] 67.3 48.2
##   .. .. .. ..@ ID       : chr "83"
##   .. .. .. ..@ area     : num 331
##   .. ..$ :Formal class 'Polygons' [package "sp"] with 5 slots
##   .. .. .. ..@ Polygons :List of 1
##   .. .. .. .. ..$ :Formal class 'Polygon' [package "sp"] with 5 slots
##   .. .. .. .. .. .. ..@ labpt  : num [1:2] 37.792 0.596
##   .. .. .. .. .. .. ..@ area   : num 48
##   .. .. .. .. .. .. ..@ hole   : logi FALSE
##   .. .. .. .. .. .. ..@ ringDir: int 1
##   .. .. .. .. .. .. ..@ coords : num [1:37, 1:2] 41 41.6 40.9 40.6 40.3 ...
##   .. .. .. .. .. .. .. ..- attr(*, "dimnames")=List of 2
##   .. .. .. .. .. .. .. .. ..$ : NULL
##   .. .. .. .. .. .. .. .. ..$ : chr [1:2] "x" "y"
##   .. .. .. ..@ plotOrder: int 1
##   .. .. .. ..@ labpt    : num [1:2] 37.792 0.596
##   .. .. .. ..@ ID       : chr "84"
##   .. .. .. ..@ area     : num 48
##   .. ..$ :Formal class 'Polygons' [package "sp"] with 5 slots
##   .. .. .. ..@ Polygons :List of 1
##   .. .. .. .. ..$ :Formal class 'Polygon' [package "sp"] with 5 slots
##   .. .. .. .. .. .. ..@ labpt  : num [1:2] 74.6 41.5
##   .. .. .. .. .. .. ..@ area   : num 21.1
##   .. .. .. .. .. .. ..@ hole   : logi FALSE
##   .. .. .. .. .. .. ..@ ringDir: int 1
##   .. .. .. .. .. .. ..@ coords : num [1:35, 1:2] 71 71.2 71.8 73.5 73.6 ...
##   .. .. .. .. .. .. .. ..- attr(*, "dimnames")=List of 2
##   .. .. .. .. .. .. .. .. ..$ : NULL
##   .. .. .. .. .. .. .. .. ..$ : chr [1:2] "x" "y"
##   .. .. .. ..@ plotOrder: int 1
##   .. .. .. ..@ labpt    : num [1:2] 74.6 41.5
##   .. .. .. ..@ ID       : chr "85"
##   .. .. .. ..@ area     : num 21.1
##   .. ..$ :Formal class 'Polygons' [package "sp"] with 5 slots
##   .. .. .. ..@ Polygons :List of 1
##   .. .. .. .. ..$ :Formal class 'Polygon' [package "sp"] with 5 slots
##   .. .. .. .. .. .. ..@ labpt  : num [1:2] 104.9 12.7
##   .. .. .. .. .. .. ..@ area   : num 15.2
##   .. .. .. .. .. .. ..@ hole   : logi FALSE
##   .. .. .. .. .. .. ..@ ringDir: int 1
##   .. .. .. .. .. .. ..@ coords : num [1:17, 1:2] 103 103 103 102 103 ...
##   .. .. .. .. .. .. .. ..- attr(*, "dimnames")=List of 2
##   .. .. .. .. .. .. .. .. ..$ : NULL
##   .. .. .. .. .. .. .. .. ..$ : chr [1:2] "x" "y"
##   .. .. .. ..@ plotOrder: int 1
##   .. .. .. ..@ labpt    : num [1:2] 104.9 12.7
##   .. .. .. ..@ ID       : chr "86"
##   .. .. .. ..@ area     : num 15.2
##   .. ..$ :Formal class 'Polygons' [package "sp"] with 5 slots
##   .. .. .. ..@ Polygons :List of 1
##   .. .. .. .. ..$ :Formal class 'Polygon' [package "sp"] with 5 slots
##   .. .. .. .. .. .. ..@ labpt  : num [1:2] 127.8 36.4
##   .. .. .. .. .. .. ..@ area   : num 9.95
##   .. .. .. .. .. .. ..@ hole   : logi FALSE
##   .. .. .. .. .. .. ..@ ringDir: int 1
##   .. .. .. .. .. .. ..@ coords : num [1:19, 1:2] 128 129 129 129 129 ...
##   .. .. .. .. .. .. .. ..- attr(*, "dimnames")=List of 2
##   .. .. .. .. .. .. .. .. ..$ : NULL
##   .. .. .. .. .. .. .. .. ..$ : chr [1:2] "x" "y"
##   .. .. .. ..@ plotOrder: int 1
##   .. .. .. ..@ labpt    : num [1:2] 127.8 36.4
##   .. .. .. ..@ ID       : chr "87"
##   .. .. .. ..@ area     : num 9.95
##   .. ..$ :Formal class 'Polygons' [package "sp"] with 5 slots
##   .. .. .. ..@ Polygons :List of 1
##   .. .. .. .. ..$ :Formal class 'Polygon' [package "sp"] with 5 slots
##   .. .. .. .. .. .. ..@ labpt  : num [1:2] 20.9 42.6
##   .. .. .. .. .. .. ..@ area   : num 1.23
##   .. .. .. .. .. .. ..@ hole   : logi FALSE
##   .. .. .. .. .. .. ..@ ringDir: int 1
##   .. .. .. .. .. .. ..@ coords : num [1:21, 1:2] 20.8 20.7 20.6 20.5 20.3 ...
##   .. .. .. .. .. .. .. ..- attr(*, "dimnames")=List of 2
##   .. .. .. .. .. .. .. .. ..$ : NULL
##   .. .. .. .. .. .. .. .. ..$ : chr [1:2] "x" "y"
##   .. .. .. ..@ plotOrder: int 1
##   .. .. .. ..@ labpt    : num [1:2] 20.9 42.6
##   .. .. .. ..@ ID       : chr "88"
##   .. .. .. ..@ area     : num 1.23
##   .. ..$ :Formal class 'Polygons' [package "sp"] with 5 slots
##   .. .. .. ..@ Polygons :List of 1
##   .. .. .. .. ..$ :Formal class 'Polygon' [package "sp"] with 5 slots
##   .. .. .. .. .. .. ..@ labpt  : num [1:2] 47.6 29.3
##   .. .. .. .. .. .. ..@ area   : num 1.55
##   .. .. .. .. .. .. ..@ hole   : logi FALSE
##   .. .. .. .. .. .. ..@ ringDir: int 1
##   .. .. .. .. .. .. ..@ coords : num [1:9, 1:2] 48 48.2 48.1 48.4 47.7 ...
##   .. .. .. .. .. .. .. ..- attr(*, "dimnames")=List of 2
##   .. .. .. .. .. .. .. .. ..$ : NULL
##   .. .. .. .. .. .. .. .. ..$ : chr [1:2] "x" "y"
##   .. .. .. ..@ plotOrder: int 1
##   .. .. .. ..@ labpt    : num [1:2] 47.6 29.3
##   .. .. .. ..@ ID       : chr "89"
##   .. .. .. ..@ area     : num 1.55
##   .. ..$ :Formal class 'Polygons' [package "sp"] with 5 slots
##   .. .. .. ..@ Polygons :List of 1
##   .. .. .. .. ..$ :Formal class 'Polygon' [package "sp"] with 5 slots
##   .. .. .. .. .. .. ..@ labpt  : num [1:2] 103.8 18.4
##   .. .. .. .. .. .. ..@ area   : num 19.6
##   .. .. .. .. .. .. ..@ hole   : logi FALSE
##   .. .. .. .. .. .. ..@ ringDir: int 1
##   .. .. .. .. .. .. ..@ coords : num [1:37, 1:2] 105 106 106 105 105 ...
##   .. .. .. .. .. .. .. ..- attr(*, "dimnames")=List of 2
##   .. .. .. .. .. .. .. .. ..$ : NULL
##   .. .. .. .. .. .. .. .. ..$ : chr [1:2] "x" "y"
##   .. .. .. ..@ plotOrder: int 1
##   .. .. .. ..@ labpt    : num [1:2] 103.8 18.4
##   .. .. .. ..@ ID       : chr "90"
##   .. .. .. ..@ area     : num 19.6
##   .. ..$ :Formal class 'Polygons' [package "sp"] with 5 slots
##   .. .. .. ..@ Polygons :List of 1
##   .. .. .. .. ..$ :Formal class 'Polygon' [package "sp"] with 5 slots
##   .. .. .. .. .. .. ..@ labpt  : num [1:2] 35.9 33.9
##   .. .. .. .. .. .. ..@ area   : num 0.984
##   .. .. .. .. .. .. ..@ hole   : logi FALSE
##   .. .. .. .. .. .. ..@ ringDir: int 1
##   .. .. .. .. .. .. ..@ coords : num [1:11, 1:2] 35.8 35.6 35.5 35.1 35.5 ...
##   .. .. .. .. .. .. .. ..- attr(*, "dimnames")=List of 2
##   .. .. .. .. .. .. .. .. ..$ : NULL
##   .. .. .. .. .. .. .. .. ..$ : chr [1:2] "x" "y"
##   .. .. .. ..@ plotOrder: int 1
##   .. .. .. ..@ labpt    : num [1:2] 35.9 33.9
##   .. .. .. ..@ ID       : chr "91"
##   .. .. .. ..@ area     : num 0.984
##   .. ..$ :Formal class 'Polygons' [package "sp"] with 5 slots
##   .. .. .. ..@ Polygons :List of 1
##   .. .. .. .. ..$ :Formal class 'Polygon' [package "sp"] with 5 slots
##   .. .. .. .. .. .. ..@ labpt  : num [1:2] -9.41 6.43
##   .. .. .. .. .. .. ..@ area   : num 8.03
##   .. .. .. .. .. .. ..@ hole   : logi FALSE
##   .. .. .. .. .. .. ..@ ringDir: int 1
##   .. .. .. .. .. .. ..@ coords : num [1:27, 1:2] -7.71 -7.97 -9 -9.91 -10.77 ...
##   .. .. .. .. .. .. .. ..- attr(*, "dimnames")=List of 2
##   .. .. .. .. .. .. .. .. ..$ : NULL
##   .. .. .. .. .. .. .. .. ..$ : chr [1:2] "x" "y"
##   .. .. .. ..@ plotOrder: int 1
##   .. .. .. ..@ labpt    : num [1:2] -9.41 6.43
##   .. .. .. ..@ ID       : chr "92"
##   .. .. .. ..@ area     : num 8.03
##   .. ..$ :Formal class 'Polygons' [package "sp"] with 5 slots
##   .. .. .. ..@ Polygons :List of 1
##   .. .. .. .. ..$ :Formal class 'Polygon' [package "sp"] with 5 slots
##   .. .. .. .. .. .. ..@ labpt  : num [1:2] 18 27
##   .. .. .. .. .. .. ..@ area   : num 149
##   .. .. .. .. .. .. ..@ hole   : logi FALSE
##   .. .. .. .. .. .. ..@ ringDir: int 1
##   .. .. .. .. .. .. ..@ coords : num [1:56, 1:2] 14.9 14.1 13.6 12 11.6 ...
##   .. .. .. .. .. .. .. ..- attr(*, "dimnames")=List of 2
##   .. .. .. .. .. .. .. .. ..$ : NULL
##   .. .. .. .. .. .. .. .. ..$ : chr [1:2] "x" "y"
##   .. .. .. ..@ plotOrder: int 1
##   .. .. .. ..@ labpt    : num [1:2] 18 27
##   .. .. .. ..@ ID       : chr "93"
##   .. .. .. ..@ area     : num 149
##   .. ..$ :Formal class 'Polygons' [package "sp"] with 5 slots
##   .. .. .. ..@ Polygons :List of 1
##   .. .. .. .. ..$ :Formal class 'Polygon' [package "sp"] with 5 slots
##   .. .. .. .. .. .. ..@ labpt  : num [1:2] 80.7 7.7
##   .. .. .. .. .. .. ..@ area   : num 5.36
##   .. .. .. .. .. .. ..@ hole   : logi FALSE
##   .. .. .. .. .. .. ..@ ringDir: int 1
##   .. .. .. .. .. .. ..@ coords : num [1:10, 1:2] 81.8 81.6 81.2 80.3 79.9 ...
##   .. .. .. .. .. .. .. ..- attr(*, "dimnames")=List of 2
##   .. .. .. .. .. .. .. .. ..$ : NULL
##   .. .. .. .. .. .. .. .. ..$ : chr [1:2] "x" "y"
##   .. .. .. ..@ plotOrder: int 1
##   .. .. .. ..@ labpt    : num [1:2] 80.7 7.7
##   .. .. .. ..@ ID       : chr "94"
##   .. .. .. ..@ area     : num 5.36
##   .. ..$ :Formal class 'Polygons' [package "sp"] with 5 slots
##   .. .. .. ..@ Polygons :List of 1
##   .. .. .. .. ..$ :Formal class 'Polygon' [package "sp"] with 5 slots
##   .. .. .. .. .. .. ..@ labpt  : num [1:2] 28.2 -29.6
##   .. .. .. .. .. .. ..@ area   : num 2.56
##   .. .. .. .. .. .. ..@ hole   : logi FALSE
##   .. .. .. .. .. .. ..@ ringDir: int 1
##   .. .. .. .. .. .. ..@ coords : num [1:12, 1:2] 29 29.3 29 28.8 28.3 ...
##   .. .. .. .. .. .. .. ..- attr(*, "dimnames")=List of 2
##   .. .. .. .. .. .. .. .. ..$ : NULL
##   .. .. .. .. .. .. .. .. ..$ : chr [1:2] "x" "y"
##   .. .. .. ..@ plotOrder: int 1
##   .. .. .. ..@ labpt    : num [1:2] 28.2 -29.6
##   .. .. .. ..@ ID       : chr "95"
##   .. .. .. ..@ area     : num 2.56
##   .. ..$ :Formal class 'Polygons' [package "sp"] with 5 slots
##   .. .. .. ..@ Polygons :List of 1
##   .. .. .. .. ..$ :Formal class 'Polygon' [package "sp"] with 5 slots
##   .. .. .. .. .. .. ..@ labpt  : num [1:2] 23.9 55.3
##   .. .. .. .. .. .. ..@ area   : num 9.02
##   .. .. .. .. .. .. ..@ hole   : logi FALSE
##   .. .. .. .. .. .. ..@ ringDir: int 1
##   .. .. .. .. .. .. ..@ coords : num [1:19, 1:2] 22.7 22.7 22.8 22.3 21.3 ...
##   .. .. .. .. .. .. .. ..- attr(*, "dimnames")=List of 2
##   .. .. .. .. .. .. .. .. ..$ : NULL
##   .. .. .. .. .. .. .. .. ..$ : chr [1:2] "x" "y"
##   .. .. .. ..@ plotOrder: int 1
##   .. .. .. ..@ labpt    : num [1:2] 23.9 55.3
##   .. .. .. ..@ ID       : chr "96"
##   .. .. .. ..@ area     : num 9.02
##   .. ..$ :Formal class 'Polygons' [package "sp"] with 5 slots
##   .. .. .. ..@ Polygons :List of 1
##   .. .. .. .. ..$ :Formal class 'Polygon' [package "sp"] with 5 slots
##   .. .. .. .. .. .. ..@ labpt  : num [1:2] 5.97 49.77
##   .. .. .. .. .. .. ..@ area   : num 0.302
##   .. .. .. .. .. .. ..@ hole   : logi FALSE
##   .. .. .. .. .. .. ..@ ringDir: int 1
##   .. .. .. .. .. .. ..@ coords : num [1:7, 1:2] 6.04 6.24 6.19 5.9 5.67 ...
##   .. .. .. .. .. .. .. ..- attr(*, "dimnames")=List of 2
##   .. .. .. .. .. .. .. .. ..$ : NULL
##   .. .. .. .. .. .. .. .. ..$ : chr [1:2] "x" "y"
##   .. .. .. ..@ plotOrder: int 1
##   .. .. .. ..@ labpt    : num [1:2] 5.97 49.77
##   .. .. .. ..@ ID       : chr "97"
##   .. .. .. ..@ area     : num 0.302
##   .. ..$ :Formal class 'Polygons' [package "sp"] with 5 slots
##   .. .. .. ..@ Polygons :List of 1
##   .. .. .. .. ..$ :Formal class 'Polygon' [package "sp"] with 5 slots
##   .. .. .. .. .. .. ..@ labpt  : num [1:2] 24.8 56.8
##   .. .. .. .. .. .. ..@ area   : num 9.4
##   .. .. .. .. .. .. ..@ hole   : logi FALSE
##   .. .. .. .. .. .. ..@ ringDir: int 1
##   .. .. .. .. .. .. ..@ coords : num [1:22, 1:2] 21.1 21.1 21.6 22.5 23.3 ...
##   .. .. .. .. .. .. .. ..- attr(*, "dimnames")=List of 2
##   .. .. .. .. .. .. .. .. ..$ : NULL
##   .. .. .. .. .. .. .. .. ..$ : chr [1:2] "x" "y"
##   .. .. .. ..@ plotOrder: int 1
##   .. .. .. ..@ labpt    : num [1:2] 24.8 56.8
##   .. .. .. ..@ ID       : chr "98"
##   .. .. .. ..@ area     : num 9.4
##   .. .. [list output truncated]
##   ..@ plotOrder  : int [1:177] 7 136 28 169 31 23 9 66 84 5 ...
##   ..@ bbox       : num [1:2, 1:2] -180 -90 180 83.6
##   .. ..- attr(*, "dimnames")=List of 2
##   .. .. ..$ : chr [1:2] "x" "y"
##   .. .. ..$ : chr [1:2] "min" "max"
##   ..@ proj4string:Formal class 'CRS' [package "sp"] with 1 slot
##   .. .. ..@ projargs: chr "+proj=longlat +datum=WGS84 +no_defs +ellps=WGS84 +towgs84=0,0,0"
\end{verbatim}

\begin{Shaded}
\begin{Highlighting}[]
\CommentTok{# Call str() on countries_sp with max.level = 2}
\KeywordTok{str}\NormalTok{(countries_sp, }\DataTypeTok{max.level =} \DecValTok{2}\NormalTok{)}
\end{Highlighting}
\end{Shaded}

\begin{verbatim}
## Formal class 'SpatialPolygons' [package "sp"] with 4 slots
##   ..@ polygons   :List of 177
##   ..@ plotOrder  : int [1:177] 7 136 28 169 31 23 9 66 84 5 ...
##   ..@ bbox       : num [1:2, 1:2] -180 -90 180 83.6
##   .. ..- attr(*, "dimnames")=List of 2
##   ..@ proj4string:Formal class 'CRS' [package "sp"] with 1 slot
\end{verbatim}

A more complicated spatial object You probably noticed something a
little different about the structure of countries\_sp. It looked a lot
like a list, but instead of the elements being proceeded by \$ in the
output they were instead proceeded by an @. This is because the sp
classes are S4 objects, so instead of having elements they have slots
and you access them with @. You'll learn more about this in the next
video.

Right now, let's take a look at another object countries\_spdf. It's a
little more complicated than countries\_sp, but you are now
well-equipped to figure out how this object differs.

Take a look!

Instructions 100 XP Call summary() on countries\_sp and then on this new
object countries\_spdf (one at a time). What kind of object is this?
What differs between this and countries\_sp? Call str() with max.level =
2 on countries\_spdf. How does the structure differ from countries\_sp?
Call plot() on countries\_spdf.

\begin{Shaded}
\begin{Highlighting}[]
\KeywordTok{load}\NormalTok{(}\DataTypeTok{file =} \StringTok{"countries_spdf.rda"}\NormalTok{)}

\CommentTok{# Call summary() on countries_spdf and countries_sp}
\KeywordTok{summary}\NormalTok{(countries_spdf)}
\end{Highlighting}
\end{Shaded}

\begin{verbatim}
## Object of class SpatialPolygonsDataFrame
## Coordinates:
##         min       max
## x -180.0000 180.00000
## y  -89.9999  83.64513
## Is projected: FALSE 
## proj4string :
## [+proj=longlat +datum=WGS84 +no_defs +ellps=WGS84 +towgs84=0,0,0]
## Data attributes:
##      name              iso_a3            population       
##  Length:177         Length:177         Min.   :1.400e+02  
##  Class :character   Class :character   1st Qu.:3.481e+06  
##  Mode  :character   Mode  :character   Median :9.048e+06  
##                                        Mean   :3.849e+07  
##                                        3rd Qu.:2.616e+07  
##                                        Max.   :1.339e+09  
##                                        NA's   :1          
##       gdp              region           subregion        
##  Min.   :      16   Length:177         Length:177        
##  1st Qu.:   13198   Class :character   Class :character  
##  Median :   43450   Mode  :character   Mode  :character  
##  Mean   :  395513                                        
##  3rd Qu.:  235100                                        
##  Max.   :15094000                                        
##  NA's   :1
\end{verbatim}

\begin{Shaded}
\begin{Highlighting}[]
\KeywordTok{summary}\NormalTok{(countries_sp)}
\end{Highlighting}
\end{Shaded}

\begin{verbatim}
## Object of class SpatialPolygons
## Coordinates:
##         min       max
## x -180.0000 180.00000
## y  -89.9999  83.64513
## Is projected: FALSE 
## proj4string :
## [+proj=longlat +datum=WGS84 +no_defs +ellps=WGS84 +towgs84=0,0,0]
\end{verbatim}

\begin{Shaded}
\begin{Highlighting}[]
\CommentTok{# Call str() with max.level = 2 on countries_spdf}
\KeywordTok{str}\NormalTok{(countries_spdf, }\DataTypeTok{max.level =} \DecValTok{2}\NormalTok{)}
\end{Highlighting}
\end{Shaded}

\begin{verbatim}
## Formal class 'SpatialPolygonsDataFrame' [package "sp"] with 5 slots
##   ..@ data       :'data.frame':  177 obs. of  6 variables:
##   ..@ polygons   :List of 177
##   ..@ plotOrder  : int [1:177] 7 136 28 169 31 23 9 66 84 5 ...
##   ..@ bbox       : num [1:2, 1:2] -180 -90 180 83.6
##   .. ..- attr(*, "dimnames")=List of 2
##   ..@ proj4string:Formal class 'CRS' [package "sp"] with 1 slot
\end{verbatim}

\begin{Shaded}
\begin{Highlighting}[]
\CommentTok{# Plot countries_spdf}
\KeywordTok{plot}\NormalTok{(countries_spdf)}
\end{Highlighting}
\end{Shaded}

\includegraphics{Geospacial-Data_files/figure-latex/mcso-1.pdf}

Walking the hierarchy Let's practice accessing slots by exploring the
way polygons are stored inside SpatialDataFrame objects. Remember there
are two ways to access slots in an S4 object:

\href{mailto:x@slot_name}{\nolinkurl{x@slot\_name}} \# or\ldots{}
slot(x, ``slot\_name'') So, to take a look at the polygons slot of
countries\_spdf you simply do
\href{mailto:countries_spdf@polygons}{\nolinkurl{countries\_spdf@polygons}}.
You can try it, but you'll get a long and not very informative output.
Let's look at the high level structure instead.

Try running the following code in the console:

str(\href{mailto:countries_spdf@polygons}{\nolinkurl{countries\_spdf@polygons}},
max.level = 2) Still a pretty long output, but scroll back to the top
and take a look. What kind of object is this? It's just a list, but
inside its elements are another kind of sp class: Polygons. There are
177 list elements. Any guesses what they might represent?

Let's dig into one of these elements.

Instructions 100 XP Create a new variable called one that contains the
169th element of the list in the polygons slot of countries\_spdf. Use
double bracket subsetting (i.e.~{[}{[}\ldots{]}{]} to extract this
element. Print one. Call summary() on one. What slots does this object
have? Call str() on one with max.level = 2.

\begin{Shaded}
\begin{Highlighting}[]
\CommentTok{# 169th element of countries_spdf@polygons: one}
\NormalTok{one <-}\StringTok{ }\NormalTok{countries_spdf}\OperatorTok{@}\NormalTok{polygons[[}\DecValTok{169}\NormalTok{]]}

\CommentTok{# Print one}
\KeywordTok{print}\NormalTok{(one)}
\end{Highlighting}
\end{Shaded}

\begin{verbatim}
## An object of class "Polygons"
## Slot "Polygons":
## [[1]]
## An object of class "Polygon"
## Slot "labpt":
## [1] -155.52045   19.60068
## 
## Slot "area":
## [1] 0.9638563
## 
## Slot "hole":
## [1] FALSE
## 
## Slot "ringDir":
## [1] 1
## 
## Slot "coords":
##               x        y
##  [1,] -155.5421 19.08348
##  [2,] -155.6882 18.91619
##  [3,] -155.9366 19.05939
##  [4,] -155.9081 19.33888
##  [5,] -156.0735 19.70294
##  [6,] -156.0237 19.81422
##  [7,] -155.8501 19.97729
##  [8,] -155.9191 20.17395
##  [9,] -155.8611 20.26721
## [10,] -155.7851 20.24870
## [11,] -155.4021 20.07975
## [12,] -155.2245 19.99302
## [13,] -155.0623 19.85910
## [14,] -154.8074 19.50871
## [15,] -154.8315 19.45328
## [16,] -155.2222 19.23972
## [17,] -155.5421 19.08348
## 
## 
## [[2]]
## An object of class "Polygon"
## Slot "labpt":
## [1] -156.36501   20.79145
## 
## Slot "area":
## [1] 0.1757331
## 
## Slot "hole":
## [1] FALSE
## 
## Slot "ringDir":
## [1] 1
## 
## Slot "coords":
##               x        y
##  [1,] -156.0793 20.64397
##  [2,] -156.4144 20.57241
##  [3,] -156.5867 20.78300
##  [4,] -156.7017 20.86430
##  [5,] -156.7106 20.92676
##  [6,] -156.6126 21.01249
##  [7,] -156.2571 20.91745
##  [8,] -155.9957 20.76404
##  [9,] -156.0793 20.64397
## 
## 
## [[3]]
## An object of class "Polygon"
## Slot "labpt":
## [1] -157.03713   21.14047
## 
## Slot "area":
## [1] 0.06098698
## 
## Slot "hole":
## [1] FALSE
## 
## Slot "ringDir":
## [1] 1
## 
## Slot "coords":
##              x        y
## [1,] -156.7582 21.17684
## [2,] -156.7893 21.06873
## [3,] -157.3252 21.09777
## [4,] -157.2503 21.21958
## [5,] -156.7582 21.17684
## 
## 
## [[4]]
## An object of class "Polygon"
## Slot "labpt":
## [1] -157.98482   21.46727
## 
## Slot "area":
## [1] 0.1575561
## 
## Slot "hole":
## [1] FALSE
## 
## Slot "ringDir":
## [1] 1
## 
## Slot "coords":
##               x        y
##  [1,] -157.6528 21.32217
##  [2,] -157.7070 21.26442
##  [3,] -157.7786 21.27729
##  [4,] -158.1267 21.31244
##  [5,] -158.2538 21.53919
##  [6,] -158.2927 21.57912
##  [7,] -158.0252 21.71696
##  [8,] -157.9416 21.65272
##  [9,] -157.6528 21.32217
## 
## 
## [[5]]
## An object of class "Polygon"
## Slot "labpt":
## [1] -159.53598   22.07776
## 
## Slot "area":
## [1] 0.104553
## 
## Slot "hole":
## [1] FALSE
## 
## Slot "ringDir":
## [1] 1
## 
## Slot "coords":
##              x        y
## [1,] -159.3451 21.98200
## [2,] -159.4637 21.88299
## [3,] -159.8005 22.06533
## [4,] -159.7488 22.13820
## [5,] -159.5962 22.23618
## [6,] -159.3657 22.21494
## [7,] -159.3451 21.98200
## 
## 
## [[6]]
## An object of class "Polygon"
## Slot "labpt":
## [1] -99.06024  39.50155
## 
## Slot "area":
## [1] 839.7635
## 
## Slot "hole":
## [1] FALSE
## 
## Slot "ringDir":
## [1] 1
## 
## Slot "coords":
##                 x        y
##   [1,]  -94.81758 49.38905
##   [2,]  -94.64000 48.84000
##   [3,]  -94.32914 48.67074
##   [4,]  -93.63087 48.60926
##   [5,]  -92.61000 48.45000
##   [6,]  -91.64000 48.14000
##   [7,]  -90.83000 48.27000
##   [8,]  -89.60000 48.01000
##   [9,]  -89.27292 48.01981
##  [10,]  -88.37811 48.30292
##  [11,]  -87.43979 47.94000
##  [12,]  -86.46199 47.55334
##  [13,]  -85.65236 47.22022
##  [14,]  -84.87608 46.90008
##  [15,]  -84.77924 46.63710
##  [16,]  -84.54375 46.53868
##  [17,]  -84.60490 46.43960
##  [18,]  -84.33670 46.40877
##  [19,]  -84.14212 46.51223
##  [20,]  -84.09185 46.27542
##  [21,]  -83.89077 46.11693
##  [22,]  -83.61613 46.11693
##  [23,]  -83.46955 45.99469
##  [24,]  -83.59285 45.81689
##  [25,]  -82.55092 45.34752
##  [26,]  -82.33776 44.44000
##  [27,]  -82.13764 43.57109
##  [28,]  -82.43000 42.98000
##  [29,]  -82.90000 42.43000
##  [30,]  -83.12000 42.08000
##  [31,]  -83.14200 41.97568
##  [32,]  -83.02981 41.83280
##  [33,]  -82.69009 41.67511
##  [34,]  -82.43928 41.67511
##  [35,]  -81.27775 42.20903
##  [36,]  -80.24745 42.36620
##  [37,]  -78.93936 42.86361
##  [38,]  -78.92000 42.96500
##  [39,]  -79.01000 43.27000
##  [40,]  -79.17167 43.46634
##  [41,]  -78.72028 43.62509
##  [42,]  -77.73789 43.62906
##  [43,]  -76.82003 43.62878
##  [44,]  -76.50000 44.01846
##  [45,]  -76.37500 44.09631
##  [46,]  -75.31821 44.81645
##  [47,]  -74.86700 45.00048
##  [48,]  -73.34783 45.00738
##  [49,]  -71.50506 45.00820
##  [50,]  -71.40500 45.25500
##  [51,]  -71.08482 45.30524
##  [52,]  -70.66000 45.46000
##  [53,]  -70.30500 45.91500
##  [54,]  -69.99997 46.69307
##  [55,]  -69.23722 47.44778
##  [56,]  -68.90500 47.18500
##  [57,]  -68.23444 47.35486
##  [58,]  -67.79046 47.06636
##  [59,]  -67.79134 45.70281
##  [60,]  -67.13741 45.13753
##  [61,]  -66.96466 44.80970
##  [62,]  -68.03252 44.32520
##  [63,]  -69.06000 43.98000
##  [64,]  -70.11617 43.68405
##  [65,]  -70.64548 43.09024
##  [66,]  -70.81489 42.86530
##  [67,]  -70.82500 42.33500
##  [68,]  -70.49500 41.80500
##  [69,]  -70.08000 41.78000
##  [70,]  -70.18500 42.14500
##  [71,]  -69.88497 41.92283
##  [72,]  -69.96503 41.63717
##  [73,]  -70.64000 41.47500
##  [74,]  -71.12039 41.49445
##  [75,]  -71.86000 41.32000
##  [76,]  -72.29500 41.27000
##  [77,]  -72.87643 41.22065
##  [78,]  -73.71000 40.93110
##  [79,]  -72.24126 41.11948
##  [80,]  -71.94500 40.93000
##  [81,]  -73.34500 40.63000
##  [82,]  -73.98200 40.62800
##  [83,]  -73.95233 40.75075
##  [84,]  -74.25671 40.47351
##  [85,]  -73.96244 40.42763
##  [86,]  -74.17838 39.70926
##  [87,]  -74.90604 38.93954
##  [88,]  -74.98041 39.19640
##  [89,]  -75.20002 39.24845
##  [90,]  -75.52805 39.49850
##  [91,]  -75.32000 38.96000
##  [92,]  -75.07183 38.78203
##  [93,]  -75.05673 38.40412
##  [94,]  -75.37747 38.01551
##  [95,]  -75.94023 37.21689
##  [96,]  -76.03127 37.25660
##  [97,]  -75.72205 37.93705
##  [98,]  -76.23287 38.31921
##  [99,]  -76.35000 39.15000
## [100,]  -76.54273 38.71762
## [101,]  -76.32933 38.08326
## [102,]  -76.99000 38.23999
## [103,]  -76.30162 37.91795
## [104,]  -76.25874 36.96640
## [105,]  -75.97180 36.89726
## [106,]  -75.86804 36.55125
## [107,]  -75.72749 35.55074
## [108,]  -76.36318 34.80854
## [109,]  -77.39763 34.51201
## [110,]  -78.05496 33.92547
## [111,]  -78.55435 33.86133
## [112,]  -79.06067 33.49395
## [113,]  -79.20357 33.15839
## [114,]  -80.30133 32.50935
## [115,]  -80.86498 32.03330
## [116,]  -81.33629 31.44049
## [117,]  -81.49042 30.72999
## [118,]  -81.31371 30.03552
## [119,]  -80.98000 29.18000
## [120,]  -80.53558 28.47213
## [121,]  -80.53000 28.04000
## [122,]  -80.05654 26.88000
## [123,]  -80.08801 26.20576
## [124,]  -80.13156 25.81677
## [125,]  -80.38103 25.20616
## [126,]  -80.68000 25.08000
## [127,]  -81.17213 25.20126
## [128,]  -81.33000 25.64000
## [129,]  -81.71000 25.87000
## [130,]  -82.24000 26.73000
## [131,]  -82.70515 27.49504
## [132,]  -82.85526 27.88624
## [133,]  -82.65000 28.55000
## [134,]  -82.93000 29.10000
## [135,]  -83.70959 29.93656
## [136,]  -84.10000 30.09000
## [137,]  -85.10882 29.63615
## [138,]  -85.28784 29.68612
## [139,]  -85.77310 30.15261
## [140,]  -86.40000 30.40000
## [141,]  -87.53036 30.27433
## [142,]  -88.41782 30.38490
## [143,]  -89.18049 30.31598
## [144,]  -89.59383 30.15999
## [145,]  -89.41374 29.89419
## [146,]  -89.43000 29.48864
## [147,]  -89.21767 29.29108
## [148,]  -89.40823 29.15961
## [149,]  -89.77928 29.30714
## [150,]  -90.15463 29.11743
## [151,]  -90.88022 29.14853
## [152,]  -91.62678 29.67700
## [153,]  -92.49906 29.55230
## [154,]  -93.22637 29.78375
## [155,]  -93.84842 29.71363
## [156,]  -94.69000 29.48000
## [157,]  -95.60026 28.73863
## [158,]  -96.59404 28.30748
## [159,]  -97.14000 27.83000
## [160,]  -97.37000 27.38000
## [161,]  -97.38000 26.69000
## [162,]  -97.33000 26.21000
## [163,]  -97.14000 25.87000
## [164,]  -97.53000 25.84000
## [165,]  -98.24000 26.06000
## [166,]  -99.02000 26.37000
## [167,]  -99.30000 26.84000
## [168,]  -99.52000 27.54000
## [169,] -100.11000 28.11000
## [170,] -100.45584 28.69612
## [171,] -100.95760 29.38071
## [172,] -101.66240 29.77930
## [173,] -102.48000 29.76000
## [174,] -103.11000 28.97000
## [175,] -103.94000 29.27000
## [176,] -104.45697 29.57196
## [177,] -104.70575 30.12173
## [178,] -105.03737 30.64402
## [179,] -105.63159 31.08383
## [180,] -106.14290 31.39995
## [181,] -106.50759 31.75452
## [182,] -108.24000 31.75485
## [183,] -108.24194 31.34222
## [184,] -109.03500 31.34194
## [185,] -111.02361 31.33472
## [186,] -113.30498 32.03914
## [187,] -114.81500 32.52528
## [188,] -114.72139 32.72083
## [189,] -115.99135 32.61239
## [190,] -117.12776 32.53534
## [191,] -117.29594 33.04622
## [192,] -117.94400 33.62124
## [193,] -118.41060 33.74091
## [194,] -118.51989 34.02778
## [195,] -119.08100 34.07800
## [196,] -119.43884 34.34848
## [197,] -120.36778 34.44711
## [198,] -120.62286 34.60855
## [199,] -120.74433 35.15686
## [200,] -121.71457 36.16153
## [201,] -122.54747 37.55176
## [202,] -122.51201 37.78339
## [203,] -122.95319 38.11371
## [204,] -123.72720 38.95166
## [205,] -123.86517 39.76699
## [206,] -124.39807 40.31320
## [207,] -124.17886 41.14202
## [208,] -124.21370 41.99964
## [209,] -124.53284 42.76599
## [210,] -124.14214 43.70838
## [211,] -124.02053 44.61590
## [212,] -123.89893 45.52341
## [213,] -124.07963 46.86475
## [214,] -124.39567 47.72017
## [215,] -124.68721 48.18443
## [216,] -124.56610 48.37971
## [217,] -123.12000 48.04000
## [218,] -122.58736 47.09600
## [219,] -122.34000 47.36000
## [220,] -122.50000 48.18000
## [221,] -122.84000 49.00000
## [222,] -120.00000 49.00000
## [223,] -117.03121 49.00000
## [224,] -116.04818 49.00000
## [225,] -113.00000 49.00000
## [226,] -110.05000 49.00000
## [227,] -107.05000 49.00000
## [228,] -104.04826 48.99986
## [229,] -100.65000 49.00000
## [230,]  -97.22872 49.00070
## [231,]  -95.15907 49.00000
## [232,]  -95.15609 49.38425
## [233,]  -94.81758 49.38905
## 
## 
## [[7]]
## An object of class "Polygon"
## Slot "labpt":
## [1] -153.50496   57.40668
## 
## Slot "area":
## [1] 1.797537
## 
## Slot "hole":
## [1] FALSE
## 
## Slot "ringDir":
## [1] 1
## 
## Slot "coords":
##               x        y
##  [1,] -153.0063 57.11584
##  [2,] -154.0051 56.73468
##  [3,] -154.5164 56.99275
##  [4,] -154.6710 57.46120
##  [5,] -153.7628 57.81657
##  [6,] -153.2287 57.96897
##  [7,] -152.5648 57.90143
##  [8,] -152.1411 57.59106
##  [9,] -153.0063 57.11584
## 
## 
## [[8]]
## An object of class "Polygon"
## Slot "labpt":
## [1] -166.36503   60.09691
## 
## Slot "area":
## [1] 0.7292136
## 
## Slot "hole":
## [1] FALSE
## 
## Slot "ringDir":
## [1] 1
## 
## Slot "coords":
##              x        y
## [1,] -165.5792 59.90999
## [2,] -166.1928 59.75444
## [3,] -166.8483 59.94141
## [4,] -167.4553 60.21307
## [5,] -166.4678 60.38417
## [6,] -165.6744 60.29361
## [7,] -165.5792 59.90999
## 
## 
## [[9]]
## An object of class "Polygon"
## Slot "labpt":
## [1] -170.30120   63.37744
## 
## Slot "area":
## [1] 1.029355
## 
## Slot "hole":
## [1] FALSE
## 
## Slot "ringDir":
## [1] 1
## 
## Slot "coords":
##               x        y
##  [1,] -171.7317 63.78252
##  [2,] -171.1144 63.59219
##  [3,] -170.4911 63.69498
##  [4,] -169.6825 63.43112
##  [5,] -168.6894 63.29751
##  [6,] -168.7719 63.18860
##  [7,] -169.5294 62.97693
##  [8,] -170.2906 63.19444
##  [9,] -170.6714 63.37582
## [10,] -171.5531 63.31779
## [11,] -171.7911 63.40585
## [12,] -171.7317 63.78252
## 
## 
## [[10]]
## An object of class "Polygon"
## Slot "labpt":
## [1] -152.72115   64.43561
## 
## Slot "area":
## [1] 277.4997
## 
## Slot "hole":
## [1] FALSE
## 
## Slot "ringDir":
## [1] 1
## 
## Slot "coords":
##                x        y
##   [1,] -155.0678 71.14778
##   [2,] -154.3442 70.69641
##   [3,] -153.9000 70.88999
##   [4,] -152.2100 70.82999
##   [5,] -152.2700 70.60001
##   [6,] -150.7400 70.43002
##   [7,] -149.7200 70.53001
##   [8,] -147.6134 70.21403
##   [9,] -145.6900 70.12001
##  [10,] -144.9200 69.98999
##  [11,] -143.5894 70.15251
##  [12,] -142.0725 69.85194
##  [13,] -140.9860 69.71200
##  [14,] -140.9860 69.71200
##  [15,] -140.9925 66.00003
##  [16,] -140.9978 60.30640
##  [17,] -140.0130 60.27684
##  [18,] -139.0390 60.00001
##  [19,] -138.3409 59.56211
##  [20,] -137.4525 58.90500
##  [21,] -136.4797 59.46389
##  [22,] -135.4758 59.78778
##  [23,] -134.9450 59.27056
##  [24,] -134.2711 58.86111
##  [25,] -133.3555 58.41029
##  [26,] -132.7304 57.69289
##  [27,] -131.7078 56.55212
##  [28,] -130.0078 55.91583
##  [29,] -129.9800 55.28500
##  [30,] -130.5361 54.80275
##  [31,] -131.0858 55.17891
##  [32,] -131.9672 55.49778
##  [33,] -132.2500 56.37000
##  [34,] -133.5392 57.17889
##  [35,] -134.0781 58.12307
##  [36,] -135.0382 58.18771
##  [37,] -136.6281 58.21221
##  [38,] -137.8000 58.50000
##  [39,] -139.8678 59.53776
##  [40,] -140.8253 59.72752
##  [41,] -142.5744 60.08445
##  [42,] -143.9589 59.99918
##  [43,] -145.9256 60.45861
##  [44,] -147.1144 60.88466
##  [45,] -148.2243 60.67299
##  [46,] -148.0181 59.97833
##  [47,] -148.5708 59.91417
##  [48,] -149.7279 59.70566
##  [49,] -150.6082 59.36821
##  [50,] -151.7164 59.15582
##  [51,] -151.8594 59.74498
##  [52,] -151.4097 60.72580
##  [53,] -150.3469 61.03359
##  [54,] -150.6211 61.28442
##  [55,] -151.8958 60.72720
##  [56,] -152.5783 60.06166
##  [57,] -154.0192 59.35028
##  [58,] -153.2875 58.86473
##  [59,] -154.2325 58.14637
##  [60,] -155.3075 57.72779
##  [61,] -156.3083 57.42277
##  [62,] -156.5561 56.97998
##  [63,] -158.1172 56.46361
##  [64,] -158.4333 55.99415
##  [65,] -159.6033 55.56669
##  [66,] -160.2897 55.64358
##  [67,] -161.2230 55.36473
##  [68,] -162.2378 55.02419
##  [69,] -163.0694 54.68974
##  [70,] -164.7856 54.40417
##  [71,] -164.9422 54.57222
##  [72,] -163.8483 55.03943
##  [73,] -162.8700 55.34804
##  [74,] -161.8042 55.89499
##  [75,] -160.5636 56.00805
##  [76,] -160.0706 56.41806
##  [77,] -158.6844 57.01668
##  [78,] -158.4611 57.21692
##  [79,] -157.7228 57.57000
##  [80,] -157.5503 58.32833
##  [81,] -157.0417 58.91888
##  [82,] -158.1947 58.61580
##  [83,] -158.5172 58.78778
##  [84,] -159.0586 58.42419
##  [85,] -159.7117 58.93139
##  [86,] -159.9813 58.57255
##  [87,] -160.3553 59.07112
##  [88,] -161.3550 58.67084
##  [89,] -161.9689 58.67166
##  [90,] -162.0550 59.26693
##  [91,] -161.8742 59.63362
##  [92,] -162.5181 59.98972
##  [93,] -163.8183 59.79806
##  [94,] -164.6622 60.26748
##  [95,] -165.3464 60.50750
##  [96,] -165.3508 61.07390
##  [97,] -166.1214 61.50002
##  [98,] -165.7345 62.07500
##  [99,] -164.9192 62.63308
## [100,] -164.5625 63.14638
## [101,] -163.7533 63.21945
## [102,] -163.0672 63.05946
## [103,] -162.2606 63.54194
## [104,] -161.5344 63.45582
## [105,] -160.7725 63.76611
## [106,] -160.9583 64.22280
## [107,] -161.5181 64.40279
## [108,] -160.7778 64.78860
## [109,] -161.3919 64.77724
## [110,] -162.4531 64.55944
## [111,] -162.7578 64.33861
## [112,] -163.5464 64.55916
## [113,] -164.9608 64.44695
## [114,] -166.4253 64.68667
## [115,] -166.8450 65.08890
## [116,] -168.1106 65.67000
## [117,] -166.7053 66.08832
## [118,] -164.4747 66.57666
## [119,] -163.6525 66.57666
## [120,] -163.7886 66.07721
## [121,] -161.6778 66.11612
## [122,] -162.4897 66.73557
## [123,] -163.7197 67.11639
## [124,] -164.4310 67.61634
## [125,] -165.3903 68.04277
## [126,] -166.7644 68.35888
## [127,] -166.2047 68.88303
## [128,] -164.4308 68.91554
## [129,] -163.1686 69.37111
## [130,] -162.9306 69.85806
## [131,] -161.9089 70.33333
## [132,] -160.9348 70.44769
## [133,] -159.0392 70.89164
## [134,] -158.1197 70.82472
## [135,] -156.5808 71.35776
## [136,] -155.0678 71.14778
## 
## 
## 
## Slot "plotOrder":
##  [1]  6 10  7  9  1  8  2  4  5  3
## 
## Slot "labpt":
## [1] -99.06024  39.50155
## 
## Slot "ID":
## [1] "168"
## 
## Slot "area":
## [1] 1122.282
\end{verbatim}

\begin{Shaded}
\begin{Highlighting}[]
\CommentTok{# Call summary() on one}
\KeywordTok{summary}\NormalTok{(one)}
\end{Highlighting}
\end{Shaded}

\begin{verbatim}
##   Length    Class     Mode 
##        1 Polygons       S4
\end{verbatim}

\begin{Shaded}
\begin{Highlighting}[]
\CommentTok{# Call str() on one with max.level = 2}
\KeywordTok{str}\NormalTok{(one, }\DataTypeTok{max.level =} \DecValTok{2}\NormalTok{)}
\end{Highlighting}
\end{Shaded}

\begin{verbatim}
## Formal class 'Polygons' [package "sp"] with 5 slots
##   ..@ Polygons :List of 10
##   ..@ plotOrder: int [1:10] 6 10 7 9 1 8 2 4 5 3
##   ..@ labpt    : num [1:2] -99.1 39.5
##   ..@ ID       : chr "168"
##   ..@ area     : num 1122
\end{verbatim}

Further down the rabbit hole In the last exercise, the
SpatialPolygonsDataFrame had a list of Polygons in its polygons slot,
and each of those Polygons objects also had a Polygons slot. So, many
polygons\ldots but you aren't at the bottom of the hierarchy yet!

Let's take another look at the 169th element in the Polygons slot of
countries\_spdf. Run this code from the previous exercise:

one \textless-
\href{mailto:countries_spdf@polygons}{\nolinkurl{countries\_spdf@polygons}}{[}{[}169{]}{]}
str(one, max.level = 2) The Polygons slot has a list inside with 10
elements. What are these objects? Let's keep digging\ldots.

Instructions 100 XP Call str() with max.level = 2 on the Polygons slot
of one. Call str() with max.level = 2 on the 6th element of the Polygons
slot of one. Do you see something that looks like it might be spatial
data? Call plot() on the coords slot of the 6th element of the Polygons
slot of one. Do you recognise what data this object contains?

\begin{Shaded}
\begin{Highlighting}[]
\NormalTok{one <-}\StringTok{ }\NormalTok{countries_spdf}\OperatorTok{@}\NormalTok{polygons[[}\DecValTok{169}\NormalTok{]]}

\CommentTok{# str() with max.level = 2, on the Polygons slot of one}
\KeywordTok{str}\NormalTok{(one}\OperatorTok{@}\NormalTok{Polygons, }\DataTypeTok{max.level =} \DecValTok{2}\NormalTok{)}
\end{Highlighting}
\end{Shaded}

\begin{verbatim}
## List of 10
##  $ :Formal class 'Polygon' [package "sp"] with 5 slots
##  $ :Formal class 'Polygon' [package "sp"] with 5 slots
##  $ :Formal class 'Polygon' [package "sp"] with 5 slots
##  $ :Formal class 'Polygon' [package "sp"] with 5 slots
##  $ :Formal class 'Polygon' [package "sp"] with 5 slots
##  $ :Formal class 'Polygon' [package "sp"] with 5 slots
##  $ :Formal class 'Polygon' [package "sp"] with 5 slots
##  $ :Formal class 'Polygon' [package "sp"] with 5 slots
##  $ :Formal class 'Polygon' [package "sp"] with 5 slots
##  $ :Formal class 'Polygon' [package "sp"] with 5 slots
\end{verbatim}

\begin{Shaded}
\begin{Highlighting}[]
\CommentTok{# str() with max.level = 2, on the 6th element of the one@Polygons}
\KeywordTok{str}\NormalTok{(one}\OperatorTok{@}\NormalTok{Polygons[[}\DecValTok{6}\NormalTok{]], }\DataTypeTok{max.level =} \DecValTok{2}\NormalTok{)}
\end{Highlighting}
\end{Shaded}

\begin{verbatim}
## Formal class 'Polygon' [package "sp"] with 5 slots
##   ..@ labpt  : num [1:2] -99.1 39.5
##   ..@ area   : num 840
##   ..@ hole   : logi FALSE
##   ..@ ringDir: int 1
##   ..@ coords : num [1:233, 1:2] -94.8 -94.6 -94.3 -93.6 -92.6 ...
##   .. ..- attr(*, "dimnames")=List of 2
\end{verbatim}

\begin{Shaded}
\begin{Highlighting}[]
\CommentTok{# Call plot on the coords slot of 6th element of one@Polygons}
\KeywordTok{plot}\NormalTok{(one}\OperatorTok{@}\NormalTok{Polygons[[}\DecValTok{6}\NormalTok{]]}\OperatorTok{@}\NormalTok{coords)}
\end{Highlighting}
\end{Shaded}

\includegraphics{Geospacial-Data_files/figure-latex/fdrh-1.pdf}

Subsetting by index The subsetting of Spatial\_\textbf{DataFrame objects
is built to work like subsetting a data frame. You think about
subsetting the data frame, but in practice what is returned is a new
Spatial}\_DataFrame with only the rows of data you want and the
corresponding spatial objects.

The simplest kind of subsetting is by index. For example, if x is a data
frame you know x{[}1, {]} returns the first row. If x is a
Spatial\_\textbf{DataFrame, you get a new Spatial}\_DataFrame that
contains the first row of data and the spatial data that correspond to
that row.

The benefit of returning a Spatial\_\_\_DataFrame is you can use all the
same methods as on the object before subsetting.

Let's test it out on the 169th country!

Instructions 100 XP Create a new variable usa by subsetting the 169th
element of countries\_spdf. Call summary() on usa. Verify usa is still a
SpatialPolygonsDataFrame. Call str() with max.level = 2 on usa. Verify
there is only one element of the polygons slot and only one row in the
data slot. Call plot() on usa.

\begin{Shaded}
\begin{Highlighting}[]
\CommentTok{# Subset the 169th object of countries_spdf: usa}
\NormalTok{usa <-}\StringTok{ }\NormalTok{countries_spdf[}\DecValTok{169}\NormalTok{,]}

\CommentTok{# Look at summary() of usa}
\KeywordTok{summary}\NormalTok{(usa)}
\end{Highlighting}
\end{Shaded}

\begin{verbatim}
## Object of class SpatialPolygonsDataFrame
## Coordinates:
##          min       max
## x -171.79111 -66.96466
## y   18.91619  71.35776
## Is projected: FALSE 
## proj4string :
## [+proj=longlat +datum=WGS84 +no_defs +ellps=WGS84 +towgs84=0,0,0]
## Data attributes:
##      name              iso_a3            population      
##  Length:1           Length:1           Min.   :3.14e+08  
##  Class :character   Class :character   1st Qu.:3.14e+08  
##  Mode  :character   Mode  :character   Median :3.14e+08  
##                                        Mean   :3.14e+08  
##                                        3rd Qu.:3.14e+08  
##                                        Max.   :3.14e+08  
##       gdp              region           subregion        
##  Min.   :15094000   Length:1           Length:1          
##  1st Qu.:15094000   Class :character   Class :character  
##  Median :15094000   Mode  :character   Mode  :character  
##  Mean   :15094000                                        
##  3rd Qu.:15094000                                        
##  Max.   :15094000
\end{verbatim}

\begin{Shaded}
\begin{Highlighting}[]
\CommentTok{# Look at str() of usa}
\KeywordTok{str}\NormalTok{(usa, }\DataTypeTok{max.level =} \DecValTok{2}\NormalTok{)}
\end{Highlighting}
\end{Shaded}

\begin{verbatim}
## Formal class 'SpatialPolygonsDataFrame' [package "sp"] with 5 slots
##   ..@ data       :'data.frame':  1 obs. of  6 variables:
##   ..@ polygons   :List of 1
##   ..@ plotOrder  : int 1
##   ..@ bbox       : num [1:2, 1:2] -171.8 18.9 -67 71.4
##   .. ..- attr(*, "dimnames")=List of 2
##   ..@ proj4string:Formal class 'CRS' [package "sp"] with 1 slot
\end{verbatim}

\begin{Shaded}
\begin{Highlighting}[]
\CommentTok{# Call plot() on usa}
\KeywordTok{plot}\NormalTok{(usa)}
\end{Highlighting}
\end{Shaded}

\includegraphics{Geospacial-Data_files/figure-latex/sbi-1.pdf}

Accessing data in sp objects It's quite unusual to know exactly the
indices of elements you want to keep, and far more likely you want to
subset based on data attributes. You've seen the data associated with a
Spatial\_\_\_DataFrame lives in the data slot, but you don't normally
access this slot directly.

Instead,\$ and {[}{[} subsetting on a Spatial\_\textbf{DataFrame pulls
columns directly from the data frame. That is, if x is a
Spatial}\_DataFrame object, then either
x\(col_name or x[["col_name"]] pulls out the col_name column from the data frame. Think of this like a shortcut; instead of having to pull the right column from the object in the data slot (i.e. x@data\)col\_name),
you can just use x\$col\_name.

Let's start by confirming the object in the data slot is just a regular
data frame, then practice pulling out columns.

Instructions 100 XP Call head() and str() (one at a time) on the data
slot of countries\_spdf. Verify that this object is just a regular data
frame. Pull out the name column of countries\_spdf using \$. Pull out
the subregion column of countries\_spdf using {[}{[}.

\begin{Shaded}
\begin{Highlighting}[]
\CommentTok{# Call head() and str() on the data slot of countries_spdf}
\KeywordTok{head}\NormalTok{(countries_spdf}\OperatorTok{@}\NormalTok{data)}
\end{Highlighting}
\end{Shaded}

\begin{verbatim}
##                   name iso_a3 population    gdp   region       subregion
## 0          Afghanistan    AFG   28400000  22270     Asia   Southern Asia
## 1               Angola    AGO   12799293 110300   Africa   Middle Africa
## 2              Albania    ALB    3639453  21810   Europe Southern Europe
## 3 United Arab Emirates    ARE    4798491 184300     Asia    Western Asia
## 4            Argentina    ARG   40913584 573900 Americas   South America
## 5              Armenia    ARM    2967004  18770     Asia    Western Asia
\end{verbatim}

\begin{Shaded}
\begin{Highlighting}[]
\KeywordTok{str}\NormalTok{(countries_spdf}\OperatorTok{@}\NormalTok{data)}
\end{Highlighting}
\end{Shaded}

\begin{verbatim}
## 'data.frame':    177 obs. of  6 variables:
##  $ name      : chr  "Afghanistan" "Angola" "Albania" "United Arab Emirates" ...
##  $ iso_a3    : chr  "AFG" "AGO" "ALB" "ARE" ...
##  $ population: num  28400000 12799293 3639453 4798491 40913584 ...
##  $ gdp       : num  22270 110300 21810 184300 573900 ...
##  $ region    : chr  "Asia" "Africa" "Europe" "Asia" ...
##  $ subregion : chr  "Southern Asia" "Middle Africa" "Southern Europe" "Western Asia" ...
\end{verbatim}

\begin{Shaded}
\begin{Highlighting}[]
\CommentTok{# Pull out the name column using $}
\NormalTok{countries_spdf}\OperatorTok{$}\NormalTok{name}
\end{Highlighting}
\end{Shaded}

\begin{verbatim}
##   [1] "Afghanistan"            "Angola"                
##   [3] "Albania"                "United Arab Emirates"  
##   [5] "Argentina"              "Armenia"               
##   [7] "Antarctica"             "Fr. S. Antarctic Lands"
##   [9] "Australia"              "Austria"               
##  [11] "Azerbaijan"             "Burundi"               
##  [13] "Belgium"                "Benin"                 
##  [15] "Burkina Faso"           "Bangladesh"            
##  [17] "Bulgaria"               "Bahamas"               
##  [19] "Bosnia and Herz."       "Belarus"               
##  [21] "Belize"                 "Bolivia"               
##  [23] "Brazil"                 "Brunei"                
##  [25] "Bhutan"                 "Botswana"              
##  [27] "Central African Rep."   "Canada"                
##  [29] "Switzerland"            "Chile"                 
##  [31] "China"                  "Cte d'Ivoire"          
##  [33] "Cameroon"               "Dem. Rep. Congo"       
##  [35] "Congo"                  "Colombia"              
##  [37] "Costa Rica"             "Cuba"                  
##  [39] "N. Cyprus"              "Cyprus"                
##  [41] "Czech Rep."             "Germany"               
##  [43] "Djibouti"               "Denmark"               
##  [45] "Dominican Rep."         "Algeria"               
##  [47] "Ecuador"                "Egypt"                 
##  [49] "Eritrea"                "Spain"                 
##  [51] "Estonia"                "Ethiopia"              
##  [53] "Finland"                "Fiji"                  
##  [55] "Falkland Is."           "France"                
##  [57] "Gabon"                  "United Kingdom"        
##  [59] "Georgia"                "Ghana"                 
##  [61] "Guinea"                 "Gambia"                
##  [63] "Guinea-Bissau"          "Eq. Guinea"            
##  [65] "Greece"                 "Greenland"             
##  [67] "Guatemala"              "Guyana"                
##  [69] "Honduras"               "Croatia"               
##  [71] "Haiti"                  "Hungary"               
##  [73] "Indonesia"              "India"                 
##  [75] "Ireland"                "Iran"                  
##  [77] "Iraq"                   "Iceland"               
##  [79] "Israel"                 "Italy"                 
##  [81] "Jamaica"                "Jordan"                
##  [83] "Japan"                  "Kazakhstan"            
##  [85] "Kenya"                  "Kyrgyzstan"            
##  [87] "Cambodia"               "Korea"                 
##  [89] "Kosovo"                 "Kuwait"                
##  [91] "Lao PDR"                "Lebanon"               
##  [93] "Liberia"                "Libya"                 
##  [95] "Sri Lanka"              "Lesotho"               
##  [97] "Lithuania"              "Luxembourg"            
##  [99] "Latvia"                 "Morocco"               
## [101] "Moldova"                "Madagascar"            
## [103] "Mexico"                 "Macedonia"             
## [105] "Mali"                   "Myanmar"               
## [107] "Montenegro"             "Mongolia"              
## [109] "Mozambique"             "Mauritania"            
## [111] "Malawi"                 "Malaysia"              
## [113] "Namibia"                "New Caledonia"         
## [115] "Niger"                  "Nigeria"               
## [117] "Nicaragua"              "Netherlands"           
## [119] "Norway"                 "Nepal"                 
## [121] "New Zealand"            "Oman"                  
## [123] "Pakistan"               "Panama"                
## [125] "Peru"                   "Philippines"           
## [127] "Papua New Guinea"       "Poland"                
## [129] "Puerto Rico"            "Dem. Rep. Korea"       
## [131] "Portugal"               "Paraguay"              
## [133] "Palestine"              "Qatar"                 
## [135] "Romania"                "Russia"                
## [137] "Rwanda"                 "W. Sahara"             
## [139] "Saudi Arabia"           "Sudan"                 
## [141] "S. Sudan"               "Senegal"               
## [143] "Solomon Is."            "Sierra Leone"          
## [145] "El Salvador"            "Somaliland"            
## [147] "Somalia"                "Serbia"                
## [149] "Suriname"               "Slovakia"              
## [151] "Slovenia"               "Sweden"                
## [153] "Swaziland"              "Syria"                 
## [155] "Chad"                   "Togo"                  
## [157] "Thailand"               "Tajikistan"            
## [159] "Turkmenistan"           "Timor-Leste"           
## [161] "Trinidad and Tobago"    "Tunisia"               
## [163] "Turkey"                 "Taiwan"                
## [165] "Tanzania"               "Uganda"                
## [167] "Ukraine"                "Uruguay"               
## [169] "United States"          "Uzbekistan"            
## [171] "Venezuela"              "Vietnam"               
## [173] "Vanuatu"                "Yemen"                 
## [175] "South Africa"           "Zambia"                
## [177] "Zimbabwe"
\end{verbatim}

\begin{Shaded}
\begin{Highlighting}[]
\CommentTok{# Pull out the subregion column using [[}
\NormalTok{countries_spdf[[}\StringTok{"subregion"}\NormalTok{]]}
\end{Highlighting}
\end{Shaded}

\begin{verbatim}
##   [1] "Southern Asia"             "Middle Africa"            
##   [3] "Southern Europe"           "Western Asia"             
##   [5] "South America"             "Western Asia"             
##   [7] "Antarctica"                "Seven seas (open ocean)"  
##   [9] "Australia and New Zealand" "Western Europe"           
##  [11] "Western Asia"              "Eastern Africa"           
##  [13] "Western Europe"            "Western Africa"           
##  [15] "Western Africa"            "Southern Asia"            
##  [17] "Eastern Europe"            "Caribbean"                
##  [19] "Southern Europe"           "Eastern Europe"           
##  [21] "Central America"           "South America"            
##  [23] "South America"             "South-Eastern Asia"       
##  [25] "Southern Asia"             "Southern Africa"          
##  [27] "Middle Africa"             "Northern America"         
##  [29] "Western Europe"            "South America"            
##  [31] "Eastern Asia"              "Western Africa"           
##  [33] "Middle Africa"             "Middle Africa"            
##  [35] "Middle Africa"             "South America"            
##  [37] "Central America"           "Caribbean"                
##  [39] "Western Asia"              "Western Asia"             
##  [41] "Eastern Europe"            "Western Europe"           
##  [43] "Eastern Africa"            "Northern Europe"          
##  [45] "Caribbean"                 "Northern Africa"          
##  [47] "South America"             "Northern Africa"          
##  [49] "Eastern Africa"            "Southern Europe"          
##  [51] "Northern Europe"           "Eastern Africa"           
##  [53] "Northern Europe"           "Melanesia"                
##  [55] "South America"             "Western Europe"           
##  [57] "Middle Africa"             "Northern Europe"          
##  [59] "Western Asia"              "Western Africa"           
##  [61] "Western Africa"            "Western Africa"           
##  [63] "Western Africa"            "Middle Africa"            
##  [65] "Southern Europe"           "Northern America"         
##  [67] "Central America"           "South America"            
##  [69] "Central America"           "Southern Europe"          
##  [71] "Caribbean"                 "Eastern Europe"           
##  [73] "South-Eastern Asia"        "Southern Asia"            
##  [75] "Northern Europe"           "Southern Asia"            
##  [77] "Western Asia"              "Northern Europe"          
##  [79] "Western Asia"              "Southern Europe"          
##  [81] "Caribbean"                 "Western Asia"             
##  [83] "Eastern Asia"              "Central Asia"             
##  [85] "Eastern Africa"            "Central Asia"             
##  [87] "South-Eastern Asia"        "Eastern Asia"             
##  [89] "Southern Europe"           "Western Asia"             
##  [91] "South-Eastern Asia"        "Western Asia"             
##  [93] "Western Africa"            "Northern Africa"          
##  [95] "Southern Asia"             "Southern Africa"          
##  [97] "Northern Europe"           "Western Europe"           
##  [99] "Northern Europe"           "Northern Africa"          
## [101] "Eastern Europe"            "Eastern Africa"           
## [103] "Central America"           "Southern Europe"          
## [105] "Western Africa"            "South-Eastern Asia"       
## [107] "Southern Europe"           "Eastern Asia"             
## [109] "Eastern Africa"            "Western Africa"           
## [111] "Eastern Africa"            "South-Eastern Asia"       
## [113] "Southern Africa"           "Melanesia"                
## [115] "Western Africa"            "Western Africa"           
## [117] "Central America"           "Western Europe"           
## [119] "Northern Europe"           "Southern Asia"            
## [121] "Australia and New Zealand" "Western Asia"             
## [123] "Southern Asia"             "Central America"          
## [125] "South America"             "South-Eastern Asia"       
## [127] "Melanesia"                 "Eastern Europe"           
## [129] "Caribbean"                 "Eastern Asia"             
## [131] "Southern Europe"           "South America"            
## [133] "Western Asia"              "Western Asia"             
## [135] "Eastern Europe"            "Eastern Europe"           
## [137] "Eastern Africa"            "Northern Africa"          
## [139] "Western Asia"              "Northern Africa"          
## [141] "Eastern Africa"            "Western Africa"           
## [143] "Melanesia"                 "Western Africa"           
## [145] "Central America"           "Eastern Africa"           
## [147] "Eastern Africa"            "Southern Europe"          
## [149] "South America"             "Eastern Europe"           
## [151] "Southern Europe"           "Northern Europe"          
## [153] "Southern Africa"           "Western Asia"             
## [155] "Middle Africa"             "Western Africa"           
## [157] "South-Eastern Asia"        "Central Asia"             
## [159] "Central Asia"              "South-Eastern Asia"       
## [161] "Caribbean"                 "Northern Africa"          
## [163] "Western Asia"              "Eastern Asia"             
## [165] "Eastern Africa"            "Eastern Africa"           
## [167] "Eastern Europe"            "South America"            
## [169] "Northern America"          "Central Asia"             
## [171] "South America"             "South-Eastern Asia"       
## [173] "Melanesia"                 "Western Asia"             
## [175] "Southern Africa"           "Eastern Africa"           
## [177] "Eastern Africa"
\end{verbatim}

Subsetting based on data attributes Subsetting based on data attributes
is a combination of creating a logical from the columns of your data
frame and subsetting the Spatial\_\_\_DataFrame object. This is similar
to how you subset an ordinary data frame.

Create a logical from a column, let's say countries in Asia:

in\_asia \textless- countries\_spdf\$region == ``Asia'' in\_asia Then,
use the logical to select rows of the Spatial\_\_\_DataFrame object:

countries\_spdf{[}in\_asia, {]} Can you subset out New Zealand and plot
it?

Instructions 100 XP Create a logical vector called is\_nz that tests if
the name column is equal to ``New Zealand''. Create a new spatial object
called nz by using is\_nz to subset countries\_spdf. Plot nz.

\begin{Shaded}
\begin{Highlighting}[]
\CommentTok{# Create logical vector: is_nz}
\NormalTok{is_nz <-}\StringTok{ }\NormalTok{countries_spdf}\OperatorTok{$}\NormalTok{name }\OperatorTok{==}\StringTok{ "New Zealand"}

\CommentTok{# Subset countries_spdf using is_nz: nz}
\NormalTok{nz <-}\StringTok{ }\NormalTok{countries_spdf[is_nz,]}

\CommentTok{# Plot nz}
\KeywordTok{plot}\NormalTok{(nz)}
\end{Highlighting}
\end{Shaded}

\includegraphics{Geospacial-Data_files/figure-latex/sbda-1.pdf}

tmap, a package that works with sp objects You've had to learn quite a
few new things just to be able to understand and do basic manipulation
of these spatial objects defined by sp, but now you get to experience
some payoff! There are a number of neat packages that expect spatial
data in sp objects and which make working with spatial data easy.

Let's take a look at the tmap package for creating maps. You'll learn
more about its philosophy and structure in the next video, but first we
want you to see how easy it is to use.

tmap has the qtm() function for quick thematic maps. It follows the
ideas of qplot() from ggplot2 but with a couple of important
differences. Instead of expecting data in a data frame like ggplot2(),
it expects data in a spatial object and uses the argument shp to specify
it. Another important difference is that tmap doesn't use non-standard
evaluation (see the Writing Functions in R course for more about this),
so variables need to be surrounded by quotes when specifying mappings.

Try this example in the console:

library(tmap) qtm(shp = countries\_spdf, fill = ``population'') How easy
was that!? Can you make a choropleth of another variable contained in
countries\_spdf: gdp?

Instructions 100 XP Using the example as a guide, create a choropleth
map of the gdp variable using qtm().

\begin{Shaded}
\begin{Highlighting}[]
\KeywordTok{library}\NormalTok{(sp)}
\KeywordTok{library}\NormalTok{(tmap)}

\CommentTok{# Use qtm() to create a choropleth map of gdp}
\KeywordTok{qtm}\NormalTok{(}\DataTypeTok{shp =}\NormalTok{ countries_spdf, }\DataTypeTok{fill =} \StringTok{"gdp"}\NormalTok{)}
\end{Highlighting}
\end{Shaded}

\includegraphics{Geospacial-Data_files/figure-latex/tmap-1.pdf}

Building a plot in layers Now that you know a bit more about tmap(),
let's build up your previous plot of population in layers and make a few
tweaks to improve it. You start with a tm\_shape() layer that defines
the data you want to use, then add a tm\_fill() layer to color-in your
polygons using the variable population:

tm\_shape(countries\_spdf) + tm\_fill(col = ``population'') Probably the
biggest problem with the resulting plot is that the color scale isn't
very informative: the first color (palest yellow) covers all countries
with population less than 200 million! Since the color scale is
associated with the tm\_fill() layer, tweaks to this scale happen in
this call. You'll learn a lot more about color in Chapter 3, but for
now, know that the style argument controls how the breaks are chosen.

Your plot also needs some country outlines. You can add a tm\_borders()
layer for this, but let's not make them too visually strong. Perhaps a
brown would be nice.

The benefit of using spatial objects becomes really clear when you
switch the kind of plot you make. Let's also try a bubble plot where the
size of the bubbles correspond to population. If you were using ggplot2,
this would involve a lot of reshaping of your data. With tmap, you just
switch out a layer.

Instructions 100 XP Add style = ``quantile'' to tm\_fill(). This chooses
the breaks in the color scale based on equal numbers of observations in
each interval. To the same plot, add a tm\_borders() layer with col =
``burlywood4''. Create new plot the same as the first, but instead of
tm\_fill() add a tm\_bubbles() layer with size mapped to population.

\begin{Shaded}
\begin{Highlighting}[]
\KeywordTok{library}\NormalTok{(sp)}
\KeywordTok{library}\NormalTok{(tmap)}

\CommentTok{# Add style argument to the tm_fill() call}
\KeywordTok{tm_shape}\NormalTok{(countries_spdf) }\OperatorTok{+}
\StringTok{  }\KeywordTok{tm_fill}\NormalTok{(}\DataTypeTok{col =} \StringTok{"population"}\NormalTok{, }\DataTypeTok{style =} \StringTok{"quantile"}\NormalTok{) }\OperatorTok{+}
\StringTok{  }\CommentTok{# Add a tm_borders() layer }
\StringTok{  }\KeywordTok{tm_borders}\NormalTok{(}\DataTypeTok{col =} \StringTok{"burlywood4"}\NormalTok{)}
\end{Highlighting}
\end{Shaded}

\includegraphics{Geospacial-Data_files/figure-latex/bpil-1.pdf}

\begin{Shaded}
\begin{Highlighting}[]
\CommentTok{# New plot, with tm_bubbles() instead of tm_fill()}
\KeywordTok{tm_shape}\NormalTok{(countries_spdf) }\OperatorTok{+}
\StringTok{  }\KeywordTok{tm_bubbles}\NormalTok{(}\DataTypeTok{size =} \StringTok{"population"}\NormalTok{, }\DataTypeTok{style =} \StringTok{"quantile"}\NormalTok{) }\OperatorTok{+}
\StringTok{  }\CommentTok{# Add a tm_borders() layer }
\StringTok{  }\KeywordTok{tm_borders}\NormalTok{(}\DataTypeTok{col =} \StringTok{"burlywood4"}\NormalTok{)}
\end{Highlighting}
\end{Shaded}

\includegraphics{Geospacial-Data_files/figure-latex/bpil-2.pdf}

Why is Greenland so big? Take a closer look at the plot. Why does
Greenland look bigger than the contiguous US when it's actually only
about one-third the size?

When you plot longitude and latitude locations on the x- and y-axes of a
plot, you are treating 1 degree of longitude as the same size no matter
where you are. However, because the earth is roughly spherical, the
distance described by 1 degree of longitude depends on your latitude,
varying from 111km at the equator, to 0 km at the poles.

The way you have taken positions on a sphere and drawn them in a two
dimensional plane is described by a projection. The default you've used
here (also known as an Equirectangular projection) distorts the width of
areas near the poles. Every projection involves some kind of distortion
(since a sphere isn't a plane!), but different projections try to
preserve different properties (e.g.~areas, angles or distances).

In tmap, tm\_shape() takes an argument projection that allows you to
swap projections for the plot.

(Note: changing the projection of a ggplot2 plot is done using the
coord\_map() function. See ?coord\_map() for more details.)

Instructions 100 XP To help you see the differences between projections,
we've added a tm\_grid() layer which adds equispaced longitude and
latitude lines to the plot.

Within your tm\_shape() call:

Try a Hobo--Dyer projection (projection = ``hd''), designed to preserve
area. In a second plot, try a Robinson projection (projection =
``robin''), designed as a compromise between preserving local angles and
area. Just for fun, repeat the previous plot, but add
tm\_style\_classic() to see how tmap can control all aspects of the maps
display.

\begin{Shaded}
\begin{Highlighting}[]
\KeywordTok{library}\NormalTok{(sp)}
\KeywordTok{library}\NormalTok{(tmap)}

\CommentTok{# Switch to a Hobo–Dyer projection}
\KeywordTok{tm_shape}\NormalTok{(countries_spdf, }\DataTypeTok{projection =} \StringTok{"hd"}\NormalTok{) }\OperatorTok{+}
\StringTok{  }\KeywordTok{tm_grid}\NormalTok{(}\DataTypeTok{n.x =} \DecValTok{11}\NormalTok{, }\DataTypeTok{n.y =} \DecValTok{11}\NormalTok{) }\OperatorTok{+}
\StringTok{  }\KeywordTok{tm_fill}\NormalTok{(}\DataTypeTok{col =} \StringTok{"population"}\NormalTok{, }\DataTypeTok{style =} \StringTok{"quantile"}\NormalTok{)  }\OperatorTok{+}
\StringTok{  }\KeywordTok{tm_borders}\NormalTok{(}\DataTypeTok{col =} \StringTok{"burlywood4"}\NormalTok{) }
\end{Highlighting}
\end{Shaded}

\includegraphics{Geospacial-Data_files/figure-latex/wigsb-1.pdf}

\begin{Shaded}
\begin{Highlighting}[]
\CommentTok{# Switch to a Robinson projection}
\KeywordTok{tm_shape}\NormalTok{(countries_spdf, }\DataTypeTok{projection =} \StringTok{"robin"}\NormalTok{) }\OperatorTok{+}
\StringTok{  }\KeywordTok{tm_grid}\NormalTok{(}\DataTypeTok{n.x =} \DecValTok{11}\NormalTok{, }\DataTypeTok{n.y =} \DecValTok{11}\NormalTok{) }\OperatorTok{+}
\StringTok{  }\KeywordTok{tm_fill}\NormalTok{(}\DataTypeTok{col =} \StringTok{"population"}\NormalTok{, }\DataTypeTok{style =} \StringTok{"quantile"}\NormalTok{)  }\OperatorTok{+}
\StringTok{  }\KeywordTok{tm_borders}\NormalTok{(}\DataTypeTok{col =} \StringTok{"burlywood4"}\NormalTok{) }
\end{Highlighting}
\end{Shaded}

\includegraphics{Geospacial-Data_files/figure-latex/wigsb-2.pdf}

\begin{Shaded}
\begin{Highlighting}[]
\CommentTok{# Add tm_style_classic() to your plot}
\KeywordTok{tm_shape}\NormalTok{(countries_spdf, }\DataTypeTok{projection =} \StringTok{"robin"}\NormalTok{) }\OperatorTok{+}
\StringTok{  }\KeywordTok{tm_grid}\NormalTok{(}\DataTypeTok{n.x =} \DecValTok{11}\NormalTok{, }\DataTypeTok{n.y =} \DecValTok{11}\NormalTok{) }\OperatorTok{+}
\StringTok{  }\KeywordTok{tm_fill}\NormalTok{(}\DataTypeTok{col =} \StringTok{"population"}\NormalTok{, }\DataTypeTok{style =} \StringTok{"quantile"}\NormalTok{)  }\OperatorTok{+}
\StringTok{  }\KeywordTok{tm_borders}\NormalTok{(}\DataTypeTok{col =} \StringTok{"burlywood4"}\NormalTok{) }\OperatorTok{+}
\StringTok{  }\KeywordTok{tm_style}\NormalTok{(}\StringTok{"classic"}\NormalTok{)}
\end{Highlighting}
\end{Shaded}

\includegraphics{Geospacial-Data_files/figure-latex/wigsb-3.pdf}

Saving a tmap plot Saving tmap plots is easy with the tmap\_save()
function. The first argument, tm, is the plot to save and the second,
filename, is the file to save it to. If you leave tm unspecified, the
last tmap plot printed will be saved.

The extension of the file name specifies the file type, for example .png
or .pdf for static plots. One really neat thing about tmap is that you
can save an interactive version which leverages the leaflet package. To
get an interactive version, use tmap\_save() but use the file name
extension .html.

Instructions 100 XP Save your plot from the previous exercise in the
following ways. Neither plot will display in your workspace, but you'll
be able to take a look at them once you complete the exercise.

Save it as a static plot by specifying the filename population.png. Save
it as an interactice plot by specifying the filename population.html.

\begin{Shaded}
\begin{Highlighting}[]
\KeywordTok{library}\NormalTok{(sp)}
\KeywordTok{library}\NormalTok{(tmap)}

\CommentTok{# Plot from last exercise}
\KeywordTok{tm_shape}\NormalTok{(countries_spdf) }\OperatorTok{+}
\StringTok{  }\KeywordTok{tm_grid}\NormalTok{(}\DataTypeTok{n.x =} \DecValTok{11}\NormalTok{, }\DataTypeTok{n.y =} \DecValTok{11}\NormalTok{, }\DataTypeTok{projection =} \StringTok{"longlat"}\NormalTok{) }\OperatorTok{+}
\StringTok{  }\KeywordTok{tm_fill}\NormalTok{(}\DataTypeTok{col =} \StringTok{"population"}\NormalTok{, }\DataTypeTok{style =} \StringTok{"quantile"}\NormalTok{)  }\OperatorTok{+}
\StringTok{  }\KeywordTok{tm_borders}\NormalTok{(}\DataTypeTok{col =} \StringTok{"burlywood4"}\NormalTok{)}
\end{Highlighting}
\end{Shaded}

\includegraphics{Geospacial-Data_files/figure-latex/satp-1.pdf}

\begin{Shaded}
\begin{Highlighting}[]
\CommentTok{# Save a static version "population.png"}
\KeywordTok{tmap_save}\NormalTok{(}\DataTypeTok{filename =} \StringTok{"population.png"}\NormalTok{)}

\CommentTok{# Save an interactive version "population.html"}
\KeywordTok{tm_shape}\NormalTok{(countries_spdf) }\OperatorTok{+}
\StringTok{  }\KeywordTok{tm_grid}\NormalTok{(}\DataTypeTok{n.x =} \DecValTok{11}\NormalTok{, }\DataTypeTok{n.y =} \DecValTok{11}\NormalTok{, }\DataTypeTok{projection =} \StringTok{"longlat"}\NormalTok{) }\OperatorTok{+}
\StringTok{  }\KeywordTok{tm_fill}\NormalTok{(}\DataTypeTok{col =} \StringTok{"population"}\NormalTok{, }\DataTypeTok{style =} \StringTok{"quantile"}\NormalTok{)  }\OperatorTok{+}
\StringTok{  }\KeywordTok{tm_borders}\NormalTok{(}\DataTypeTok{col =} \StringTok{"burlywood4"}\NormalTok{)}
\end{Highlighting}
\end{Shaded}

\includegraphics{Geospacial-Data_files/figure-latex/satp-2.pdf}

\begin{Shaded}
\begin{Highlighting}[]
\KeywordTok{tmap_save}\NormalTok{(}\DataTypeTok{filename =} \StringTok{"population.html"}\NormalTok{)}
\end{Highlighting}
\end{Shaded}

\hypertarget{raster}{%
\subsection{Raster}\label{raster}}

What's a raster object? Just like sp classes, the raster classes have
methods to help with basic viewing and manipulation of objects, like
print() and summary(), and you can always dig deeper into their
structure with str().

Let's jump in and take a look at a raster we've loaded for you, pop.
Keep an eye out for a few things:

Can you see where the coordinate information is kept? Can you tell from
the summary() how big the raster is? What do you think might be stored
in this raster? Instructions 100 XP Print pop to the console. Call str()
on pop with max.level = 2. Call summary() on pop.

\begin{Shaded}
\begin{Highlighting}[]
\KeywordTok{load}\NormalTok{(}\DataTypeTok{file =} \StringTok{"pop.rda"}\NormalTok{)}

\KeywordTok{library}\NormalTok{(raster)}

\CommentTok{# Print pop}
\KeywordTok{print}\NormalTok{(pop)}
\end{Highlighting}
\end{Shaded}

\begin{verbatim}
## class      : RasterLayer 
## dimensions : 480, 660, 316800  (nrow, ncol, ncell)
## resolution : 0.008333333, 0.008333333  (x, y)
## extent     : -75, -69.5, 39, 43  (xmin, xmax, ymin, ymax)
## crs        : +proj=longlat +datum=NAD83 +no_defs +ellps=GRS80 +towgs84=0,0,0 
## source     : memory
## names      : num_people 
## values     : 0, 41140  (min, max)
\end{verbatim}

\begin{Shaded}
\begin{Highlighting}[]
\CommentTok{# Call str() on pop, with max.level = 2}
\KeywordTok{str}\NormalTok{(pop, }\DataTypeTok{max.level =} \DecValTok{2}\NormalTok{)}
\end{Highlighting}
\end{Shaded}

\begin{verbatim}
## Formal class 'RasterLayer' [package "raster"] with 12 slots
##   ..@ file    :Formal class '.RasterFile' [package "raster"] with 13 slots
##   ..@ data    :Formal class '.SingleLayerData' [package "raster"] with 13 slots
##   ..@ legend  :Formal class '.RasterLegend' [package "raster"] with 5 slots
##   ..@ title   : chr(0) 
##   ..@ extent  :Formal class 'Extent' [package "raster"] with 4 slots
##   ..@ rotated : logi FALSE
##   ..@ rotation:Formal class '.Rotation' [package "raster"] with 2 slots
##   ..@ ncols   : int 660
##   ..@ nrows   : int 480
##   ..@ crs     :Formal class 'CRS' [package "sp"] with 1 slot
##   ..@ history : list()
##   ..@ z       : list()
\end{verbatim}

\begin{Shaded}
\begin{Highlighting}[]
\CommentTok{# Call summary on pop}
\KeywordTok{summary}\NormalTok{(pop)}
\end{Highlighting}
\end{Shaded}

\begin{verbatim}
##         num_people
## Min.             0
## 1st Qu.          0
## Median           0
## 3rd Qu.         23
## Max.         41140
## NA's             0
\end{verbatim}

Some useful methods pop is a RasterLayer object, which contains the
population around the Boston and NYC areas. Each grid cell simply
contains a count of the number of people that live inside that cell.

You saw in the previous exercise that print() gives a useful summary of
the object including the coordinate reference system, the size of the
grid (both in number of rows and columns and geographical coordinates),
and some basic info on the values stored in the grid. But it was very
succinct; what if you want to see some of the values in the object?

The first way is to simply plot() the object. There is a plot() method
for raster objects that creates a heatmap of the values.

If you want to extract the values from a raster object you can use the
values() function, which pulls out a vector of the values. There are
316,800 values in the pop raster, so you won't want to print them all
out, but you can use str() and head() to take a peek.

Instructions 100 XP Call plot() on pop. Can you see where NYC is? Call
str() on values(pop). Call head() on values(pop).

\begin{Shaded}
\begin{Highlighting}[]
\CommentTok{# Call plot() on pop}
\KeywordTok{plot}\NormalTok{(pop)}
\end{Highlighting}
\end{Shaded}

\includegraphics{Geospacial-Data_files/figure-latex/sum-1.pdf}

\begin{Shaded}
\begin{Highlighting}[]
\CommentTok{# Call str() on values(pop)}
\KeywordTok{str}\NormalTok{(}\KeywordTok{values}\NormalTok{(pop))}
\end{Highlighting}
\end{Shaded}

\begin{verbatim}
##  int [1:316800] 15 13 12 12 12 5 3 3 4 4 ...
\end{verbatim}

\begin{Shaded}
\begin{Highlighting}[]
\CommentTok{# Call head() on values(pop)}
\KeywordTok{head}\NormalTok{(}\KeywordTok{values}\NormalTok{(pop))}
\end{Highlighting}
\end{Shaded}

\begin{verbatim}
## [1] 15 13 12 12 12  5
\end{verbatim}

A more complicated object The raster package provides the RasterLayer
object, but also a couple of more complicated objects: RasterStack and
RasterBrick. These two objects are designed for storing many rasters,
all of the same extents and dimension (a.k.a. multi-band, or multi-layer
rasters).

You can think of RasterLayer like a matrix, but RasterStack and
RasterBrick objects are more like three dimensional arrays. One
additional thing you need to know to handle them is how to specify a
particular layer.

You can use \$ or {[}{[} subsetting on a RasterStack or RasterBrick to
grab one layer and return a new RasterLayer object. For example, if x is
a RasterStack, x\$layer\_name or x{[}{[}``layer\_name''{]}{]} will
return a RasterLayer with only the layer called layer\_name in it.

Let's look at a RasterStack object called pop\_by\_age that covers the
same area as pop but now contains layers for population broken into few
different age groups.

Instructions 100 XP Print pop\_by\_age. Can you see the names of all the
layers? Subset out the under\_1 layer using {[}{[} subsetting. Plot the
under\_1 layer by passing your code from the previous instruction to
plot().

\begin{Shaded}
\begin{Highlighting}[]
\KeywordTok{load}\NormalTok{(}\DataTypeTok{file =} \StringTok{"pop_by_age.rda"}\NormalTok{)}

\CommentTok{# Print pop_by_age}
\NormalTok{pop_by_age}
\end{Highlighting}
\end{Shaded}

\begin{verbatim}
## class      : RasterStack 
## dimensions : 480, 660, 316800, 7  (nrow, ncol, ncell, nlayers)
## resolution : 0.008333333, 0.008333333  (x, y)
## extent     : -75, -69.5, 39, 43  (xmin, xmax, ymin, ymax)
## crs        : +proj=longlat +datum=NAD83 +no_defs +ellps=GRS80 +towgs84=0,0,0 
## names      : under_1, age_1_4, age_5_17, age_18_24, age_25_64, age_65_79, age_80_over 
## min values :       0,       0,        0,         0,         0,         0,           0 
## max values :     698,    2467,     8017,      7714,     29263,      3720,        1743
\end{verbatim}

\begin{Shaded}
\begin{Highlighting}[]
\CommentTok{# Subset out the under_1 layer using [[}
\NormalTok{pop_by_age[[}\StringTok{"under_1"}\NormalTok{]]}
\end{Highlighting}
\end{Shaded}

\begin{verbatim}
## class      : RasterLayer 
## dimensions : 480, 660, 316800  (nrow, ncol, ncell)
## resolution : 0.008333333, 0.008333333  (x, y)
## extent     : -75, -69.5, 39, 43  (xmin, xmax, ymin, ymax)
## crs        : +proj=longlat +datum=NAD83 +no_defs +ellps=GRS80 +towgs84=0,0,0 
## source     : memory
## names      : under_1 
## values     : 0, 698  (min, max)
\end{verbatim}

\begin{Shaded}
\begin{Highlighting}[]
\CommentTok{# Plot the under_1 layer}
\KeywordTok{plot}\NormalTok{(pop_by_age[[}\StringTok{"under_1"}\NormalTok{]])}
\end{Highlighting}
\end{Shaded}

\includegraphics{Geospacial-Data_files/figure-latex/amco-1.pdf}

A package that uses Raster objects You saw the tmap package makes
visualizing spatial classes in sp easy. The good news is that it works
with the raster classes too! You simply pass your Raster\_\_\_ object as
the shp argument to the tm\_shape() function, and then add a
tm\_raster() layer like this:

tm\_shape(raster\_object) + tm\_raster() When working with a RasterStack
or a RasterBrick object, such as the pop\_by\_age object you created in
the last exercise, you can display one of its layers using the col
(short for ``color'') argument in tm\_raster(), surrounding the layer
name in quotes.

You'll work with tmap throughout the course, but we also want to show
you another package, rasterVis, also designed specifically for
visualizing raster objects. There are a few different functions you can
use in rasterVis to make plots, but let's just try one of them for now:
levelplot().

Instructions 100 XP Use tmap to plot the pop object, by specifying pop
as the shp argument to tm\_shape() and adding a tm\_raster() layer. Use
tmap to plot the under\_1 layer of pop\_by\_age, a RasterStack object.
Call the rasterVis function levelplot() on pop.

\begin{Shaded}
\begin{Highlighting}[]
\KeywordTok{library}\NormalTok{(tmap)}

\CommentTok{# Specify pop as the shp and add a tm_raster() layer}
\KeywordTok{tm_shape}\NormalTok{(pop) }\OperatorTok{+}
\StringTok{  }\KeywordTok{tm_raster}\NormalTok{()}
\end{Highlighting}
\end{Shaded}

\includegraphics{Geospacial-Data_files/figure-latex/apturo-1.pdf}

\begin{Shaded}
\begin{Highlighting}[]
\CommentTok{# Plot the under_1 layer in pop_by_age}
\KeywordTok{tm_shape}\NormalTok{(pop_by_age) }\OperatorTok{+}
\StringTok{  }\KeywordTok{tm_raster}\NormalTok{(}\DataTypeTok{col =} \StringTok{"under_1"}\NormalTok{)}
\end{Highlighting}
\end{Shaded}

\includegraphics{Geospacial-Data_files/figure-latex/apturo-2.pdf}

\begin{Shaded}
\begin{Highlighting}[]
\KeywordTok{library}\NormalTok{(rasterVis)}
\CommentTok{# Call levelplot() on pop}
\KeywordTok{levelplot}\NormalTok{(pop)}
\end{Highlighting}
\end{Shaded}

\includegraphics{Geospacial-Data_files/figure-latex/apturo-3.pdf}

Adding a custom continuous color palette to ggplot2 plots The most
versatile way to add a custom continuous scale to ggplot2 plots is with
scale\_color\_gradientn() or scale\_fill\_gradientn(). How do you know
which to use? Match the function to the aesthetic you have mapped. For
example, in your plot of predicted house price from Chapter 1, you
mapped fill to price, so you'd need to use scale\_fill\_gradientn().

These two functions take an argument colors where you pass a vector of
colors that defines your palette. This is where the versatility comes
in. You can generate your palette in any way you choose, automatically
using something like RColorBrewer or viridisLite, or manually by
specifying colors by name or hex code.

The scale\_\_\_gradientn() functions handle how these colors are mapped
to values of your variable, although there is control available through
the values argument.

Let's play with some alternative color scales for your predicted house
price heatmap from Chapter 1 (we've dropped the map background to reduce
computation time, so you can see your plots quickly).

Instructions 3/3 0 XP Create a palette called blups from 9 steps on the
RColorBrewer palette ``BuPu''. Add scale\_fill\_gradientn() and pass the
blups palette as the colors argument. Create a palette called vir from 9
steps on the viridis() palette from viridisLite. Add
scale\_fill\_gradientn() and pass the vir palette as the colors
argument. Create a palette called mag from 9 steps on the magma()
palette from viridisLite. Add scale\_fill\_gradientn() and pass the mag
palette as the colors argument.

\begin{Shaded}
\begin{Highlighting}[]
\KeywordTok{library}\NormalTok{(RColorBrewer)}
\CommentTok{# 9 steps on the RColorBrewer "BuPu" palette: blups}
\NormalTok{blups <-}\StringTok{ }\KeywordTok{brewer.pal}\NormalTok{(}\DecValTok{9}\NormalTok{, }\StringTok{"BuPu"}\NormalTok{)}

\CommentTok{# Add scale_fill_gradientn() with the blups palette}
\KeywordTok{ggplot}\NormalTok{(preds) }\OperatorTok{+}
\StringTok{  }\KeywordTok{geom_tile}\NormalTok{(}\KeywordTok{aes}\NormalTok{(lon, lat, }\DataTypeTok{fill =}\NormalTok{ predicted_price), }\DataTypeTok{alpha =} \FloatTok{0.8}\NormalTok{) }\OperatorTok{+}
\StringTok{  }\KeywordTok{scale_fill_gradientn}\NormalTok{(}\DataTypeTok{colors =}\NormalTok{ blups)}
\end{Highlighting}
\end{Shaded}

\includegraphics{Geospacial-Data_files/figure-latex/accc-1.pdf}

\begin{Shaded}
\begin{Highlighting}[]
\KeywordTok{library}\NormalTok{(viridisLite)}
\CommentTok{# viridisLite viridis palette with 9 steps: vir}
\NormalTok{vir <-}\StringTok{ }\KeywordTok{viridis}\NormalTok{(}\DecValTok{9}\NormalTok{)}

\CommentTok{# Add scale_fill_gradientn() with the vir palette}
\KeywordTok{ggplot}\NormalTok{(preds) }\OperatorTok{+}
\StringTok{  }\KeywordTok{geom_tile}\NormalTok{(}\KeywordTok{aes}\NormalTok{(lon, lat, }\DataTypeTok{fill =}\NormalTok{ predicted_price), }\DataTypeTok{alpha =} \FloatTok{0.8}\NormalTok{) }\OperatorTok{+}
\StringTok{  }\KeywordTok{scale_fill_gradientn}\NormalTok{(}\DataTypeTok{colors =}\NormalTok{ vir)}
\end{Highlighting}
\end{Shaded}

\includegraphics{Geospacial-Data_files/figure-latex/accc-2.pdf}

\begin{Shaded}
\begin{Highlighting}[]
\KeywordTok{library}\NormalTok{(viridisLite)}
\CommentTok{# mag: a viridisLite magma palette with 9 steps}
\NormalTok{mag <-}\StringTok{ }\KeywordTok{magma}\NormalTok{(}\DecValTok{9}\NormalTok{)}

\CommentTok{# Add scale_fill_gradientn() with the mag palette}
\KeywordTok{ggplot}\NormalTok{(preds) }\OperatorTok{+}
\StringTok{  }\KeywordTok{geom_tile}\NormalTok{(}\KeywordTok{aes}\NormalTok{(lon, lat, }\DataTypeTok{fill =}\NormalTok{ predicted_price), }\DataTypeTok{alpha =} \FloatTok{0.8}\NormalTok{) }\OperatorTok{+}
\StringTok{  }\KeywordTok{scale_fill_gradientn}\NormalTok{(}\DataTypeTok{colors =}\NormalTok{ mag) }
\end{Highlighting}
\end{Shaded}

\includegraphics{Geospacial-Data_files/figure-latex/accc-3.pdf}

Adding a custom continuous color palette to ggplot2 plots The most
versatile way to add a custom continuous scale to ggplot2 plots is with
scale\_color\_gradientn() or scale\_fill\_gradientn(). How do you know
which to use? Match the function to the aesthetic you have mapped. For
example, in your plot of predicted house price from Chapter 1, you
mapped fill to price, so you'd need to use scale\_fill\_gradientn().

These two functions take an argument colors where you pass a vector of
colors that defines your palette. This is where the versatility comes
in. You can generate your palette in any way you choose, automatically
using something like RColorBrewer or viridisLite, or manually by
specifying colors by name or hex code.

The scale\_\_\_gradientn() functions handle how these colors are mapped
to values of your variable, although there is control available through
the values argument.

Let's play with some alternative color scales for your predicted house
price heatmap from Chapter 1 (we've dropped the map background to reduce
computation time, so you can see your plots quickly).

Instructions 3/3 0 XP Create a palette called blups from 9 steps on the
RColorBrewer palette ``BuPu''. Add scale\_fill\_gradientn() and pass the
blups palette as the colors argument. Create a palette called vir from 9
steps on the viridis() palette from viridisLite. Add
scale\_fill\_gradientn() and pass the vir palette as the colors
argument. Create a palette called mag from 9 steps on the magma()
palette from viridisLite. Add scale\_fill\_gradientn() and pass the mag
palette as the colors argument.

\begin{Shaded}
\begin{Highlighting}[]
\KeywordTok{library}\NormalTok{(RColorBrewer)}
\CommentTok{# 9 steps on the RColorBrewer "BuPu" palette: blups}
\NormalTok{blups <-}\StringTok{ }\KeywordTok{brewer.pal}\NormalTok{(}\DecValTok{9}\NormalTok{, }\StringTok{"BuPu"}\NormalTok{)}

\CommentTok{# Add scale_fill_gradientn() with the blups palette}
\KeywordTok{ggplot}\NormalTok{(preds) }\OperatorTok{+}
\StringTok{  }\KeywordTok{geom_tile}\NormalTok{(}\KeywordTok{aes}\NormalTok{(lon, lat, }\DataTypeTok{fill =}\NormalTok{ predicted_price), }\DataTypeTok{alpha =} \FloatTok{0.8}\NormalTok{) }\OperatorTok{+}
\StringTok{  }\KeywordTok{scale_fill_gradientn}\NormalTok{(}\DataTypeTok{colors =}\NormalTok{ blups)}
\end{Highlighting}
\end{Shaded}

\includegraphics{Geospacial-Data_files/figure-latex/accc2-1.pdf}

\begin{Shaded}
\begin{Highlighting}[]
\KeywordTok{library}\NormalTok{(viridisLite)}
\CommentTok{# viridisLite viridis palette with 9 steps: vir}
\NormalTok{vir <-}\StringTok{ }\KeywordTok{viridis}\NormalTok{(}\DecValTok{9}\NormalTok{)}

\CommentTok{# Add scale_fill_gradientn() with the vir palette}
\KeywordTok{ggplot}\NormalTok{(preds) }\OperatorTok{+}
\StringTok{  }\KeywordTok{geom_tile}\NormalTok{(}\KeywordTok{aes}\NormalTok{(lon, lat, }\DataTypeTok{fill =}\NormalTok{ predicted_price), }\DataTypeTok{alpha =} \FloatTok{0.8}\NormalTok{) }\OperatorTok{+}
\StringTok{  }\KeywordTok{scale_fill_gradientn}\NormalTok{(}\DataTypeTok{colors =}\NormalTok{ vir)}
\end{Highlighting}
\end{Shaded}

\includegraphics{Geospacial-Data_files/figure-latex/accc2-2.pdf}

\begin{Shaded}
\begin{Highlighting}[]
\KeywordTok{library}\NormalTok{(viridisLite)}
\CommentTok{# mag: a viridisLite magma palette with 9 steps}
\NormalTok{mag <-}\StringTok{ }\KeywordTok{magma}\NormalTok{(}\DecValTok{9}\NormalTok{)}

\CommentTok{# Add scale_fill_gradientn() with the mag palette}
\KeywordTok{ggplot}\NormalTok{(preds) }\OperatorTok{+}
\StringTok{  }\KeywordTok{geom_tile}\NormalTok{(}\KeywordTok{aes}\NormalTok{(lon, lat, }\DataTypeTok{fill =}\NormalTok{ predicted_price), }\DataTypeTok{alpha =} \FloatTok{0.8}\NormalTok{) }\OperatorTok{+}
\StringTok{  }\KeywordTok{scale_fill_gradientn}\NormalTok{(}\DataTypeTok{colors =}\NormalTok{ mag) }
\end{Highlighting}
\end{Shaded}

\includegraphics{Geospacial-Data_files/figure-latex/accc2-3.pdf}

\textless\textless\textless\textless\textless\textless\textless{} HEAD
Custom palette in tmap Unlike ggplot2, where setting a custom color
scale happens in a scale\_ call, colors in tmap layers are specified in
the layer in which they are mapped. For example, take a plot of the
age\_18\_24 variable from prop\_by\_age:

tm\_shape(prop\_by\_age) + tm\_raster(col = ``age\_18\_24'') Since color
is mapped in the tm\_raster() call, the specification of the palette
also occurs in this call. You simply specify a vector of colors in the
palette argument. This is a another reason it's worth learning ways to
generate a vector of colors. While different packages could have very
different shortcuts for specifying palettes from color packages, they
will generally always have a way to pass in a vector of colors.

Let's use some palettes from the last exercise with this plot.

Instructions 100 XP In the first plot, use the blups palette instead of
the default. In the second plot, use the vir palette instead of the
default. In the third plot, use the rev(mag) palette instead of the
default. rev() just reverses the order of a vector, so this uses the
same colors but in the opposite order.

\begin{Shaded}
\begin{Highlighting}[]
\KeywordTok{load}\NormalTok{(}\DataTypeTok{file =} \StringTok{"prop_by_age.rda"}\NormalTok{)}

\CommentTok{# Generate palettes from last time}
\KeywordTok{library}\NormalTok{(RColorBrewer)}
\NormalTok{blups <-}\StringTok{ }\KeywordTok{brewer.pal}\NormalTok{(}\DecValTok{9}\NormalTok{, }\StringTok{"BuPu"}\NormalTok{)}

\KeywordTok{library}\NormalTok{(viridisLite)}
\NormalTok{vir <-}\StringTok{ }\KeywordTok{viridis}\NormalTok{(}\DecValTok{9}\NormalTok{)}
\NormalTok{mag <-}\StringTok{ }\KeywordTok{magma}\NormalTok{(}\DecValTok{9}\NormalTok{)}

\CommentTok{# Use the blups palette}
\KeywordTok{tm_shape}\NormalTok{(prop_by_age) }\OperatorTok{+}
\StringTok{  }\KeywordTok{tm_raster}\NormalTok{(}\StringTok{"age_18_24"}\NormalTok{, }\DataTypeTok{palette =}\NormalTok{ blups) }\OperatorTok{+}
\StringTok{  }\KeywordTok{tm_legend}\NormalTok{(}\DataTypeTok{position =} \KeywordTok{c}\NormalTok{(}\StringTok{"right"}\NormalTok{, }\StringTok{"bottom"}\NormalTok{))}
\end{Highlighting}
\end{Shaded}

\includegraphics{Geospacial-Data_files/figure-latex/cpot-1.pdf}

\begin{Shaded}
\begin{Highlighting}[]
\CommentTok{# Use the vir palette}
\KeywordTok{tm_shape}\NormalTok{(prop_by_age) }\OperatorTok{+}
\StringTok{  }\KeywordTok{tm_raster}\NormalTok{(}\StringTok{"age_18_24"}\NormalTok{, }\DataTypeTok{palette =}\NormalTok{ vir) }\OperatorTok{+}
\StringTok{  }\KeywordTok{tm_legend}\NormalTok{(}\DataTypeTok{position =} \KeywordTok{c}\NormalTok{(}\StringTok{"right"}\NormalTok{, }\StringTok{"bottom"}\NormalTok{))}
\end{Highlighting}
\end{Shaded}

\includegraphics{Geospacial-Data_files/figure-latex/cpot-2.pdf}

\begin{Shaded}
\begin{Highlighting}[]
\CommentTok{# Use the mag palette but reverse the order}
\KeywordTok{tm_shape}\NormalTok{(prop_by_age) }\OperatorTok{+}
\StringTok{  }\KeywordTok{tm_raster}\NormalTok{(}\StringTok{"age_18_24"}\NormalTok{, }\DataTypeTok{palette =} \KeywordTok{rev}\NormalTok{(mag)) }\OperatorTok{+}
\StringTok{  }\KeywordTok{tm_legend}\NormalTok{(}\DataTypeTok{position =} \KeywordTok{c}\NormalTok{(}\StringTok{"right"}\NormalTok{, }\StringTok{"bottom"}\NormalTok{))}
\end{Highlighting}
\end{Shaded}

\includegraphics{Geospacial-Data_files/figure-latex/cpot-3.pdf}

An interval scale example Let's return to your plot of the proportion of
the population that is between 18 and 24:

tm\_shape(prop\_by\_age) + tm\_raster(``age\_18\_24'', palette = vir) +
tm\_legend(position = c(``right'', ``bottom'')) Your plot was
problematic because most of the proportions fell in the lowest color
level and consequently you didn't see much detail in your plot. One way
to solve this problem is this: instead of breaking the range of your
variable into equal length bins, you can break it into more useful
categories.

Let's start by replicating the tmap default bins: five categories, cut
using ``pretty'' breaks. Then you can try out a few of the other methods
to cut a variable into intervals. Using the classIntervals() function
directly gives you quick feedback on what the breaks will be, but the
best way to try out a set of breaks is to plot them.

(As an aside, another way to solve this kind of problem is to look for a
transform of the variable so that equal length bins of the transformed
scale are more useful.)

Instructions 100 XP Call classIntervals() on
values(prop\_by\_age{[}{[}``age\_18\_24''{]}{]}) with n = 5 and style =
``pretty''. See the problem? 130,770 of your grid cells end up in the
first bin. Now call classIntervals() as above, but with style =
``quantile''. Use the equisized bins by passing the n and style
arguments into the tm\_raster() layer of your plot. Make a histogram of
values(prop\_by\_age{[}{[}``age\_18\_24''{]}{]}). Where would you make
the breaks? Create your own breaks in tm\_raster() by specifying breaks
= c(0.025, 0.05, 0.1, 0.2, 0.25, 0.3, 1). Save your final plot as a
leaflet plot using save\_tmap() and the filename ``prop\_18-24.html''.

\begin{Shaded}
\begin{Highlighting}[]
\NormalTok{mag <-}\StringTok{ }\NormalTok{viridisLite}\OperatorTok{::}\KeywordTok{magma}\NormalTok{(}\DecValTok{7}\NormalTok{)}

\KeywordTok{library}\NormalTok{(classInt)}

\CommentTok{# Create 5 "pretty" breaks with classIntervals()}
\KeywordTok{classIntervals}\NormalTok{(}\KeywordTok{values}\NormalTok{(prop_by_age[[}\StringTok{"age_18_24"}\NormalTok{]]), }
               \DataTypeTok{n =} \DecValTok{5}\NormalTok{, }\DataTypeTok{style =} \StringTok{"pretty"}\NormalTok{)}
\end{Highlighting}
\end{Shaded}

\begin{verbatim}
## style: pretty
##   [0,0.2) [0.2,0.4) [0.4,0.6) [0.6,0.8)   [0.8,1] 
##    130770      1775       302       170       138
\end{verbatim}

\begin{Shaded}
\begin{Highlighting}[]
\CommentTok{# Create 5 "quantile" breaks with classIntervals()}
\KeywordTok{classIntervals}\NormalTok{(}\KeywordTok{values}\NormalTok{(prop_by_age[[}\StringTok{"age_18_24"}\NormalTok{]]), }
               \DataTypeTok{n =} \DecValTok{5}\NormalTok{, }\DataTypeTok{style =} \StringTok{"quantile"}\NormalTok{)}
\end{Highlighting}
\end{Shaded}

\begin{verbatim}
## style: quantile
##                   [0,0)          [0,0.04054054) [0.04054054,0.05882353) 
##                       0                   53218                   25925 
## [0.05882353,0.08108108)          [0.08108108,1] 
##                   27217                   26795
\end{verbatim}

\begin{Shaded}
\begin{Highlighting}[]
\CommentTok{# Use 5 "quantile" breaks in tm_raster()}
\KeywordTok{tm_shape}\NormalTok{(prop_by_age) }\OperatorTok{+}
\StringTok{  }\KeywordTok{tm_raster}\NormalTok{(}\StringTok{"age_18_24"}\NormalTok{, }\DataTypeTok{palette =}\NormalTok{ mag, }\DataTypeTok{style =} \StringTok{"quantile"}\NormalTok{) }\OperatorTok{+}
\StringTok{  }\KeywordTok{tm_legend}\NormalTok{(}\DataTypeTok{position =} \KeywordTok{c}\NormalTok{(}\StringTok{"right"}\NormalTok{, }\StringTok{"bottom"}\NormalTok{))}
\end{Highlighting}
\end{Shaded}

\includegraphics{Geospacial-Data_files/figure-latex/aise-1.pdf}

\begin{Shaded}
\begin{Highlighting}[]
\CommentTok{# Create histogram of proportions}
\KeywordTok{hist}\NormalTok{(}\KeywordTok{values}\NormalTok{(prop_by_age)[, }\StringTok{"age_18_24"}\NormalTok{])}
\end{Highlighting}
\end{Shaded}

\includegraphics{Geospacial-Data_files/figure-latex/aise-2.pdf}

\begin{Shaded}
\begin{Highlighting}[]
\CommentTok{# Use fixed breaks in tm_raster()}
\KeywordTok{tm_shape}\NormalTok{(prop_by_age) }\OperatorTok{+}
\StringTok{  }\KeywordTok{tm_raster}\NormalTok{(}\StringTok{"age_18_24"}\NormalTok{, }\DataTypeTok{palette =}\NormalTok{ mag,}
    \DataTypeTok{style =} \StringTok{"fixed"}\NormalTok{, }\DataTypeTok{breaks =} \KeywordTok{c}\NormalTok{(}\FloatTok{0.025}\NormalTok{, }\FloatTok{0.05}\NormalTok{, }\FloatTok{0.1}\NormalTok{, }\FloatTok{0.2}\NormalTok{, }\FloatTok{0.25}\NormalTok{, }\FloatTok{0.3}\NormalTok{, }\DecValTok{1}\NormalTok{))}
\end{Highlighting}
\end{Shaded}

\includegraphics{Geospacial-Data_files/figure-latex/aise-3.pdf}

\begin{Shaded}
\begin{Highlighting}[]
\CommentTok{# Save your plot to "prop_18-24.html"}
\KeywordTok{tmap_save}\NormalTok{(}\DataTypeTok{filename =} \StringTok{"prop_18-24.html"}\NormalTok{)}
\end{Highlighting}
\end{Shaded}

A diverging scale example Let's take a look at another dataset where the
default color scale isn't appropriate. This raster, migration, has an
estimate of the net number of people who have moved into each cell of
the raster between the years of 1990 and 2000. A positive number
indicates a net immigration, and a negative number an emigration. Take a
look:

tm\_shape(migration) + tm\_raster() + tm\_legend(outside = TRUE,
outside.position = c(``bottom'')) The default color scale doesn't look
very helpful, but tmap is actually doing something quite clever: it has
automatically chosen a diverging color scale. A diverging scale is
appropriate since large movements of people are large positive numbers
or large (in magnitude) negative numbers. Zero (i.e.~no net migration)
is a natural midpoint.

tmap chooses a diverging scale when there are both positive and negative
values in the mapped variable and chooses zero as the midpoint. This
isn't always the right approach. Imagine you are mapping a relative
change as percentages; 100\% might be the most intuitive midpoint. If
you need something different, the best way to proceed is to generate a
diverging palette (with an odd number of steps, so there is a middle
color) and specify the breaks yourself.

Let's see if you can get a more informative map by adding a diverging
scale yourself.

(Data source: de Sherbinin, A., M. Levy, S. Adamo, K. MacManus, G.
Yetman, V. Mara, L. Razafindrazay, B. Goodrich, T. Srebotnjak, C.
Aichele, and L. Pistolesi. 2015. Global Estimated Net Migration Grids by
Decade: 1970-2000. Palisades, NY: NASA Socioeconomic Data and
Applications Center (SEDAC). \url{http://dx.doi.org/10.7927/H4319SVC}
Accessed 27 Sep 2016)

Instructions 100 XP Print migration to verify this is a RasterLayer
object and take a look at the range in migration values. Generate a
diverging palette, called red\_gray, of 7 colors from the ``RdGy''
palette in RColorBrewer. Use the diverging set of colors, red\_gray, as
the palette for your plot. This uses your colors, but the breaks aren't
useful. Add fixed breaks for the color scale of: c(-5e6, -5e3, -5e2,
-5e1, 5e1, 5e2, 5e3, 5e6)

\begin{Shaded}
\begin{Highlighting}[]
\CommentTok{# Print migration}
\KeywordTok{load}\NormalTok{(}\DataTypeTok{file =} \StringTok{"migration.rda"}\NormalTok{)}
\NormalTok{migration}
\end{Highlighting}
\end{Shaded}

\begin{verbatim}
## class      : RasterLayer 
## dimensions : 49, 116, 5684  (nrow, ncol, ncell)
## resolution : 0.5, 0.5  (x, y)
## extent     : -125, -67, 25, 49.5  (xmin, xmax, ymin, ymax)
## crs        : +proj=longlat +datum=WGS84 +no_defs +ellps=WGS84 +towgs84=0,0,0 
## source     : memory
## names      : net_migration 
## values     : -4560234, 806052.2  (min, max)
\end{verbatim}

\begin{Shaded}
\begin{Highlighting}[]
\CommentTok{# Diverging "RdGy" palette}
\NormalTok{red_gray <-}\StringTok{ }\KeywordTok{brewer.pal}\NormalTok{(}\DecValTok{7}\NormalTok{, }\StringTok{"RdGy"}\NormalTok{)}

\CommentTok{# Use red_gray as the palette }
\KeywordTok{tm_shape}\NormalTok{(migration) }\OperatorTok{+}
\StringTok{  }\KeywordTok{tm_raster}\NormalTok{(}\DataTypeTok{palette =}\NormalTok{ red_gray) }\OperatorTok{+}
\StringTok{  }\KeywordTok{tm_legend}\NormalTok{(}\DataTypeTok{outside =} \OtherTok{TRUE}\NormalTok{, }\DataTypeTok{outside.position =} \KeywordTok{c}\NormalTok{(}\StringTok{"bottom"}\NormalTok{))}
\end{Highlighting}
\end{Shaded}

\includegraphics{Geospacial-Data_files/figure-latex/adse-1.pdf}

\begin{Shaded}
\begin{Highlighting}[]
\CommentTok{# Add fixed breaks }
\KeywordTok{tm_shape}\NormalTok{(migration) }\OperatorTok{+}
\StringTok{  }\KeywordTok{tm_raster}\NormalTok{(}\DataTypeTok{palette =}\NormalTok{ red_gray, }\DataTypeTok{style =} \StringTok{"fixed"}\NormalTok{, }
     \DataTypeTok{breaks =} \KeywordTok{c}\NormalTok{(}\OperatorTok{-}\FloatTok{5e6}\NormalTok{, }\FloatTok{-5e3}\NormalTok{, }\FloatTok{-5e2}\NormalTok{, }\FloatTok{-5e1}\NormalTok{, }\FloatTok{5e1}\NormalTok{, }\FloatTok{5e2}\NormalTok{, }\FloatTok{5e3}\NormalTok{, }\FloatTok{5e6}\NormalTok{)) }\OperatorTok{+}
\StringTok{  }\KeywordTok{tm_legend}\NormalTok{(}\DataTypeTok{outside =} \OtherTok{TRUE}\NormalTok{, }\DataTypeTok{outside.position =} \KeywordTok{c}\NormalTok{(}\StringTok{"bottom"}\NormalTok{))}
\end{Highlighting}
\end{Shaded}

\includegraphics{Geospacial-Data_files/figure-latex/adse-2.pdf}

A qualitative example Finally, let's look at an example of a categorical
variable. The land\_cover raster contains a gridded categorization of
the earth's surface. Have a look at land\_cover by printing it:

land\_cover You will notice that the values are numeric, but there are
attributes that map these numbers to categories (just like the way
factors work).

Choosing colors for categorical variables depends a lot on the purpose
of the graphic. When you want the categories to have roughly equal
visual weight -- that is, you don't want one category to stand out more
than the others -- one approach is to use colors of varying hues, but
equal chroma (a measure of vibrancy) and lightness (this is default for
discrete color scales in ggplot2 and can be generated using the hcl()
function).

The RColorBrewer qualitative palettes balance having equal visual weight
colors with ease of color identification. The ``paired'' and ``accent''
schemes deviate from this by providing pairs of colors of different
lightness and a palette with some more intense colors that may be used
to highlight certain categories, respectively.

For this particular data, it might make more sense to choose intuitive
colors, like green for forest and blue for water. Whichever is more
appropriate, setting new colors is just a matter of passing in a vector
of colors through the palette argument in the corresponding tm\_***
layer.

Instructions 100 XP Instructions 100 XP Plot the land\_cover raster by
combining tm\_shape() and tm\_raster(). By default tmap uses the
RColorBrewer ``Set3'' qualitative palette. Examine the code for
hcl\_cols, which mimics the palette used by ggplot2. Then plot the
land\_cover raster again, passing hcl\_cols to the palette argument to
tm\_raster(). Call levels() on land\_cover to see the categories. This
time, use intuitive\_cols as the palette and add a tm\_legend() layer
with the argument position = c(``left'', ``bottom'').

\begin{Shaded}
\begin{Highlighting}[]
\KeywordTok{library}\NormalTok{(raster)}

\KeywordTok{load}\NormalTok{(}\DataTypeTok{file =} \StringTok{"land_cover.rda"}\NormalTok{)}

\CommentTok{# Plot land_cover}
\KeywordTok{tm_shape}\NormalTok{(land_cover) }\OperatorTok{+}
\StringTok{  }\KeywordTok{tm_raster}\NormalTok{() }
\end{Highlighting}
\end{Shaded}

\includegraphics{Geospacial-Data_files/figure-latex/aqe-1.pdf}

\begin{Shaded}
\begin{Highlighting}[]
\CommentTok{# Palette like the ggplot2 default}
\NormalTok{hcl_cols <-}\StringTok{ }\KeywordTok{hcl}\NormalTok{(}\DataTypeTok{h =} \KeywordTok{seq}\NormalTok{(}\DecValTok{15}\NormalTok{, }\DecValTok{375}\NormalTok{, }\DataTypeTok{length =} \DecValTok{9}\NormalTok{), }
                \DataTypeTok{c =} \DecValTok{100}\NormalTok{, }\DataTypeTok{l =} \DecValTok{65}\NormalTok{)[}\OperatorTok{-}\DecValTok{9}\NormalTok{]}

\CommentTok{# Use hcl_cols as the palette}
\KeywordTok{tm_shape}\NormalTok{(land_cover) }\OperatorTok{+}
\StringTok{  }\KeywordTok{tm_raster}\NormalTok{(}\DataTypeTok{palette =}\NormalTok{ hcl_cols) }
\end{Highlighting}
\end{Shaded}

\includegraphics{Geospacial-Data_files/figure-latex/aqe-2.pdf}

\begin{Shaded}
\begin{Highlighting}[]
\CommentTok{# Examine levels of land_cover}
\KeywordTok{levels}\NormalTok{(land_cover)}
\end{Highlighting}
\end{Shaded}

\begin{verbatim}
## NULL
\end{verbatim}

\begin{Shaded}
\begin{Highlighting}[]
\CommentTok{# A set of intuitive colors}
\NormalTok{intuitive_cols <-}\StringTok{ }\KeywordTok{c}\NormalTok{(}
  \StringTok{"darkgreen"}\NormalTok{,}
  \StringTok{"darkolivegreen4"}\NormalTok{,}
  \StringTok{"goldenrod2"}\NormalTok{,}
  \StringTok{"seagreen"}\NormalTok{,}
  \StringTok{"wheat"}\NormalTok{,}
  \StringTok{"slategrey"}\NormalTok{,}
  \StringTok{"white"}\NormalTok{,}
  \StringTok{"lightskyblue1"}
\NormalTok{)}

\CommentTok{# Use intuitive_cols as palette}
\KeywordTok{tm_shape}\NormalTok{(land_cover) }\OperatorTok{+}
\StringTok{  }\KeywordTok{tm_raster}\NormalTok{(}\DataTypeTok{palette =}\NormalTok{ intuitive_cols) }\OperatorTok{+}
\StringTok{  }\KeywordTok{tm_legend}\NormalTok{(}\DataTypeTok{position =} \KeywordTok{c}\NormalTok{(}\StringTok{"left"}\NormalTok{, }\StringTok{"bottom"}\NormalTok{))}
\end{Highlighting}
\end{Shaded}

\includegraphics{Geospacial-Data_files/figure-latex/aqe-3.pdf}

\hypertarget{data-import-and-projections}{%
\subsection{Data Import and
Projections}\label{data-import-and-projections}}

Reading in a shapefile Shapefiles are one of the most common ways
spatial data are shared and are easily read into R using readOGR() from
the rgdal package. readOGR() has two important arguments: dsn and layer.
Exactly what you pass to these arguments depends on what kind of data
you are reading in. You learned in the video that for shapefiles, dsn
should be the path to the directory that holds the files that make up
the shapefile and layer is the file name of the particular shapefile
(without any extension).

For your map, you want neighborhood boundaries. We downloaded the
Neighborhood Tabulation Areas, as defined by the City of New York, from
the Open Data Platform of the Department of City Planning. The download
was in the form of a zip archive and we have put the result of unzipping
the downloaded file in your working directory.

You'll use the dir() function from base R to examine the contents of
your working directory, then read in the shapefile to R.

Instructions 100 XP Instructions 100 XP Use dir() with no arguments to
find out the name of the directory of the shapefile. Use dir(), passing
in the path to the shapefile directory, to see the files inside. Now you
know the directory and file name. Use readOGR() to read the neighborhood
shapefile into an object called neighborhoods. Check the contents by
calling summary() on neighborhoods. Check the contents by plotting
neighborhoods.

\begin{Shaded}
\begin{Highlighting}[]
\KeywordTok{library}\NormalTok{(sp)}
\KeywordTok{library}\NormalTok{(rgdal)}

\CommentTok{# https://www1.nyc.gov/site/planning/data-maps/open-data/dwn-nynta.page}

\CommentTok{# Use dir() to find directory name}
\KeywordTok{dir}\NormalTok{()}
\end{Highlighting}
\end{Shaded}

\begin{verbatim}
##  [1] "countries_sp.rda"                          
##  [2] "countries_spdf.rda"                        
##  [3] "Geospacial-Data.html"                      
##  [4] "Geospacial-Data.Rmd"                       
##  [5] "Geospacial-Data.tex"                       
##  [6] "Geospacial-Data_files"                     
##  [7] "Geospacial Data.Rmd"                       
##  [8] "land_cover.rda"                            
##  [9] "migration.rda"                             
## [10] "MooreSean_CyberAwareness2019_16OCT2019.pdf"
## [11] "nyc_grid_data"                             
## [12] "nyc_income.rda"                            
## [13] "nyc_income_map.png"                        
## [14] "nynta_19c"                                 
## [15] "nynta_19c.zip"                             
## [16] "pop.rda"                                   
## [17] "pop_by_age.rda"                            
## [18] "population.html"                           
## [19] "population.png"                            
## [20] "preds.csv"                                 
## [21] "prop_18-24.html"                           
## [22] "prop_by_age.rda"                           
## [23] "sales.csv"                                 
## [24] "ward_sales.csv"                            
## [25] "water.rda"
\end{verbatim}

\begin{Shaded}
\begin{Highlighting}[]
\CommentTok{# Call dir() with directory name}
\KeywordTok{dir}\NormalTok{(}\StringTok{"nynta_19c"}\NormalTok{)}
\end{Highlighting}
\end{Shaded}

\begin{verbatim}
## [1] "nynta.dbf"     "nynta.prj"     "nynta.shp"     "nynta.shp.xml"
## [5] "nynta.shx"
\end{verbatim}

\begin{Shaded}
\begin{Highlighting}[]
\CommentTok{# Read in shapefile with readOGR(): neighborhoods}
\NormalTok{neighborhoods <-}\StringTok{ }\KeywordTok{readOGR}\NormalTok{(}\StringTok{"nynta_19c"}\NormalTok{, }\StringTok{"nynta"}\NormalTok{)}
\end{Highlighting}
\end{Shaded}

\begin{verbatim}
## OGR data source with driver: ESRI Shapefile 
## Source: "E:\RStudio\DataCamp\Geospacial Data\nynta_19c", layer: "nynta"
## with 195 features
## It has 7 fields
\end{verbatim}

\begin{Shaded}
\begin{Highlighting}[]
\CommentTok{# summary() of neighborhoods}
\KeywordTok{summary}\NormalTok{(neighborhoods)}
\end{Highlighting}
\end{Shaded}

\begin{verbatim}
## Object of class SpatialPolygonsDataFrame
## Coordinates:
##        min       max
## x 913175.1 1067382.5
## y 120121.9  272844.3
## Is projected: TRUE 
## proj4string :
## [+proj=lcc +lat_1=40.66666666666666 +lat_2=41.03333333333333
## +lat_0=40.16666666666666 +lon_0=-74 +x_0=300000 +y_0=0
## +datum=NAD83 +units=us-ft +no_defs +ellps=GRS80 +towgs84=0,0,0]
## Data attributes:
##     BoroCode          BoroName  CountyFIPS    NTACode   
##  Min.   :1   Bronx        :38   005:38     BK09   :  1  
##  1st Qu.:2   Brooklyn     :51   047:51     BK17   :  1  
##  Median :3   Manhattan    :29   061:29     BK19   :  1  
##  Mean   :3   Queens       :58   081:58     BK21   :  1  
##  3rd Qu.:4   Staten Island:19   085:19     BK23   :  1  
##  Max.   :5                                 BK25   :  1  
##                                            (Other):189  
##                                        NTAName      Shape_Leng    
##  Airport                                   :  1   Min.   : 12022  
##  Allerton-Pelham Gardens                   :  1   1st Qu.: 23921  
##  Annadale-Huguenot-Prince's Bay-Eltingville:  1   Median : 30549  
##  Arden Heights                             :  1   Mean   : 42026  
##  Astoria                                   :  1   3rd Qu.: 41877  
##  Auburndale                                :  1   Max.   :490738  
##  (Other)                                   :189                   
##    Shape_Area       
##  Min.   :  5582283  
##  1st Qu.: 19388672  
##  Median : 32629788  
##  Mean   : 43233076  
##  3rd Qu.: 50223047  
##  Max.   :327768574  
## 
\end{verbatim}

\begin{Shaded}
\begin{Highlighting}[]
\CommentTok{# Plot neighborhoods}
\KeywordTok{plot}\NormalTok{(neighborhoods)}
\end{Highlighting}
\end{Shaded}

\includegraphics{Geospacial-Data_files/figure-latex/risf-1.pdf}

Reading in a raster file Raster files are most easily read in to R with
the raster() function from the raster package. You simply pass in the
filename (including the extension) of the raster as the first argument,
x.

The raster() function uses some native raster package functions for
reading in certain file types (based on the extension in the file name)
and otherwise hands the reading of the file on to readGDAL() from the
rgdal package. The benefit of not using readGDAL() directly is simply
that raster() returns a RasterLayer object.

A common kind of raster file is the GeoTIFF, with file extension .tif or
.tiff. We've downloaded a median income raster from the US census and
put it in your working directory.

Let's take a look and read it in.

Instructions 100 XP Instructions 100 XP Use dir() to take a look in your
working directory. Use dir() again to look inside the directory
nyc\_grid\_data. Use raster() to read in the median income raster to the
variable income\_grid by passing in the complete path to the .tif file.
Use summary() to verify the raster is stored in a RasterLayer. Use
plot() to verify the raster's contents.

\begin{Shaded}
\begin{Highlighting}[]
\KeywordTok{library}\NormalTok{(raster) }

\CommentTok{# Call dir()}
\KeywordTok{dir}\NormalTok{()}
\end{Highlighting}
\end{Shaded}

\begin{verbatim}
##  [1] "countries_sp.rda"                          
##  [2] "countries_spdf.rda"                        
##  [3] "Geospacial-Data.html"                      
##  [4] "Geospacial-Data.Rmd"                       
##  [5] "Geospacial-Data.tex"                       
##  [6] "Geospacial-Data_files"                     
##  [7] "Geospacial Data.Rmd"                       
##  [8] "land_cover.rda"                            
##  [9] "migration.rda"                             
## [10] "MooreSean_CyberAwareness2019_16OCT2019.pdf"
## [11] "nyc_grid_data"                             
## [12] "nyc_income.rda"                            
## [13] "nyc_income_map.png"                        
## [14] "nynta_19c"                                 
## [15] "nynta_19c.zip"                             
## [16] "pop.rda"                                   
## [17] "pop_by_age.rda"                            
## [18] "population.html"                           
## [19] "population.png"                            
## [20] "preds.csv"                                 
## [21] "prop_18-24.html"                           
## [22] "prop_by_age.rda"                           
## [23] "sales.csv"                                 
## [24] "ward_sales.csv"                            
## [25] "water.rda"
\end{verbatim}

\begin{Shaded}
\begin{Highlighting}[]
\CommentTok{# Call dir() on the directory}
\KeywordTok{dir}\NormalTok{(}\StringTok{"nyc_grid_data"}\NormalTok{)}
\end{Highlighting}
\end{Shaded}

\begin{verbatim}
## [1] "m5602ahhi00.tif"
\end{verbatim}

\begin{Shaded}
\begin{Highlighting}[]
\CommentTok{# Use raster() with file path: income_grid}
\NormalTok{income_grid <-}\StringTok{ }\KeywordTok{raster}\NormalTok{(}\StringTok{"nyc_grid_data/m5602ahhi00.tif"}\NormalTok{)}

\CommentTok{# Call summary() on income_grid}
\KeywordTok{summary}\NormalTok{(income_grid)}
\end{Highlighting}
\end{Shaded}

\begin{verbatim}
##         m5602ahhi00
## Min.              0
## 1st Qu.           0
## Median            0
## 3rd Qu.           0
## Max.         171435
## NA's              0
\end{verbatim}

\begin{Shaded}
\begin{Highlighting}[]
\CommentTok{# Call plot() on income_grid}
\KeywordTok{plot}\NormalTok{(income_grid)}
\end{Highlighting}
\end{Shaded}

\includegraphics{Geospacial-Data_files/figure-latex/riarf-1.pdf}

Getting data using a package Reading in spatial data from a file is one
way to get spatial data into R, but there are also some packages that
provide commonly used spatial data. For example, the rnaturalearth
package provides data from Natural Earth, a source of high resolution
world maps including coastlines, states, and populated places. In fact,
this was the source of the data from Chapter 2.

You will be examining median income at the census tract level in New
York County (a.k.a. the Bourough of Manhattan), but to do this you'll
need to know the boundaries of the census tracts. The tigris package in
R provides a way to easily download and import shapefiles based on US
Census geographies. You'll use the tracts() function to download tract
boundaries, but tigris also provides states(), counties(), places() and
many other functions that match the various levels of geographic
entities defined by the Census.

Let's grab the spatial data for the tracts.

Instructions 100 XP Call tracts() with state = ``NY'', county = ``New
York'', and cb = TRUE. Store the result in nyc\_tracts. Use summary() on
nyc\_tracts to verify the retuned object is a SpatialPolygonsDataFrame.
Plot nyc\_tracts to check the contents with plot().

\begin{Shaded}
\begin{Highlighting}[]
\KeywordTok{library}\NormalTok{(sp)}
\KeywordTok{library}\NormalTok{(tigris)}

\CommentTok{# Call tracts(): nyc_tracts}
\NormalTok{nyc_tracts <-}\StringTok{ }\KeywordTok{tracts}\NormalTok{(}\DataTypeTok{state =} \StringTok{"NY"}\NormalTok{, }\DataTypeTok{county =} \StringTok{"New York"}\NormalTok{, }\DataTypeTok{cb =} \OtherTok{TRUE}\NormalTok{)}
\end{Highlighting}
\end{Shaded}

\begin{verbatim}
## 
  |                                                                       
  |                                                                 |   0%
  |                                                                       
  |=                                                                |   1%
  |                                                                       
  |=                                                                |   2%
  |                                                                       
  |==                                                               |   3%
  |                                                                       
  |==                                                               |   4%
  |                                                                       
  |===                                                              |   5%
  |                                                                       
  |====                                                             |   6%
  |                                                                       
  |=====                                                            |   8%
  |                                                                       
  |======                                                           |  10%
  |                                                                       
  |=======                                                          |  11%
  |                                                                       
  |========                                                         |  12%
  |                                                                       
  |=========                                                        |  13%
  |                                                                       
  |=========                                                        |  14%
  |                                                                       
  |==========                                                       |  15%
  |                                                                       
  |==========                                                       |  16%
  |                                                                       
  |===========                                                      |  17%
  |                                                                       
  |===========                                                      |  18%
  |                                                                       
  |============                                                     |  19%
  |                                                                       
  |=============                                                    |  20%
  |                                                                       
  |==============                                                   |  21%
  |                                                                       
  |===============                                                  |  22%
  |                                                                       
  |===============                                                  |  23%
  |                                                                       
  |===============                                                  |  24%
  |                                                                       
  |================                                                 |  25%
  |                                                                       
  |=================                                                |  26%
  |                                                                       
  |=================                                                |  27%
  |                                                                       
  |==================                                               |  28%
  |                                                                       
  |===================                                              |  29%
  |                                                                       
  |====================                                             |  30%
  |                                                                       
  |====================                                             |  31%
  |                                                                       
  |=====================                                            |  32%
  |                                                                       
  |=====================                                            |  33%
  |                                                                       
  |======================                                           |  34%
  |                                                                       
  |=======================                                          |  35%
  |                                                                       
  |========================                                         |  36%
  |                                                                       
  |========================                                         |  37%
  |                                                                       
  |=========================                                        |  38%
  |                                                                       
  |==========================                                       |  40%
  |                                                                       
  |===========================                                      |  41%
  |                                                                       
  |===========================                                      |  42%
  |                                                                       
  |============================                                     |  43%
  |                                                                       
  |============================                                     |  44%
  |                                                                       
  |=============================                                    |  44%
  |                                                                       
  |=============================                                    |  45%
  |                                                                       
  |==============================                                   |  46%
  |                                                                       
  |==============================                                   |  47%
  |                                                                       
  |===============================                                  |  48%
  |                                                                       
  |================================                                 |  49%
  |                                                                       
  |================================                                 |  50%
  |                                                                       
  |==================================                               |  52%
  |                                                                       
  |==================================                               |  53%
  |                                                                       
  |===================================                              |  53%
  |                                                                       
  |===================================                              |  54%
  |                                                                       
  |====================================                             |  55%
  |                                                                       
  |====================================                             |  56%
  |                                                                       
  |=====================================                            |  57%
  |                                                                       
  |======================================                           |  58%
  |                                                                       
  |======================================                           |  59%
  |                                                                       
  |=======================================                          |  60%
  |                                                                       
  |=======================================                          |  61%
  |                                                                       
  |========================================                         |  61%
  |                                                                       
  |========================================                         |  62%
  |                                                                       
  |=========================================                        |  63%
  |                                                                       
  |==========================================                       |  65%
  |                                                                       
  |============================================                     |  67%
  |                                                                       
  |=============================================                    |  69%
  |                                                                       
  |=============================================                    |  70%
  |                                                                       
  |==============================================                   |  71%
  |                                                                       
  |===============================================                  |  72%
  |                                                                       
  |================================================                 |  73%
  |                                                                       
  |================================================                 |  74%
  |                                                                       
  |=================================================                |  75%
  |                                                                       
  |=================================================                |  76%
  |                                                                       
  |==================================================               |  76%
  |                                                                       
  |==================================================               |  77%
  |                                                                       
  |===================================================              |  79%
  |                                                                       
  |====================================================             |  80%
  |                                                                       
  |=====================================================            |  81%
  |                                                                       
  |=====================================================            |  82%
  |                                                                       
  |======================================================           |  82%
  |                                                                       
  |======================================================           |  83%
  |                                                                       
  |=======================================================          |  84%
  |                                                                       
  |=======================================================          |  85%
  |                                                                       
  |========================================================         |  86%
  |                                                                       
  |=========================================================        |  88%
  |                                                                       
  |==========================================================       |  89%
  |                                                                       
  |==========================================================       |  90%
  |                                                                       
  |===========================================================      |  90%
  |                                                                       
  |===========================================================      |  91%
  |                                                                       
  |===========================================================      |  92%
  |                                                                       
  |============================================================     |  92%
  |                                                                       
  |=============================================================    |  93%
  |                                                                       
  |=============================================================    |  94%
  |                                                                       
  |==============================================================   |  95%
  |                                                                       
  |===============================================================  |  96%
  |                                                                       
  |===============================================================  |  97%
  |                                                                       
  |================================================================ |  98%
  |                                                                       
  |================================================================ |  99%
  |                                                                       
  |=================================================================| 100%
\end{verbatim}

\begin{Shaded}
\begin{Highlighting}[]
\CommentTok{# Call summary() on nyc_tracts}
\KeywordTok{summary}\NormalTok{(nyc_tracts)}
\end{Highlighting}
\end{Shaded}

\begin{verbatim}
## Object of class SpatialPolygonsDataFrame
## Coordinates:
##         min       max
## x -74.04731 -73.90700
## y  40.68419  40.88207
## Is projected: FALSE 
## proj4string :
## [+proj=longlat +datum=NAD83 +no_defs +ellps=GRS80 +towgs84=0,0,0]
## Data attributes:
##    STATEFP            COUNTYFP           TRACTCE         
##  Length:288         Length:288         Length:288        
##  Class :character   Class :character   Class :character  
##  Mode  :character   Mode  :character   Mode  :character  
##    AFFGEOID            GEOID               NAME          
##  Length:288         Length:288         Length:288        
##  Class :character   Class :character   Class :character  
##  Mode  :character   Mode  :character   Mode  :character  
##      LSAD              ALAND              AWATER         
##  Length:288         Length:288         Length:288        
##  Class :character   Class :character   Class :character  
##  Mode  :character   Mode  :character   Mode  :character
\end{verbatim}

\begin{Shaded}
\begin{Highlighting}[]
\CommentTok{# Plot nyc_tracts}
\KeywordTok{plot}\NormalTok{(nyc_tracts)}
\end{Highlighting}
\end{Shaded}

\includegraphics{Geospacial-Data_files/figure-latex/gduap-1.pdf}

Merging data from different CRS/projections Every spatial object has a
coordinate reference system (CRS) associated with it. Generally, this is
set when the data are imported and will be read directly from the
spatial files. This is how the neighborhoods and nyc\_tracts obtained
their coordinate system information.

Both the sp and raster packages have a proj4string() function that
returns the CRS of the object it's called on.

Trying to work with spatial data using different CRSs is a bit like
trying to work with a dataset in miles and another in kilometers. They
are measuring the same thing, but the numbers aren't directly
comparable.

Let's take a look at our two polygon objects.

Instructions 100 XP Call proj4string() on neighborhoods, then again on
nyc\_tracts. Verify the two strings are different. Take a look at the
head() of the coordinates() of neighborhoods and repeat for nyc\_tracts.
Can you see the problem? nyc\_tracts has x coordinates around -70, but
neighborhoods is around 1,000,000! Plot neighborhoods, then plot
nyc\_tracts with col = ``red'' and add = TRUE to add them on top.

\begin{Shaded}
\begin{Highlighting}[]
\KeywordTok{library}\NormalTok{(sp)}

\CommentTok{# proj4string() on nyc_tracts and neighborhoods}
\KeywordTok{proj4string}\NormalTok{(nyc_tracts)}
\end{Highlighting}
\end{Shaded}

\begin{verbatim}
## [1] "+proj=longlat +datum=NAD83 +no_defs +ellps=GRS80 +towgs84=0,0,0"
\end{verbatim}

\begin{Shaded}
\begin{Highlighting}[]
\KeywordTok{proj4string}\NormalTok{(neighborhoods)}
\end{Highlighting}
\end{Shaded}

\begin{verbatim}
## [1] "+proj=lcc +lat_1=40.66666666666666 +lat_2=41.03333333333333 +lat_0=40.16666666666666 +lon_0=-74 +x_0=300000 +y_0=0 +datum=NAD83 +units=us-ft +no_defs +ellps=GRS80 +towgs84=0,0,0"
\end{verbatim}

\begin{Shaded}
\begin{Highlighting}[]
\CommentTok{# coordinates() on nyc_tracts and neighborhoods}
\KeywordTok{head}\NormalTok{(}\KeywordTok{coordinates}\NormalTok{(nyc_tracts))}
\end{Highlighting}
\end{Shaded}

\begin{verbatim}
##          [,1]     [,2]
## 150 -73.98916 40.71012
## 151 -73.99326 40.71595
## 152 -74.00381 40.71532
## 153 -74.00283 40.72241
## 154 -73.97923 40.74007
## 155 -74.00445 40.74220
\end{verbatim}

\begin{Shaded}
\begin{Highlighting}[]
\KeywordTok{head}\NormalTok{(}\KeywordTok{coordinates}\NormalTok{(neighborhoods))}
\end{Highlighting}
\end{Shaded}

\begin{verbatim}
##        [,1]     [,2]
## 0  987397.5 169148.4
## 1 1037005.2 219265.3
## 2 1020706.7 217413.9
## 3 1050471.5 198271.8
## 4  997042.7 237291.1
## 5 1047209.5 184218.6
\end{verbatim}

\begin{Shaded}
\begin{Highlighting}[]
\CommentTok{# plot() neighborhoods and nyc_tracts}
\KeywordTok{plot}\NormalTok{(neighborhoods)}
\KeywordTok{plot}\NormalTok{(nyc_tracts, }\DataTypeTok{add =} \OtherTok{TRUE}\NormalTok{, }\DataTypeTok{col =} \StringTok{"red"}\NormalTok{)}
\end{Highlighting}
\end{Shaded}

\includegraphics{Geospacial-Data_files/figure-latex/mdfdCRS-1.pdf}

Converting from one CRS/projection to another The process of converting
from one CRS or projection to another is handled by the spTransform()
methods in the rgdal package. spTransform() has methods for all sp
objects including SpatialPolygonsDataFrame, but doesn't work on raster
objects. This is because transforming a raster is a little more
complicated; the transformed rectangular grid will no longer be
rectangular. You can look at ?raster::projectRaster if you are curious
about transforming rasters.

Transformation is simple. The first argument to spTransform(), x, is the
spatial object to be transformed and the second, CRS, is a specification
of the desired CRS. The CRS can be specified by a PROJ4 string, which
you could construct by hand, but it's much easier to take it from an
existing object (e.g.~with the proj4string() function).

Time to get your two polygons datasets into the same CRS.

Instructions 100 XP Transform neighborhoods to have the same CRS as
nyc\_tracts by using spTransform() with the CRS argument set to
proj4string(nyc\_tracts). Verify the transformation by looking at the
head() of coordinates(neighborhoods). Check the datasets now line up by
plotting neighborhoods, then plotting nyc\_tracts with add = TRUE and
col = ``red'', and finally plotting water with add = TRUE and col =
``blue''.

\begin{Shaded}
\begin{Highlighting}[]
\KeywordTok{library}\NormalTok{(sp)}
\KeywordTok{library}\NormalTok{(raster)}

\CommentTok{# Use spTransform on neighborhoods: neighborhoods}
\NormalTok{neighborhoods <-}\StringTok{ }\KeywordTok{spTransform}\NormalTok{(neighborhoods,   }
                             \KeywordTok{proj4string}\NormalTok{(nyc_tracts))}

\CommentTok{# head() on coordinates() of neighborhoods}
\KeywordTok{head}\NormalTok{(}\KeywordTok{coordinates}\NormalTok{(neighborhoods))}
\end{Highlighting}
\end{Shaded}

\begin{verbatim}
##        [,1]     [,2]
## 0 -73.98866 40.63095
## 1 -73.80955 40.76835
## 2 -73.86840 40.76335
## 3 -73.76114 40.71064
## 4 -73.95378 40.81798
## 5 -73.77303 40.67209
\end{verbatim}

\begin{Shaded}
\begin{Highlighting}[]
\CommentTok{# Plot neighborhoods, nyc_tracts and water}
\KeywordTok{plot}\NormalTok{(neighborhoods)}
\KeywordTok{plot}\NormalTok{(nyc_tracts, }\DataTypeTok{add =} \OtherTok{TRUE}\NormalTok{, }\DataTypeTok{col =} \StringTok{"red"}\NormalTok{)}
\KeywordTok{load}\NormalTok{(}\DataTypeTok{file =} \StringTok{"water.rda"}\NormalTok{)}
\KeywordTok{plot}\NormalTok{(water, }\DataTypeTok{add =} \OtherTok{TRUE}\NormalTok{, }\DataTypeTok{col =} \StringTok{"blue"}\NormalTok{)}
\end{Highlighting}
\end{Shaded}

\includegraphics{Geospacial-Data_files/figure-latex/cfoCRSta-1.pdf}

The wrong way When a Spatial***DataFrame object is created, there are
two ways the spatial objects (e.g.~Polygons) might be matched up to the
rows of the data. The most robust is to use IDs on the spatial objects
that are matched up to row names in the data. This ensures if there are
any that don't match you are quickly alerted. The other way is simply by
order -- the first spatial object is assumed to correspond to the first
row of data.

Once created, the correspondence is based purely on order. If you
manipulate the data slot, there is no checking the spatial objects still
correspond to the right rows. What does this mean in practice? It's very
dangerous to manipulate the data slot directly!

To create your plot of income, you need to match up the income data
frame with the tracts SpatialPolygonsDataFrame. To illustrate the danger
of manipulating @data directly, let's see what happens if you try to
force nyc\_income in to nyc\_tracts.

Instructions 1/2 30 XP 1 2 Use str() to look at nyc\_income. Do the same
for the data slot of nyc\_tracts. They both have the same number of
rows, with information about the same tracts (tract in nyc\_income and
TRACTCE in nyc\_tracts), but in different orders.

\begin{Shaded}
\begin{Highlighting}[]
\KeywordTok{library}\NormalTok{(sp)}

\KeywordTok{load}\NormalTok{(}\DataTypeTok{file =} \StringTok{"nyc_income.rda"}\NormalTok{)}

\CommentTok{# Use str() on nyc_income }
\KeywordTok{str}\NormalTok{(nyc_income)}
\end{Highlighting}
\end{Shaded}

\begin{verbatim}
## 'data.frame':    288 obs. of  6 variables:
##  $ name    : chr  "Census Tract 1, New York County, New York" "Census Tract 2.01, New York County, New York" "Census Tract 2.02, New York County, New York" "Census Tract 5, New York County, New York" ...
##  $ state   : int  36 36 36 36 36 36 36 36 36 36 ...
##  $ county  : int  61 61 61 61 61 61 61 61 61 61 ...
##  $ tract   : chr  "000100" "000201" "000202" "000500" ...
##  $ estimate: num  NA 23036 29418 NA 18944 ...
##  $ se      : num  NA 3083 1877 NA 1442 ...
\end{verbatim}

\begin{Shaded}
\begin{Highlighting}[]
\CommentTok{# ...and on nyc_tracts@data}
\KeywordTok{str}\NormalTok{(nyc_tracts}\OperatorTok{@}\NormalTok{data)}
\end{Highlighting}
\end{Shaded}

\begin{verbatim}
## 'data.frame':    288 obs. of  9 variables:
##  $ STATEFP : chr  "36" "36" "36" "36" ...
##  $ COUNTYFP: chr  "061" "061" "061" "061" ...
##  $ TRACTCE : chr  "000600" "001600" "003100" "004700" ...
##  $ AFFGEOID: chr  "1400000US36061000600" "1400000US36061001600" "1400000US36061003100" "1400000US36061004700" ...
##  $ GEOID   : chr  "36061000600" "36061001600" "36061003100" "36061004700" ...
##  $ NAME    : chr  "6" "16" "31" "47" ...
##  $ LSAD    : chr  "CT" "CT" "CT" "CT" ...
##  $ ALAND   : chr  "240406" "207377" "204969" "165421" ...
##  $ AWATER  : chr  "176018" "0" "0" "0" ...
\end{verbatim}

\begin{Shaded}
\begin{Highlighting}[]
\CommentTok{# Highlight tract 002201 in nyc_tracts}
\KeywordTok{plot}\NormalTok{(nyc_tracts)}
\KeywordTok{plot}\NormalTok{(nyc_tracts[nyc_tracts}\OperatorTok{$}\NormalTok{TRACTCE }\OperatorTok{==}\StringTok{ "002201"}\NormalTok{, ], }
     \DataTypeTok{col =} \StringTok{"red"}\NormalTok{, }\DataTypeTok{add =} \OtherTok{TRUE}\NormalTok{)}
\end{Highlighting}
\end{Shaded}

\includegraphics{Geospacial-Data_files/figure-latex/tww-1.pdf}

\begin{Shaded}
\begin{Highlighting}[]
\CommentTok{# Set nyc_tracts@data to nyc_income}
\NormalTok{nyc_tracts}\OperatorTok{@}\NormalTok{data <-}\StringTok{ }\NormalTok{nyc_income}

\CommentTok{# Highlight tract 002201 again}
\KeywordTok{plot}\NormalTok{(nyc_tracts)}
\KeywordTok{plot}\NormalTok{(nyc_tracts[nyc_tracts}\OperatorTok{$}\NormalTok{tract }\OperatorTok{==}\StringTok{ "002201"}\NormalTok{, ], }
     \DataTypeTok{col =} \StringTok{"red"}\NormalTok{, }\DataTypeTok{add =} \OtherTok{TRUE}\NormalTok{)}
\end{Highlighting}
\end{Shaded}

\includegraphics{Geospacial-Data_files/figure-latex/tww-2.pdf}

Checking data will match Forcing your data into the data slot doesn't
work because you lose the correct correspondence between rows and
spatial objects. How do you add the income data to the polygon data? The
merge() function in sp is designed exactly for this purpose.

You might have seen merge() before with data frames. sp::merge() has
almost the exact same structure, but you pass it a Spatial*** object and
a data frame and it returns a new Spatial*** object where the data slot
is now a merge of the original data slot and the data frame. To do this
merge, you'll require both the spatial object and data frame to have a
column that contains IDs to match on.

Both nyc\_tracts and nyc\_income have columns with tract IDs, so these
are great candidates for merging the two datasets. However, it's always
a good idea to check that the proposed IDs are unique and that there is
a match for every row in both datasets.

Let's check this before moving on to the merge.

Instructions 100 XP Use any() with duplicated() on
nyc\_income\(tract to check if every row in nyc_income has a unique tract ID. Use any() with duplicated() on nyc_tracts\)TRACTCE
to check if every row in nyc\_tracts has a unique tract ID. Use all() on
nyc\_tracts\(TRACTCE %in% nyc_income
\)tract to check the nyc\_tracts tracts are all in nyc\_income. Use
all() on nyc\_income\(tract %in% nyc_tracts
\)TRACTCE to check the nyc\_income tracts are all in nyc\_tracts.

\begin{Shaded}
\begin{Highlighting}[]
\CommentTok{# Check for duplicates in nyc_income}
\KeywordTok{any}\NormalTok{(}\KeywordTok{duplicated}\NormalTok{(nyc_income}\OperatorTok{$}\NormalTok{tract))}
\end{Highlighting}
\end{Shaded}

\begin{verbatim}
## [1] FALSE
\end{verbatim}

\begin{Shaded}
\begin{Highlighting}[]
\CommentTok{# Check for duplicates in nyc_tracts}
\KeywordTok{any}\NormalTok{(}\KeywordTok{duplicated}\NormalTok{(nyc_tracts}\OperatorTok{$}\NormalTok{TRACTCE))}
\end{Highlighting}
\end{Shaded}

\begin{verbatim}
## [1] FALSE
\end{verbatim}

\begin{Shaded}
\begin{Highlighting}[]
\CommentTok{# Check nyc_tracts in nyc_income}
\KeywordTok{all}\NormalTok{((nyc_tracts}\OperatorTok{$}\NormalTok{TRACTCE }\OperatorTok\StringTok{ }\NormalTok{nyc_income}\OperatorTok{$}\NormalTok{tract))}
\end{Highlighting}
\end{Shaded}

\begin{verbatim}
## [1] TRUE
\end{verbatim}

\begin{Shaded}
\begin{Highlighting}[]
\CommentTok{# Check nyc_income in nyc_tracts}
\KeywordTok{all}\NormalTok{((nyc_income}\OperatorTok{$}\NormalTok{tract }\OperatorTok\StringTok{ }\NormalTok{nyc_tracts}\OperatorTok{$}\NormalTok{TRACTCE))}
\end{Highlighting}
\end{Shaded}

\begin{verbatim}
## [1] FALSE
\end{verbatim}

Merging data attributes merge() by default merges based on columns with
the same name in both datasets. In your case, this isn't appropriate
since the column of IDs is called tract in one dataset and TRACTCE in
the other. To handle this, merge() has the optional arguments by.x and
by.y, where you can specify the names of the column to merge on in the
two datasets, respectively.

merge() returns a new Spatial\_\_\_DataFrame object, so you can take a
look at the result by plotting it with tmap.

Let's go ahead and merge.

Instructions 100 XP Use merge(), passing the spatial object nyc\_tracts
first and the data frame nyc\_income second. Specify by.x = ``TRACTCE''
and by.y = ``tract''. Store the result in nyc\_tracts\_merge. Use
summary() on nyc\_tracts\_merge to verify the object is a
SpatialPolygonsDataFrame and the data also contain the needed estimate
column from nyc\_income. Use tm\_shape() and add a tm\_fill() layer to
create a choropleth map of nyc\_tracts\_merge, mapping color to
estimate.

\begin{Shaded}
\begin{Highlighting}[]
\KeywordTok{library}\NormalTok{(sp)}
\KeywordTok{library}\NormalTok{(tmap)}

\CommentTok{# Merge nyc_tracts and nyc_income: nyc_tracts_merge}
\NormalTok{nyc_tracts_merge <-}\StringTok{ }\KeywordTok{merge}\NormalTok{(nyc_tracts, nyc_income, }\DataTypeTok{by.x =} \StringTok{"tract"}\NormalTok{, }\DataTypeTok{by.y =} \StringTok{"tract"}\NormalTok{)}

\CommentTok{# Call summary() on nyc_tracts_merge}
\KeywordTok{summary}\NormalTok{(nyc_tracts_merge)}
\end{Highlighting}
\end{Shaded}

\begin{verbatim}
## Object of class SpatialPolygonsDataFrame
## Coordinates:
##         min       max
## x -74.04731 -73.90700
## y  40.68419  40.88207
## Is projected: FALSE 
## proj4string :
## [+proj=longlat +datum=NAD83 +no_defs +ellps=GRS80 +towgs84=0,0,0]
## Data attributes:
##     tract              name.x             state.x      county.x 
##  Length:288         Length:288         Min.   :36   Min.   :61  
##  Class :character   Class :character   1st Qu.:36   1st Qu.:61  
##  Mode  :character   Mode  :character   Median :36   Median :61  
##                                        Mean   :36   Mean   :61  
##                                        3rd Qu.:36   3rd Qu.:61  
##                                        Max.   :36   Max.   :61  
##                                                                 
##    estimate.x          se.x           name.y             state.y  
##  Min.   : 12479   Min.   :  1117   Length:288         Min.   :36  
##  1st Qu.: 39038   1st Qu.:  5107   Class :character   1st Qu.:36  
##  Median : 81786   Median :  8998   Mode  :character   Median :36  
##  Mean   : 82405   Mean   : 11678                      Mean   :36  
##  3rd Qu.:112561   3rd Qu.: 14281                      3rd Qu.:36  
##  Max.   :232266   Max.   :132737                      Max.   :36  
##  NA's   :9        NA's   :9                                       
##     county.y    estimate.y          se.y       
##  Min.   :61   Min.   : 12479   Min.   :  1117  
##  1st Qu.:61   1st Qu.: 39038   1st Qu.:  5107  
##  Median :61   Median : 81786   Median :  8998  
##  Mean   :61   Mean   : 82405   Mean   : 11678  
##  3rd Qu.:61   3rd Qu.:112561   3rd Qu.: 14281  
##  Max.   :61   Max.   :232266   Max.   :132737  
##               NA's   :9        NA's   :9
\end{verbatim}

\begin{Shaded}
\begin{Highlighting}[]
\CommentTok{# Choropleth with col mapped to estimate}
\KeywordTok{tm_shape}\NormalTok{(nyc_tracts_merge) }\OperatorTok{+}
\StringTok{  }\KeywordTok{tm_fill}\NormalTok{(}\DataTypeTok{col =} \StringTok{"estimate.x"}\NormalTok{) }
\end{Highlighting}
\end{Shaded}

\includegraphics{Geospacial-Data_files/figure-latex/mda-1.pdf}

A first plot So far, you've read in some spatial files, transformed
spatial data to the same projection, and merged a data frame with a
spatial object. Time to put your work together and see how your map
looks. For each dataset, you need a tm\_shape() call to specify the data
source, followed by a tm\_* layer (like tm\_fill(), tm\_borders() or
tm\_bubbles()) to draw on the map.

First, you'll add the neighborhoods and water areas to your plot from
the previous exercise.

Instructions 100 XP Add a layer for the water object with tm\_shape().
Then use tm\_fill() and set the color to ``grey90''. Similarly, add a
layer for the neighborhoods object. Use tm\_borders() to draw the
neighborhood outlines.

\begin{Shaded}
\begin{Highlighting}[]
\KeywordTok{library}\NormalTok{(tmap)}

\KeywordTok{tm_shape}\NormalTok{(nyc_tracts_merge) }\OperatorTok{+}
\StringTok{  }\KeywordTok{tm_fill}\NormalTok{(}\DataTypeTok{col =} \StringTok{"estimate.x"}\NormalTok{) }\OperatorTok{+}
\StringTok{  }\CommentTok{# Add a water layer, tm_fill() with col = "grey90"}
\StringTok{  }\KeywordTok{tm_shape}\NormalTok{(water) }\OperatorTok{+}
\StringTok{  }\KeywordTok{tm_fill}\NormalTok{(}\DataTypeTok{col =} \StringTok{"grey90"}\NormalTok{) }\OperatorTok{+}
\StringTok{  }\CommentTok{# Add a neighborhood layer, tm_borders()}
\StringTok{  }\KeywordTok{tm_shape}\NormalTok{(neighborhoods) }\OperatorTok{+}
\StringTok{  }\KeywordTok{tm_borders}\NormalTok{() }
\end{Highlighting}
\end{Shaded}

\includegraphics{Geospacial-Data_files/figure-latex/afp-1.pdf}

Subsetting the neighborhoods You don't need all those extraneous
neighborhoods in New York, so you'll subset out just the neighborhoods
in New York County. You already know how!

neighborhoods is a SpatialPolygonsDataFrame and you learned back in
Chapter 2 how to subset based on the column in the data slot. The key
was creating a logical vector, then subsetting the
SpatialPolygonsDataFrame like a data frame.

How can you identify the right neighborhoods? Check out:

head(\href{mailto:neighborhoods@data}{\nolinkurl{neighborhoods@data}})
The CountyFIPS is a numeric code that identifies the county. If you can
figure out the code for New York County, you can keep just the rows with
that value.

Instructions 100 XP The nyc\_tracts\_merge object also has country codes
in the column COUNTYFP. Find the unique() values to find the code for
New York County. Subset neighborhoods by adding a logical that tests if
neighborhoods\$CountyFIPS has the right value. Edit your plot to use
manhat\_hoods instead of neighborhoods. Add a tm\_text() layer, mapping
text to ``NTAName''.

\begin{Shaded}
\begin{Highlighting}[]
\KeywordTok{library}\NormalTok{(tmap)}

\CommentTok{# Find unique() nyc_tracts_merge$COUNTYFP}
\KeywordTok{unique}\NormalTok{(nyc_tracts_merge}\OperatorTok{$}\NormalTok{COUNTYFP)}
\end{Highlighting}
\end{Shaded}

\begin{verbatim}
## NULL
\end{verbatim}

\begin{Shaded}
\begin{Highlighting}[]
\CommentTok{# Add logical expression to pull out New York County}
\NormalTok{manhat_hoods <-}\StringTok{ }\NormalTok{neighborhoods[neighborhoods}\OperatorTok{$}\NormalTok{CountyFIPS }\OperatorTok{==}\StringTok{ "061"}\NormalTok{, ]}

\KeywordTok{tm_shape}\NormalTok{(nyc_tracts_merge) }\OperatorTok{+}
\StringTok{  }\KeywordTok{tm_fill}\NormalTok{(}\DataTypeTok{col =} \StringTok{"estimate.x"}\NormalTok{) }\OperatorTok{+}
\StringTok{  }\KeywordTok{tm_shape}\NormalTok{(water) }\OperatorTok{+}
\StringTok{  }\KeywordTok{tm_fill}\NormalTok{(}\DataTypeTok{col =} \StringTok{"grey90"}\NormalTok{) }\OperatorTok{+}
\StringTok{  }\CommentTok{# Edit to use manhat_hoods instead}
\StringTok{  }\KeywordTok{tm_shape}\NormalTok{(manhat_hoods) }\OperatorTok{+}
\StringTok{  }\KeywordTok{tm_borders}\NormalTok{() }\OperatorTok{+}
\StringTok{  }\CommentTok{# Add a tm_text() layer}
\StringTok{  }\KeywordTok{tm_text}\NormalTok{(}\DataTypeTok{text =} \StringTok{"NTAName"}\NormalTok{)}
\end{Highlighting}
\end{Shaded}

\includegraphics{Geospacial-Data_files/figure-latex/stn-1.pdf}

Adding neighborhood labels The neighborhood labels are so long and big
they are obscuring our data. Take a look at manhat\_hoods\$NTAName.
You'll see some neighborhoods are really the combination of a couple of
places. One option to make the names a little more concise is to split
them into a few lines. For example, turning

Midtown-Midtown South into

Midtown / Midtown South To do this, you can make use of the gsub()
function in base R. gsub() replaces the first argument by the second
argument in the strings provided in the third argument. For example,
gsub(``a'', ``A'', x) replaces all the ``a''s in x with ``A''.

You also might play with the size of the text to shrink the impact of
the neighborhood names.

Instructions 100 XP Create a new column name in manhat\_hoods by using
gsub() to replace all the spaces (" ``) with newlines (''\n``) in
manhat\_hoods\(NTAName. Update name in manhat_hoods by using gsub() to replace all the dashes ("-") with a forward slash then newline ("/\n") in manhat_hoods\)name.
Edit your plot to map text to''name" and set the size to 0.5.

\begin{Shaded}
\begin{Highlighting}[]
\KeywordTok{library}\NormalTok{(tmap)}

\CommentTok{# gsub() to replace " " with "\textbackslash{}n"}
\NormalTok{manhat_hoods}\OperatorTok{$}\NormalTok{name <-}\StringTok{ }\KeywordTok{gsub}\NormalTok{(}\StringTok{" "}\NormalTok{, }\StringTok{"}\CharTok{\textbackslash{}n}\StringTok{"}\NormalTok{, manhat_hoods}\OperatorTok{$}\NormalTok{NTAName)}

\CommentTok{# gsub() to replace "-" with "/\textbackslash{}n"}
\NormalTok{manhat_hoods}\OperatorTok{$}\NormalTok{name <-}\StringTok{ }\KeywordTok{gsub}\NormalTok{(}\StringTok{"-"}\NormalTok{, }\StringTok{"/}\CharTok{\textbackslash{}n}\StringTok{"}\NormalTok{, manhat_hoods}\OperatorTok{$}\NormalTok{name)}

\CommentTok{# Edit to map text to name, set size to 0.5}
\KeywordTok{tm_shape}\NormalTok{(nyc_tracts_merge) }\OperatorTok{+}
\StringTok{    }\KeywordTok{tm_fill}\NormalTok{(}\DataTypeTok{col =} \StringTok{"estimate.x"}\NormalTok{) }\OperatorTok{+}
\StringTok{  }\KeywordTok{tm_shape}\NormalTok{(water) }\OperatorTok{+}
\StringTok{    }\KeywordTok{tm_fill}\NormalTok{(}\DataTypeTok{col =} \StringTok{"grey90"}\NormalTok{) }\OperatorTok{+}
\StringTok{  }\KeywordTok{tm_shape}\NormalTok{(manhat_hoods) }\OperatorTok{+}
\StringTok{    }\KeywordTok{tm_borders}\NormalTok{() }\OperatorTok{+}
\StringTok{    }\KeywordTok{tm_text}\NormalTok{(}\DataTypeTok{text =} \StringTok{"name"}\NormalTok{, }\DataTypeTok{size =} \FloatTok{0.5}\NormalTok{)}
\end{Highlighting}
\end{Shaded}

\includegraphics{Geospacial-Data_files/figure-latex/anl-1.pdf}

Tidying up the legend and some final tweaks Time for some final tweaks
and then to save your plot.

Every element in your plot is a target for tweaks. Is it the right
color? Is it the right size? Does it have intuitive labels? Your goal is
to emphasize the data and de-emphasise the non-data elements.

We've got some ideas for this plot. Let's tweak a few things.

Instructions 100 XP Make it clear what the color represents by adding
title = ``Median Income'' and palette = ``Greens'' in the tm\_fill()
call, which will map income to a green color scale. Add subtle borders
to the tracts to make it more clear where their boundaries are by adding
a tm\_borders() layer with col = ``grey60'' and lwd = 0.5. Make the
neighborhood boundaries a little more important than tract boundaries by
setting col = ``grey40'' and lwd = 2. Add a data source credit using a
tm\_credits() call with first argument ``Source: ACS 2014 5-year
Estimates, \n accessed via acs package'' and second argument position =
c(``right'', ``bottom''). Finally, save your plot as
``nyc\_income\_map.png'' using the save\_tmap() function with arguments
width = 4 and height = 7.

\begin{Shaded}
\begin{Highlighting}[]
\KeywordTok{library}\NormalTok{(tmap)}

\KeywordTok{tm_shape}\NormalTok{(nyc_tracts_merge) }\OperatorTok{+}
\StringTok{  }\CommentTok{# Add title and change palette}
\StringTok{  }\KeywordTok{tm_fill}\NormalTok{(}\DataTypeTok{col =} \StringTok{"estimate.x"}\NormalTok{, }
          \DataTypeTok{title =} \StringTok{"Median Income"}\NormalTok{,}
          \DataTypeTok{palette =} \StringTok{"Greens"}\NormalTok{) }\OperatorTok{+}
\StringTok{  }\CommentTok{# Add tm_borders()}
\StringTok{  }\KeywordTok{tm_borders}\NormalTok{(}\DataTypeTok{col =} \StringTok{"grey60"}\NormalTok{, }\DataTypeTok{lwd =} \FloatTok{0.5}\NormalTok{) }\OperatorTok{+}
\StringTok{  }\KeywordTok{tm_shape}\NormalTok{(water) }\OperatorTok{+}
\StringTok{  }\KeywordTok{tm_fill}\NormalTok{(}\DataTypeTok{col =} \StringTok{"grey90"}\NormalTok{) }\OperatorTok{+}
\StringTok{  }\KeywordTok{tm_shape}\NormalTok{(manhat_hoods) }\OperatorTok{+}
\StringTok{  }\CommentTok{# Change col and lwd of neighborhood boundaries}
\StringTok{  }\KeywordTok{tm_borders}\NormalTok{(}\DataTypeTok{col =} \StringTok{"grey40"}\NormalTok{, }\DataTypeTok{lwd =} \DecValTok{2}\NormalTok{) }\OperatorTok{+}
\StringTok{  }\KeywordTok{tm_text}\NormalTok{(}\DataTypeTok{text =} \StringTok{"name"}\NormalTok{, }\DataTypeTok{size =} \FloatTok{0.5}\NormalTok{) }\OperatorTok{+}
\StringTok{  }\CommentTok{# Add tm_credits()}
\StringTok{  }\KeywordTok{tm_credits}\NormalTok{(}\StringTok{"Source: ACS 2014 5-year Estimates, }\CharTok{\textbackslash{}n}\StringTok{ accessed via acs package"}\NormalTok{, }
             \DataTypeTok{position =} \KeywordTok{c}\NormalTok{(}\StringTok{"right"}\NormalTok{, }\StringTok{"bottom"}\NormalTok{))}
\end{Highlighting}
\end{Shaded}

\includegraphics{Geospacial-Data_files/figure-latex/tutl-1.pdf}

\begin{Shaded}
\begin{Highlighting}[]
\CommentTok{# Save map as "nyc_income_map.png"}
\KeywordTok{save_tmap}\NormalTok{(}\DataTypeTok{filename =} \StringTok{"nyc_income_map.png"}\NormalTok{, }\DataTypeTok{width =} \DecValTok{4}\NormalTok{, }\DataTypeTok{height =} \DecValTok{7}\NormalTok{)}
\end{Highlighting}
\end{Shaded}


\end{document}
